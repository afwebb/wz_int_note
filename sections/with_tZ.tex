
\subsubsection{tZ Inclusive Fit}
\label{sec:inc_tZ}

While tZ is often considered as a distinct process from WZ + b, this could also be considered part of the signal. Alternative studies are performed where, using the same framework as the nominal analysis, a measurement of WZ + b is performed that includes tZ as part of WZ+b. 

Because of this change, the tZ CR is no longer necessary, and only the five pseudo-continuous b-tag regions are used in the fit. Further, systematics related to the tZ cross-section are removed from the fit, as they are now encompassed by the normalization measurement of WZ + b. All other systematic uncertainties are carried over from the nominal analysis.

%A post-fit summary of the fit regions are shown in Figure \ref{fig:tZ_inc_1j}.

%\begin{figure}[H]
%  \center
%  \includegraphics[width=.9\linewidth]{sys_tZsig/Plots/Summary_postFit.png}
%  \caption{Post-fit summary of the fit regions.}
%  \label{fig:tZ_inc_1j}
%\end{figure}

An expected WZ + b cross-section of $4.1^{+0.78}_{-0.74} (stat)^{+0.53}_{-0.52}(sys)$ fb is extracted from the fit, with an expected significance of 4.0$\sigma$.

The impact of the predominate systematics are summarized in Table \ref{tab:systematics_tZ1j}.

\begin{table}[H]                                                                                                             
    \centering                                                                                                               
    \begin{tabular}{l|cc}
        \hline\hline
        Uncertainty Source & \multicolumn{2}{c}{$\Delta \mu$ }  \\
        \hline
        WZ + light cross-section & 0.08 & -0.08 \\
        Jet Energy Scale & -0.06 & 0.08 \\
        Luminosity & -0.05 & 0.06 \\
        WZ + charm cross-section & -0.04 & 0.05 \\
        Other Diboson + b cross-section & -0.04 & 0.04 \\
        WZ cross-section - QCD scale & -0.04 & 0.03 \\
        $t\bar{t}$ cross-section & -0.03 & 0.03 \\
        Jet Energy Resolution & -0.03 & 0.03 \\
        Flavor tagging & -0.03 & 0.03 \\
        Z+jets cross section & -0.02 & 0.02 \\
        \hline
        Total Systematic Uncertainty & -0.15 & 0.16 \\
        \hline\hline
  \end{tabular}
  \caption{Summary of the most significant sources of systematic uncertainty on the measurement of $WZ+b$ with exactly one associated jet.}
  \label{tab:systematics_tZ1j}
\end{table}

\subsubsection{Floating tZ}
\label{sec:float_tZ}

In order to quantify the impact of the tZ uncertainty on the fit, an alternative fit strategy is used where the tZ normalization is allowed to float. This normalization factor replaces the cross-section uncertainty on tZ, and all other parameters of the fit remain the same.

An uncertainty of 17\% on the normalization of tZ is extracted from the fit, compared to a theory uncertainty of 15\% applied to the tZ cross-section. The measured uncertainties on WZ remain the same.
