
The production of $WZ$ in association with a heavy flavor jet represents an important background for many major analyses. This includes any process with leptons and b-jets in the final state, such as $t\bar{t}H$, $t\bar{t}W$, and $t\bar{t}Z$. While precise measurements have been made of $WZ$ production \cite{WZ_36}, $WZ$ + heavy flavor remains poorly understood. This is largely because the QCD processes involved in the production of the b-jet make it difficult to simulate accurately. This introduces a large uncertainty for analyses that include this process as a background.

Motivated by its relevance to the $t\bar{t}H$ multilepton analysis, we perform a study of the fully leptonic decay mode of this channel; that is, events where both the W and Z decay leptonically. Because WZ has no associated jets at leading order, while the major backgrounds for this channel tend to have high jet multiplicity, events with more than two jets are rejcted. This gives a final state signature of three leptons and one or two jets.

Events that meet this selection criteria are sorted into pseudo-continuous b-tagging regions based on the DL1r b-tag score of their associated jets. This is done to separate $WZ$ + b-jet events from $WZ$ + charm and $WZ$ + light jets. These regions are fit to data in order make a more accurate estimate of the contribution of $WZ$ + heavy-flavor, where heavy-flavor jets include b-jets and charm jets. The full Run-2 dataset collected by the ATLAS detector at the LHC, representing 139 $fb^{-1}$ of data from pp collisions at $\sqrt{s} = 13$ TeV, is used for this study.

All backgrounds are accounted for using Monte Carlo, with the simulation of non-prompt lepton backgrounds - Z+jets and $t\bar{t}$ - validated using non-prompt Validation Regions.

Section \ref{sec:data} details the data and Monte Carlo (MC) samples used in the analysis. The reconstruction of various physics objects is described in section \ref{sec:obj}. Section \ref{sec:evt_selection} describes the event selection applied to these samples, along the definitions of the various regions used in the fit. The multivariate analysis techniques used to separate the tZ background from WZ + heavy flavor are described in section \ref{sec:tZ_bdt}. Section \ref{sec:sys} describes the various sources of systematic uncertainties considered in the fit. Finally, the results of the analysis are summarized in section \ref{sec:results}, followed by a brief conclusion in section \ref{sec:conclusion}.

The analysis aims to report a cross-section measurement of WZ+b and WZ+charm, along with their correlations, for both 1-jet and 2-jet exclusive regions. The proposed fiducial region for these measurements includes events with three leptons, where the invariant mass of at least one opposite charge, same flavor lepton pair falls within 10 GeV of 91.2 GeV, with 1 or 2 associated jets. An alternate version of the measurement is included in the appendix, which considers tZ as part of the WZ+b signal.

The current state of thee analysis shows blinded results for thee full Run 2 dataset. Regions containing $>$5\% WZ+b events are blinded, and results are from Asimov, MC only fits. Expected significance and cross-section numbers are reported.
