
All regions defined in this analysis share a common lepton, jet, and overall event preselection. The selection applied to each physics object is detailed here; the event preselection, and the selection used to define the various fit regions, is described in Section \ref{sec:evt_selection}.

All events are required to be selected by dilepton triggers. The $p_T$ thresholds of the dilepton trigger on two electrons were 12 GeV in 2015, 17 GeV in 2016, and 24 GeV in 2017 and 2018, while for the dimuon triggers the $p_T$ thresholds on the leading (sub-leading) muon were 18 GeV (8 GeV) in 2015, and 22 GeV (8 GeV) in 2016-2018. For the electron+muon triggers, the $p_T$ thresholds on the electron (muon) were 17 GeV (14 GeV) for all datasets.

Electron candidates are reconstructed from energy clusters in the electromagnetic calorimeter that are associated with charged particle tracks reconstructed in the inner detector~\cite{ele_eff}. Electron candidates are required to have $\pt > 10$ GeV and $|\eta_\textrm{cluster}| < 2.47$. Candidates in the transition region between different electromagnetic calorimeter components, $1.37 < |\eta_\textrm{cluster}| < 1.52$, are rejected. A multivariate likelihood discriminant combining shower shape and track information is used to distinguish real electrons from hadronic showers (fake electrons). To further reduce the non-prompt electron contribution, the track is required to be consistent with originating from the primary vertex; requirements are imposed on the transverse impact parameter significance ($|d_0|/\sigma_{d_0}<5$) and the longitudinal impact parameter ($|\Delta z_0 \sin \theta_\ell| < 0.5$ mm). Electron candidates are required to pass the \textsc{TightLH} identification requirement detailed in \cite{Aad:2014fxa}.
                   
Muon candidates are reconstructed by combining inner detector tracks with track segments or full tracks in the muon spectrometer \cite{PERF-2014-05}. Muon candidates are required to have $\pt > 10$~GeV and $|\eta| < 2.5$. The longitutinal impact parameter is the same for both electrons and muons, while muons are required to pass a slightly tighter transverse imapact parameter selection, $|d_0|/\sigma_{d_0}<3$. Muons are also required to pass Medium ID requirements, as detailed in \cite{Aad:2014fxa}. Leptons are additionally required to pass a non-prompt BDT selection, described in detail in \cite{ttH_comb}. Optimized working points and scale factors for this BDT are taken from that analysis.

%\begin{table}
%\begin{center}
% \begin{tabular}{l|cccccc}
% \hline\hline
% & \multicolumn{3}{c|}{$e$} & \multicolumn{3}{c}{$\mu$} \\
% \hline
% & \multicolumn{1}{c|}{L}  & \multicolumn{1}{c|}{L*} & \multicolumn{1}{c|}{T} & \multicolumn{1}{c|}{L} & \multicolumn{1}{c|}%{L*} & \multicolumn{1}{c|}{T}  \\
%  \hline
%  FixedCutLoose           &  No & \multicolumn{2}{|c|}{Yes} & No & \multicolumn{2}{|c}{Yes} \\
%  \hline
%  Non-prompt lepton BDT   &  \multicolumn{2}{c|}{No} & \multicolumn{1}{c|}{Yes} & \multicolumn{2}{c|}{No} & \multicolumn{1}{%c}{Yes} \\
%  \hline
%  Identification  & \multicolumn{2}{c|}{Loose} & \multicolumn{1}{c|}{Tight} & \multicolumn{2}{c}{Loose} & \multicolumn{1}{c|%}{Medium}\\
%  \hline
%  %Charge mis-assignment veto &  \multicolumn{2}{c|}{No} & Yes & \multicolumn{3}{|c}{N/A} \\ - not needed, only for 2lSS
%  %\hline
%  %ambiguity bit == 0 &  \multicolumn{2}{c|}{No} & Yes & \multicolumn{3}{|c}{N/A} \\
%  %\hline
%  Transverse impact parameter significance  &  \multicolumn{3}{c|}{$<5$} & \multicolumn{3}{c}{$<3$ } \\
%  $|d_0|/\sigma_{d_0}$ & \multicolumn{3}{c|}{} &  \multicolumn{3}{c}{}  \\
%  \hline
%  Longitudinal impact parameter &  \multicolumn{6}{c}{$< 0.5$ mm} \\
%  $|z_0 \sin \theta|$ &  \multicolumn{6}{c}{} \\
%  \hline\hline
% \end{tabular}
%\caption{\label{tbl:tightleps} Loose (L), loose and minimally-isolated (L*), and tight (T) light lepton definitions.}
%\end{center}
%\end{table}


%UPDATE FOR PFLOW
Jets are reconstructed from calibrated topological clusters built from energy deposits in the calorimeters using the anti-$k_t$ algorithm \cite{Cacciari_2008} with a radius parameter $R=0.4$. Jets with energy contributions likely arising from noise or detector effects are removed from consideration, and only jets satisfying $\pt > 25$~GeV and $|\eta| < 2.5$ are used in this analysis.  For jets with $\pt < 60$~GeV and $|\eta| < 2.4$, a jet-track association algorithm is used to confirm that the jet originates from the selected primary vertex, in order to reject jets arising from pileup collisions \cite{PERF-2014-03}.


In order to make a measurement of $WZ$ + heavy flavor it is necessary to distinguish these events from $WZ$ + light jets. For this purpose, the DL1r b-tagging algorithm is used to distinguish heavy flavor jets from lighter ones \cite{btagging}. The DL1r algorithm uses jet kinematics, particularly jet vertex information, as input for a neural network which assigns each jet a score designed to reflect how likely that jet is to have originated from a b-quark. 

From the output of the BDT, calibrated working points (WPs) are developed based on the efficiency of truth b-jets at particular values of the DL1r algorithm. The working points used in this analysis are summarized in Table \ref{tab:btag_WPs}. 

\begin{table}[H] 
\begin{center}
\begin{tabular}{|c|c|c|}
    \hline
    WP & \multicolumn{2}{c|}{Rejection}\\
    \hline
    $b$-jet eff. & $c$-jet & light jet\\
    \hline
    85\% & 2.6 & 29 \\
    77\% & 4.9 & 130 \\
    70\% & 9.4 & 390 \\
    60\% & 27 & 1300 \\ 
    \hline
    \end{tabular}    
    \caption{$c$-jet and light-flavor jet rejections corresponding to each $b$-tagging Working Point by b-jet efficiency, evaluated on $t\bar{t}$ events.}
    \label{tab:btag_WPs}
    \end{center}
\end{table}

%rewrite paragraph
As shown in table \ref{tab:btag_WPs}, a tighter WP will accept fewer b-jets, but reject a higher fraction of charm and light jets. Generally, analyses that include b-jets will use a fixed working point, for example, requiring that a jet pass the 70\% threshold. By instead treating these working point as bins, e.g. events with jets that fall between the 85\% and 77\% WPs fall into one bin, while events with jets passing the 60\% WP fall into another, additional information can be gained. This analysis uses each of these working points to form orthogonal regions in order to provide separation between WZ + b, WZ + charm, and WZ + light. 

Missing transverse momentum ($E_T^{miss}$) is used as part of the event selection. The missing transverse momentum vector is defined as the inverse of the sum of the transverse momenta of all reconstructed physics objects as well as remaining unclustered energy, the latter of which is estimated from low-\pt tracks associated with the primary vertex but not assigned to a hard object, with object definitions taken from \cite{met_2018}. Light leptons considered in the $E_T^{miss}$ reconstruction are required to have $p_T$>10 GeV, while jets are required to have $p_T$>20 GeV.

To avoid double counting objects and remove leptons originating from decays of hadrons, overlap removal is performed in the following order: any electron candidate within $\Delta R = 0.1$ of another electron candidate with higher \pt\ is removed; any electron candidate within $\Delta R = 0.1$ of a muon candidate is removed; any jet within $\Delta R = 0.3$ of an electron candidate is removed; if a muon candidate and a jet lie within $\Delta R = min(0.4, $0.04+10$[GeV]/\pt(muon))$ of each other, the jet is kept and the muon is removed. This algorithm is applied to the preselected objects. 

