
The production of $WZ$ in association with a heavy flavor jet represents an important background for many major analyses. This includes any process with leptons and b-jets in the final state, such as $t\bar{t}H$, $t\bar{t}W$, and $t\bar{t}Z$. While precise measurements have been made of $WZ$ production \cite{WZ_36}, $WZ$ + heavy flavor remains poorly understood. This is largely because the QCD processes involved in the production of the b-jet make it difficult to simulate accurately. This introduces a large uncertainty for analyses that include this process as a background.  

Motivated by its relevance to the $t\bar{t}H$ multilepton analysis, we perform a study of the fully leptonic decay mode of this channel, that is, events where both the W and Z decay leptonically. This gives a final state signature of three leptons and at least one jet.

Events with three leptons and one or two jets are sorted into pseudo-continuous b-tagging regions based on the MV2c10 b-tag score of their associated jets. This is done to separate $WZ$ + b-jet events from $WZ$ + charm and $WZ$ + light jets. These regions are fit to data in order make a more accurate estimate of the contribution of $WZ$ + heavy-flavor, where heavy-flavor jets include b-jets and charm jets. The full Run-2 dataset collected by the ATLAS detector, representing 139 $fb^{-1}$ of data from pp collisions at $\sqrt{s} = 13$ TeV, is used for this study.

Section \ref{sec:data} details the data and Monte Carlo (MC) samples used in the analysis, and the reconstruction of various physics objects is described in section \ref{sec:obj}. Section \ref{sec:evt_selection} describes the event selection applied to these samples. The multivariate analysis techniques used to separate the tZ background from WZ + heavy flavor are described in section \ref{sec:tZ_bdt}. The regions defined for the fit are then described in section \ref{sec:signal_region}. Section \ref{sec:sys} describes the various sources of systematic uncertainties considered in the fit. Finally, the results of the analysis are summarized in section \ref{sec:results}, followed by a brief conclusion in section \ref{sec:conclusion}.

\textbf{The current state of the analysis shows blinded results for the full 2018 dataset, awaiting unblinding approval. 2018 recommendations and working points have not yet been fully implemented, and the 2018 dataset contributions currently use 2017 recommendations.}
