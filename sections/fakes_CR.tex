
In order to further validate the modeling in each of the non-prompt CRs, additional kinematic plots are made in the Z+jets CR and $t\bar{t}$ CR in each of the continuous b-tag regions.

In the case of the Z+jets CR, the $p_T$ spectrum of the lepton originating from the W candidate is shown, as this is the distribution used to extract the scale factor applied to Z+jets. These plots are shown in Figures \ref{fig:zjets_ptW_1j} and \ref{fig:zjets_ptW_2j}.

\begin{figure}[H]
    \centering                                                                                                               
    \subfigure[]{\includegraphics[width=.29\linewidth]{valZjets_lepPtW/Plots/not_85.png}}%                         
    \subfigure[]{\includegraphics[width=.29\linewidth]{valZjets_lepPtW/Plots/WP_1b_77_85.png}}%               
    \subfigure[]{\includegraphics[width=.29\linewidth]{valZjets_lepPtW/Plots/WP_1b_70_77.png}}\\
    \subfigure[]{\includegraphics[width=.29\linewidth]{valZjets_lepPtW/Plots/WP_1b_60_70.png}}%      
    \subfigure[]{\includegraphics[width=.29\linewidth]{valZjets_lepPtW/Plots/WP_1b_60.png}}%                
    \caption{Comparisons between the data and MC distributions in the Z+jets CR for each of the 1-jet b-tag working point regions}
   \label{fig:zjets_ptW_1j}
\end{figure}

\begin{figure}[H]
    \centering
    \subfigure[]{\includegraphics[width=.29\linewidth]{valZjets_lepPtW/Plots/not_85_2j.png}}%            
    \subfigure[]{\includegraphics[width=.29\linewidth]{valZjets_lepPtW/Plots/WP_2b_77_85.png}}%                      
    \subfigure[]{\includegraphics[width=.29\linewidth]{valZjets_lepPtW/Plots/WP_2b_70_77.png}}\\
    \subfigure[]{\includegraphics[width=.29\linewidth]{valZjets_lepPtW/Plots/WP_2b_60_70.png}}%                      
    \subfigure[]{\includegraphics[width=.29\linewidth]{valZjets_lepPtW/Plots/WP_2b_60.png}}%                         
    \caption{Comparisons between the data and MC distributions in the Z+jets CR for each of the 2-jet b-tag working point regions}
   \label{fig:zjets_ptW_2j}
\end{figure}

The same is shown for the $t\bar{t}$ CR, but the $p_T$ of the OS lepton is used instead as a representation of the modeling, as the lepton from the W is not well defined for $t\bar{t}$ events. These plots are shown in Figures \ref{fig:ttbar_pt0_1j} and \ref{fig:ttbar_pt0_2j}.

\begin{figure}[H]
    \centering
    \subfigure[]{\includegraphics[width=.29\linewidth]{valTtbar_lepPt0/Plots/not_85.png}}%                           
    \subfigure[]{\includegraphics[width=.29\linewidth]{valTtbar_lepPt0/Plots/WP_1b_77_85.png}}%                      
    \subfigure[]{\includegraphics[width=.29\linewidth]{valTtbar_lepPt0/Plots/WP_1b_70_77.png}}\\
    \subfigure[]{\includegraphics[width=.29\linewidth]{valTtbar_lepPt0/Plots/WP_1b_60_70.png}}%                      
    \subfigure[]{\includegraphics[width=.29\linewidth]{valTtbar_lepPt0/Plots/WP_1b_60.png}}%                         
    \caption{Comparisons between the data and MC distributions in the $t\bar{t}$ CR for each of the 1-jet b-tag working point regions}
   \label{fig:ttbar_pt0_1j}
\end{figure}                                                                                                                 
 
\begin{figure}[H]                                                                                                            
    \centering
    \subfigure[]{\includegraphics[width=.29\linewidth]{valTtbar_lepPt0/Plots/not_85_2j.png}}%                        
    \subfigure[]{\includegraphics[width=.29\linewidth]{valTtbar_lepPt0/Plots/WP_2b_77_85.png}}%
    \subfigure[]{\includegraphics[width=.29\linewidth]{valTtbar_lepPt0/Plots/WP_2b_70_77.png}}\\
    \subfigure[]{\includegraphics[width=.29\linewidth]{valTtbar_lepPt0/Plots/WP_2b_60_70.png}}%
    \subfigure[]{\includegraphics[width=.29\linewidth]{valTtbar_lepPt0/Plots/WP_2b_60.png}}%
    \caption{Comparisons between the data and MC distributions in the $t\bar{t}$ CR for each of the 2-jet b-tag working point regions}
   \label{fig:ttbar_pt0_2j}
\end{figure}
