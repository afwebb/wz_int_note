The ATLAS detector \cite{} at the LHC is a general purpose detector that covers nearly the entire solid angle around the collision point. It consists of several concentric subdectors: The inner tracking detector, electromagnetic and hadronic calorimeters, and a muon spectrometer. The inner detector is made up of a high-granularity silicon pixel detector, designed to reconstruct the tracks of charged particles in a range of $|\eta|< 2.5$, and a transition radiation tracker which provides additional tracking and electron identification information for $|\eta|< 2.0$ \cite{}. A 2 T axial magnetic field is produced in the inner detector, in order to bend the path of charged particles. The calorimeter system cover a pseudorapidity range of $|\eta|< 4.9$, with a a lead/liquid-argon (LAr) electromagnetic calorimeter covering $|\eta|< 3.2$, and a steel/scintillator-tile hadronic calorimeter. The rest of the solid-angle coverage of the calorimeter system comes from forward copper/LAr and tungsten/LAr modules. The muon spectrometer measures muons with $|\eta|< 2.7$ using several layered tracking chambers placed within a magnetic field of approximately 0.5 T. A two-level trigger system \cite{} is used to reduce the event rate from 40 MHz to around 1 kHz, using a hardware based Level-1 trigger, followed by a second software based High-Level Trigger (HLT).
