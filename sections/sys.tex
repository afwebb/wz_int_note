
The systematic uncertainties that are considered are summarized in table \ref{tab:SystSummary}. These are implemented in the fit either as a normalization factors or as a shape variation or both in the signal and background estimations. The numerical impact of each of these uncertainties is outlined in section \ref{sec:results}.

\begin{table}[hbt!]
\centering
\caption{Sources of systematic uncertainty considered in the analysis.}
\begin{tabular}{lr}
\hline\hline
Systematic uncertainty & Components  	      \\
\hline
\hline
Luminosity	& 1		      \\
Pileup reweighting 	& 1		      \\
\textbf {Physics Objects}     	&		      \\
\ \ Electron                               	& 6		      \\
\ \ Muon	& 15		      \\
\ \ Jet energy scale and resolution  	& 28                  \\
\ \ Jet vertex fraction  	& 1		      \\
\ \ Jet flavor tagging   	& 131		      \\
\ \ $E^{miss}_T$  	& 3		      \\
\hline
Total (Experimental)        & 186		     \\
\hline
\hline
\textbf {Background Modeling}          	&		      \\
\ \ Cross section                 	& 24		      \\
\ \ Renormalization and factorization scales 	& 10		      \\
\ \ Parton shower and hadronization model       	& 2		      \\
\ \ Shower tune				& 4		      \\
\hline
Total (Signal and background modeling)       & 40		     \\
\hline\hline
Total (Overall)                             & 226	      \\
\hline\hline
\end{tabular}
\label{tab:SystSummary}
\end{table}

The uncertainty in the combined integrated luminosity is derived from a calibration of the luminosity scale performed for 13 TeV proton-proton collisions \cite{lumi}, \cite{LUCID2}.

The experimental uncertainties are related to the reconstruction and identification of light leptons and b-tagging of jets, and to the reconstruction of $E^{miss}_T$. The sources which contribute to the uncertainty in the jet energy scale (JES) \cite{jes} are decomposed into uncorrelated components and treated as independent sources in the analysis. The CategoryReduction model is used to account for JES uncertainties, which decomposes the uncertainties into 30 nuiscance parameters included in the fit. The SimpleJER model is used to account for jet energy resolution (JER) uncertainties, and 8 JER uncertainty components uncluded as NPs in the fit. 

The uncertainties in the b-tagging efficiencies measured in dedicated calibration analyses \cite{btag_cal} are also decomposed into uncorrelated components. The large number of components for b-tagging is due to the calibration of the distribution of the MVA discriminant.  

The systematic uncertainties associated with the signal and background processes are accounted for by varying the cross-section of each process within its uncertainty.

The full list of systematic uncertainties considered in the analysis is summarized in tables
\ref{Tab:LeptonExperimentalSyst}, \ref{Tab:JetsExperimentalSyst} and \ref{Tab:BTagExperimentalSyst}.

\hspace{-1in}\begin{table}[hbt!]
  \begin{center}
    {\small
    \begin{tabular}{|llcl|}
      \hline
      \multicolumn{4}{|c|}{\bf Experimental Systematics on Leptons and $E_T^{miss}$} \\
     % \hline
      Type     & Description  & Systematics Name & Application \\
     \hline
     \hline
     \multicolumn{4}{|c|}{\bf{Trigger}}\\
     \hline
    Scale Factors    & Trigger Efficiency        & lepSFTrigTight$\_$MU(EL)$\_$SF$\_$Trigger$\_$STAT(SYST)    & Event Weight      \\
      \hline
      \multicolumn{4}{|c|}{\bf{Muons}} \\
      \hline
      Efficiencies   & Reconstruction and        & lepSFObjTight$\_$MU$\_$SF$\_$ID$\_$STAT(SYST)              & Event Weight       \\
     & Identification    &       &        \\
      & Isolation                 &       lepSFObjTight$\_$MU$\_$SF$\_$Isol$\_$STAT(SYST)            & Event Weight       \\
         & Track To Vertex   	 & lepSFObjTight$\_$MU$\_$SF$\_$TTVA$\_$STAT(SYST )           & Event Weight       \\
    & Association  		 &   							      &           \\
     \pt Scale   & \pt Scale & MUONS$\_$SCALE    & \pt Correction     \\
     &   &   &           \\
      Resolution     & Inner Detector            & MUONS$\_$ID        					      & \pt Correction     \\
         & Energy Resolution      	 &     &         \\
    & Muon Spectrometer    	 & MUONS$\_$MS      & \pt Correction     \\
     & Energy Resolution         &       &        \\
     &   &   &         \\
     \hline
     \multicolumn{4}{|c|}{\bf{Electrons}}\\
     \hline
     Efficiencies    & Reconstruction       	 & lepSFObjTight$\_$EL$\_$SF$\_$ID  			      & Event Weight   	    \\
     & Identification   & lepSFObjTight$\_$EL$\_$SF$\_$Reco       		      & Event Weight            \\
        & Isolation                 & lepSFObjTight$\_$EL$\_$SF$\_$Isol      		      & Event Weight        \\
       &   &   &          \\
     Scale Factor    & Energy  Scale             & EG$\_$SCALE$\_$ALL  					      & Energy Correction    \\
         	     &   &   &          \\
     Resolution      & Energy Resolution  	 & EG$\_$RESOLUTION$\_$ALL      			      & Energy Correction     \\
         	     &   &   &             \\
     \hline
     \multicolumn{4}{|c|}{\bf{$E_T^{miss}$}}\\
     \hline
     Soft Tracks Terms         &             Resolution                   &      MET$\_$SoftTrk$\_$ResoPerp       &   \pt Correction  \\
                               &             Resolution                   &      MET$\_$SoftTrk$\_$ResoPara        &    \pt Correction    \\
                               &             Scale                        &      MET$\_$SoftTrk$\_$ScaleUp         &   \pt Correction     \\
                               &             Scale                        &      MET$\_$SoftTrk$\_$ScaleDown         &   \pt Correction     \\

     \hline
     
    \end{tabular}
   }
   \caption{\label{Tab:LeptonExperimentalSyst} Summary of experimental systematics considered for leptons and $E_T^{miss}$. Includes type, description, name of systematic as used in the fit, and mode of application. The mode of application indicates the systematic evaluation, e.g. as an  overall event re-weighting (Event Weight) or rescaling (\pt Correction).}
  \end{center}
\end{table}


\begin{table}[hbt!]
  \begin{center}
    {\small
    \begin{tabular}{|llcc|}
      \hline
      \multicolumn{4}{|c|}{\bf Experimental Systematics on Jets} \\
      \hline
      Type     & Origin   & Systematics Name  & Application \\
      \hline
      Jet Vertex Tagger         &     & JVT      &        Event Weight          \\
     	&   &   &     \\
      Energy Scale              & Calibration Method              & JET$\_$21NP$\_$           &      \pt Correction         \\
       &   & JET$\_$EffectiveNP$\_$1-19     &    \pt Correction  \\
       &   &   &       \\
        & $\eta$ inter-calibration        & JET$\_$EtaIntercalibration$\_$Modelling    & \pt Correction          \\
     &                                 & JET$\_$EtaIntercalibration$\_$NonClosure   & \pt Correction      \\
     &                                 & JET$\_$EtaIntercalibration$\_$TotalStat    & \pt Correction      \\
    &   &   &        \\
     & High \pt jets                   & JET$\_$SingleParticle$\_$HighPt         &     \pt Correction             \\
     	&   &   &           \\
        & Pile-Up                         & JET$\_$Pileup$\_$OffsetNPV            &     \pt Correction             \\
        &       & JET$\_$Pileup$\_$OffsetMu             &     \pt Correction               \\
        &        & JET$\_$Pileup$\_$PtTerm       &     \pt Correction         \\
        &                                         & JET$\_$Pileup$\_$RhoTopology      &     \pt Correction             \\
    	&   &   &            \\
          & Non Closure                     & JET$\_$PunchThrough$\_$MC15    & \pt Correction    \\
    	&   &   &       \\
         & Flavour                         & JET$\_$Flavor$\_$Response          &   \pt Correction            \\
     &         & JET$\_$BJES$\_$Response          &   \pt Correction           \\
           &                                 & JET$\_$Flavor$\_$Composition        &    \pt Correction             \\
         	&   &   &          \\
      Resolution         	&                                 & JET$\_$JER$\_$SINGLE$\_$NP          &  Event Weight       \\
        			&   &   &          \\
        			
    \hline

     \end{tabular}
    }
    \caption{\label{Tab:JetsExperimentalSyst} Jet systematics take into account effects of jets calibration method, $\eta$ inter-calibration, high \pt jets, pile-up, and flavor response. They are all diagonalised into effective parameters.}
 \end{center}
\end{table}

\begin{table}[htb!] 
  \begin{center}
    {\small
    \begin{tabular}{|llc|}
      \hline
     \multicolumn{3}{|c|}{\bf Experimental Systematics on b-tagging} \\
      \hline
      Type     & Origin   & Systematic Name \\
     \hline
     &   &                \\
      Scale Factors & DL1r b-tagger efficiency & DL1r$\_$Continuous$\_$EventWeight$\_$B0-29 \\
      &    on b originated jets in bins of $\eta$  &   \\
      &   &                \\
      &    DL1r b-tagger efficiency & DL1r$\_$Continuous$\_$EventWeight$\_$C0-19  \\
      &    on c originated jets in bins of $\eta$    &     \\
      &   &   \\
      &    DL1r b-tagger efficiency & DL1r$\_$Continuous$\_$EventWeight$\_$Light0-79           \\
      &    on light flavoured originated jets         &   \\
     &     in bins of $\eta$ and \pt      &    \\
         &   &             \\
     &    DL1r b-tagger                        & DL1r$\_$Continuous$\_$EventWeight$\_$extrapolation  \\
     &    extrapolation efficiency    &         DL1r$\_$Continuous$\_$EventWeight$\_$extrapolation$\_$from$\_$charm             \\
     \hline
    \end{tabular}
    }
    \caption{\label{Tab:BTagExperimentalSyst} Summary of experimental systematics to be included for $b$-tagging of jets in the analysis, using the continuous DL1r tagging algorithm. All of the b-tagging related systematics are applied as event weights. From left: type, description, and the name of systematic used in the fit.}
  \end{center}
\end{table}

Theoritical uncertainties applied to backgrounds, including cross section, PDF, and scale uncertainties are taken from theory calculations, with the exception of non-prompt and diboson backgrounds. Derivation of the non-prompt background uncertainties, Z+jets and $t\bar{t}$, are explained in detail in section \ref{sec:fakes}. Because the other VV + heavy flavor processes are also poorly understood, a conservative 50\% uncertainty is applied to those samples. The theory uncertainties applied to the predominate background estimates are summarized in table \ref{tab:xsecUnc}. 

\begin{table}[!htbp]
{\footnotesize
\centering
\begin{tabular}{c|c}
\hline
Process                 & X-section [\%]                \\
\hline
WZ                      & QCD Scale: $^{+3.7}_{-3.4}$ \\                                                                     
                        & PDF($+\alpha_S$): $\pm 3.1$ \\
%\hline
tZ                      & X-sec: $\pm 15.2$    \\ 
%                        & QCD Scale: $^{+5.2}_{-1.3}$ \\
%                        & PDF($+\alpha_S$): $\pm 1.2$ \\
\hline
\ttbar H                & QCD Scale:$^{+5.8}_{-9.2}$    \\
(aMC$@$NLO$+$Pythia8)   & PDF($+\alpha_S$): $\pm$ 3.6   \\
\hline
\ttbar Z                & QCD Scale:$^{+9.6}_{-11.3}$   \\
(aMC$@$NLO$+$Pythia8)   & PDF($+\alpha_S$): $\pm$ 4     \\
\hline
\ttbar W                & QCD Scale:$^{+12.9}_{-11.5}$  \\
(aMC$@$NLO$+$Pythia8)   & PDF($+\alpha _S$): $\pm$ 3.4  \\
\hline
VV + b/charm            & $\pm$ 50                      \\
(Sherpa 2.2.1)          &                               \\
\hline
VV + light              & $\pm$ 6                      \\   
(Sherpa 2.2.1)          &                               \\
\hline
$t\bar{t}$              & $\pm$ 20 \\                          
\hline
Z + jets                & $\pm$ 25 \\
\hline
Others                  & $\pm$ 50 \\
\hline
\hline
\end{tabular}



\caption{Summary of theoretical uncertainties for MC predictions in the analysis.}
\label{tab:xsecUnc}
}
\end{table}
