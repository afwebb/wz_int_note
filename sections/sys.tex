
The systematic uncertainties that are considered are summarized in Table \ref{tab:SystSummary}. These are implemented in the fit either as a normalization factors or as a shape variation or both in the signal and background estimations. The numerical impact of each of these uncertainties is outlined in Section \ref{sec:results}.

\begin{table}[H]
\centering
\caption{Sources of systematic uncertainty considered in the analysis.}
\begin{tabular}{lr}
\hline\hline
Systematic uncertainty & Components  	      \\
\hline
\hline
Luminosity	& 1		      \\
Pileup reweighting 	& 1		      \\
\textbf {Physics Objects}     	&		      \\
\ \ Electron                               	& 6		      \\
\ \ Muon	& 15		      \\
\ \ Jet energy scale   	& 28                  \\
\ \ Jet energy resolution & 8 \\
\ \ Jet vertex fraction  	& 1		      \\
\ \ Jet flavor tagging   	& 131		      \\
\ \ $E^{miss}_T$  	& 3		      \\
\hline
Total (Experimental)        & 194		     \\
\hline
\hline
\textbf {Signal Modeling}           &                     \\
\ \ Shape modelling & 3 \\
\ \ Renormalization and factorization scales    & 5                  \\
\ \ nJet Migration & 5 \\
\textbf {Background Modeling}          	&		      \\
\ \ Cross section                 	& 15		      \\
\ \ Renormalization and factorization scales 	& 12		      \\
%\ \ Parton shower and hadronization model       	& 2		      \\
%\ \ Shower tune				& 4		      \\
\hline
Total (Signal and background modeling)       &  35		     \\
\hline\hline
Total (Overall)                             & 230	      \\
\hline\hline
\end{tabular}
\label{tab:SystSummary}
\end{table}

%The uncertainty in the combined integrated luminosity is derived from a calibration of the luminosity scale performed for 13 TeV proton-proton collisions \cite{lumi}, \cite{LUCID2}.

The uncertainty in the combined 2015--2018 integrated luminosity is 1.7\% \cite{ATLAS:2019pzw}, obtained using the LUCID-2 detector \cite{LUCID2} for the primary luminosity measurements.

The experimental uncertainties are related to the reconstruction and identification of light leptons and b-tagging of jets, and to the reconstruction of $E^{miss}_T$. The sources which contribute to the uncertainty in the jet energy scale (JES) \cite{PERF-2016-04} are decomposed into uncorrelated components and treated as independent sources of uncertainty in the analysis. A similar approach is used for the jet energy resolution (JER) uncertainty.

The uncertainties in the b-tagging efficiencies measured in dedicated calibration analyses \cite{btag_cal} are also decomposed into uncorrelated components. The large number of components for b-tagging is due to the calibration of the distribution of the MVA discriminant for each individual WP bin.

The fit involves varying the overall normalization of signal templates over the regions described in Section \ref{subsec:regions}, which are defined by the flavor and number of associated jets at truth-level. The modelling of these template shapes therefore significantly impacts the final result. Additional signal uncertainties, probing the shape of the signal templates as well as the rate of migrations between the number of truth-jets and reconstructed jets, are estimated by comparing estimates from the nominal Sherpa WZ samples with alternative WZ samples generated with \textsc{Powheg}+\textsc{Pythia8}. Separate systematics are included in the fit for WZ + $b$, WZ + $c$ and WZ + light, where the distribution among each of the fit regions is varied based on the prediction of the Powheg sample.

A similar approach is taken to account for uncertainties in migrations between the number of reco and truth jets. The fraction of events with 1 truth jet which fall into the 1 jet bin versus the 2 jet bin at reco level is compared for Sherpa and Powheg. The same is done for events with 2 truth jets. A systematic is included where events are shifted between the 1-jet and 2-jet regions based on the differences between these two shapes. This is done independently for each of the WZ + $b$, WZ + $c$, and WZ + light templates.

Additional systematics are included to account for the uncertainty in the contamination of 0 jet and 3 or more jet events (as defined at truth level) in the 1 and 2 reco jet bins. Because these events fall outside the scope of this measurement, these events are included as a background. As such, a normalization, rather than a shape, uncertainty is applied for this background. The number of WZ events with 0-jets and $>=$3-jets in the reconstructed 1-jet and 2-jet regions are compared for Sherpa and Powheg, and these differences are taken as separate normalization systematics on the yield of WZ+0-jet and WZ+$>=$3-jet events.

Theoretical uncertainties applied to MC background predictions, including cross section, PDF, and scale uncertainties are taken from theory calculations, with the exception of non-prompt and diboson backgrounds. The cross-section uncertainty on tZ is taken from \cite{tZ_paper}. Derivation of the non-prompt background uncertainties, Z+jets and $t\bar{t}$, are explained in Section \ref{sec:fakes}. These normalization uncertainties are chosen so as to account for the complete uncertainty in the non-prompt contribution, and therefore no additional modelling uncertainties are considered for Z+jets and $t\bar{t}$.

Due to its importance as a background, additional modelling uncertainties are considered for tZ. Alternative tZ samples with variations in scale and shower modelling are included as systematics. The other VV + heavy flavor processes (namely VV+b and VV+charm, which primarily consist of ZZ events) are also poorly understood, because these processes involve the same physics as WZ + heavy flavor, and have also not been measured. Therefore, a conservative 50\% uncertainty is applied to those samples. While this uncertainty is large, it is found to have little impact on the significance of the final result.

The theory uncertainties applied to the MC estimates are summarized in Table \ref{tab:xsecUnc}. 

\begin{table}[H]
{\footnotesize
\centering
\begin{tabular}{|c|c|}
\hline
Process                 & X-section [\%]                \\
\hline
WZ                      & QCD Scale: $^{+3.7}_{-3.4}$ \\                                                                     
                        & PDF($+\alpha_S$): $\pm 3.1$ \\
\hline
tZ                      & X-sec: $\pm 15.2$    \\ 
                        & QCD Scale: $^{+5.2}_{-1.3}$ \\
                        & PDF($+\alpha_S$): $\pm 1.2$ \\
\hline
\ttbar H                & QCD Scale:$^{+5.8}_{-9.2}$    \\
(aMC$@$NLO$+$Pythia8)   & PDF($+\alpha_S$): $\pm$ 3.6   \\
\hline
\ttbar Z                & QCD Scale:$^{+9.6}_{-11.3}$   \\
(aMC$@$NLO$+$Pythia8)   & PDF($+\alpha_S$): $\pm$ 4     \\
\hline
\ttbar W                & QCD Scale:$^{+12.9}_{-11.5}$  \\
(aMC$@$NLO$+$Pythia8)   & PDF($+\alpha _S$): $\pm$ 3.4  \\
\hline
VV + b/charm            & $\pm$ 50                      \\
(Sherpa 2.2.1)          &                               \\
\hline
VV + light              & $\pm$ 6                      \\   
(Sherpa 2.2.1)          &                               \\
\hline
$t\bar{t}$              & $\pm$ 20 \\                          
\hline
Z + jets                & $\pm$ 25 \\
\hline
Others                  & $\pm$ 50 \\
\hline
\end{tabular}



\caption{Summary of theoretical uncertainties for normalization of MC predictions in the analysis.}
\label{tab:xsecUnc}
}
\end{table}

