
The systematic uncertainties that are considered are summarized in Table \ref{tab:SystSummary}. These are implemented in the fit either as a normalization factors or as a shape variation or both in the signal and background estimations. These systematic uncertainties are treated as independent nuiscance parameters (NPs) in the fit, and are assumed to be gaussian. Each NP is varied up and down by its uncertainty, and the difference in the result of the fit is taken to be the systematic uncertainty of that NP. The numerical impact of each of these uncertainties is outlined in Section \ref{sec:results}.

\begin{table}[H]
\centering
\caption{Sources of systematic uncertainty considered in the analysis.}
\begin{tabular}{lr}
\hline\hline
Systematic uncertainty & Components  	      \\
\hline
\hline
Luminosity	& 1		      \\
Pileup reweighting 	& 1		      \\
\textbf {Physics Objects}     	&		      \\
\ \ Electron                               	& 5		      \\
\ \ Muon	& 14		      \\
\ \ Prompt Lepton Veto & 1 \\
\ \ Jet energy scale   	& 28                  \\
\ \ Jet energy resolution & 8 \\
\ \ Jet vertex fraction  	& 1		      \\
\ \ Jet flavor tagging   	& 131		      \\
\ \ $E^{miss}_T$  	& 3		      \\
\hline
Total (Experimental)        & 194		     \\
\hline
\hline
\textbf {Signal Modeling}           &                     \\
\ \ Shape modelling & 3 \\
\ \ Renormalization and factorization scales    & 5                  \\
\ \ nJet Migration & 5 \\
\textbf {Background Modeling}          	&		      \\
\ \ Cross section                 	& 15		      \\
\ \ Renormalization and factorization scales 	& 12		      \\
\ \ Parton shower and hadronization model       	& 2		      \\
\ \ Shower tune				& 4		      \\
\hline
Total (Signal and background modeling)       &  41		     \\
\hline\hline
Total (Overall)                             & 235	      \\
\hline\hline
\end{tabular}
\label{tab:SystSummary}
\end{table}

The uncertainty in the combined 2015--2018 integrated luminosity is 1.7\% \cite{ATLAS:2019pzw}, obtained using the LUCID-2 detector \cite{LUCID2} for the primary luminosity measurements.

\paragraph{}
The experimental uncertainties are related to the reconstruction and identification of light leptons and b-tagging of jets, and to the reconstruction of $E^{miss}_T$. The \verb!TOTAL! electron ID correlation model is used, corresponding to 1 electron ID systematic \cite{ele_eff}. Electron ID is found to be a subleading systematic that is unconstrained by the fit, making it an appropriate choice for this analysis.

\paragraph{}
The sources which contribute to the uncertainty in the jet energy scale (JES) \cite{PERF-2016-04} are decomposed into uncorrelated components and treated as independent sources in the analysis. The CategoryReduction model is used to account for JES uncertainties, which decomposes the uncertainties into 30 nuiscance parameters included in the fit. The SimpleJER model is used to account for jet energy resolution (JER) uncertainties, and 8 JER uncertainty components are included as NPs in the fit. 

\paragraph{}
The uncertainties in the b-tagging efficiencies measured in dedicated calibration analyses \cite{btag_cal} are also decomposed into uncorrelated components. The large number of components for b-tagging is due to the use of the full continuous b-tagging spectrum: The calibration of the distribution of the MVA discriminant for each of the individual working points considered for $b-$ $c-$ and light jets, and the correlations between them, produces a large number of systematic uncertainties.

%The systematic uncertainties associated with the signal and background processes are accounted for by varying the cross-section of each process within its uncertainty.

The full list of systematic uncertainties considered in the analysis is summarized in Tables
\ref{Tab:LeptonExperimentalSyst}, \ref{Tab:JetsExperimentalSyst} and \ref{Tab:BTagExperimentalSyst}.

\hspace{-1in}\begin{table}[H]
  \begin{center}
    {\small
    \begin{tabular}{|llcl|}
      \hline
      \multicolumn{4}{|c|}{\textbf{ Experimental Systematics on Leptons and $E_T^{miss}$} }\\
     % \hline
      Type     & Description  & Systematics Name & Application \\
     \hline
     \hline
     \multicolumn{4}{|c|}{\textbf{Trigger}}\\
     \hline
    Scale Factors    & Trigger Efficiency        & lepSFTrigTight$\_$MU(EL)$\_$SF$\_$Trigger$\_$STAT(SYST)    & Event Weight      \\
      \hline
      \multicolumn{4}{|c|}{\textbf{Muons}} \\
      \hline
      Efficiencies   & Reconstruction and        & lepSFObjTight$\_$MU$\_$SF$\_$ID$\_$STAT(SYST)              & Event Weight       \\
     & Identification    &       &        \\
      & Isolation                 &       lepSFObjTight$\_$MU$\_$SF$\_$Isol$\_$STAT(SYST)            & Event Weight       \\
         & Track To Vertex   	 & lepSFObjTight$\_$MU$\_$SF$\_$TTVA$\_$STAT(SYST )           & Event Weight       \\
    & Association  		 &   							      &           \\
     \pt Scale   & \pt Scale & MUONS$\_$SCALE    & \pt Correction     \\
     &   &   &           \\
      Resolution     & Inner Detector            & MUONS$\_$ID        					      & \pt Correction     \\
         & Energy Resolution      	 &     &         \\
    & Muon Spectrometer    	 & MUONS$\_$MS      & \pt Correction     \\
     & Energy Resolution         &       &        \\
     &   &   &         \\
     \hline
     \multicolumn{4}{|c|}{\textbf{Electrons}}\\
     \hline
     Efficiencies    & Reconstruction       	 & lepSFObjTight$\_$EL$\_$SF$\_$ID  			      & Event Weight   	    \\
     & Identification   & lepSFObjTight$\_$EL$\_$SF$\_$Reco       		      & Event Weight            \\
        & Isolation                 & lepSFObjTight$\_$EL$\_$SF$\_$Isol      		      & Event Weight        \\
       &   &   &          \\
     Scale Factor    & Energy  Scale             & EG$\_$SCALE$\_$ALL  					      & Energy Correction    \\
         	     &   &   &          \\
     Resolution      & Energy Resolution  	 & EG$\_$RESOLUTION$\_$ALL      			      & Energy Correction     \\
         	     &   &   &             \\
     \hline
     \multicolumn{4}{|c|}{\textbf{$E_T^{miss}$}}\\
     \hline
     Soft Tracks Terms         &             Resolution                   &      MET$\_$SoftTrk$\_$ResoPerp       &   \pt Correction  \\
                               &             Resolution                   &      MET$\_$SoftTrk$\_$ResoPara        &    \pt Correction    \\
                               &             Scale                        &      MET$\_$SoftTrk$\_$ScaleUp         &   \pt Correction     \\
                               &             Scale                        &      MET$\_$SoftTrk$\_$ScaleDown         &   \pt Correction     \\

     \hline
     
    \end{tabular}
   }
   \caption{\label{Tab:LeptonExperimentalSyst} Summary of experimental systematics considered for leptons and $E_T^{miss}$. Includes type, description, name of systematic as used in the fit, and mode of application. The mode of application indicates the systematic evaluation, e.g. as an  overall event re-weighting (Event Weight) or rescaling (\pt Correction).}
  \end{center}
\end{table}


\begin{table}[H]
  \begin{center}
    {\small
    \begin{tabular}{|llcc|}
      \hline
      \multicolumn{4}{|c|}{\textbf{ Experimental Systematics on Jets}} \\
      \hline
      Type     & Origin   & Systematics Name  & Application \\
      \hline
      Jet Vertex Tagger         &     & JVT      &        Event Weight          \\
     	&   &   &     \\
      Energy Scale              & Calibration Method              & JET$\_$21NP$\_$           &      \pt Correction         \\
       &   & JET$\_$EffectiveNP$\_$1-19     &    \pt Correction  \\
       &   &   &       \\
        & $\eta$ inter-calibration        & JET$\_$EtaIntercalibration$\_$Modelling    & \pt Correction          \\
     &                                 & JET$\_$EtaIntercalibration$\_$NonClosure   & \pt Correction      \\
     &                                 & JET$\_$EtaIntercalibration$\_$TotalStat    & \pt Correction      \\
    &   &   &        \\
     & High \pt jets                   & JET$\_$SingleParticle$\_$HighPt         &     \pt Correction             \\
     	&   &   &           \\
        & Pile-Up                         & JET$\_$Pileup$\_$OffsetNPV            &     \pt Correction             \\
        &       & JET$\_$Pileup$\_$OffsetMu             &     \pt Correction               \\
        &        & JET$\_$Pileup$\_$PtTerm       &     \pt Correction         \\
        &                                         & JET$\_$Pileup$\_$RhoTopology      &     \pt Correction             \\
    	&   &   &            \\
          & Non Closure                     & JET$\_$PunchThrough$\_$MC15    & \pt Correction    \\
    	&   &   &       \\
         & Flavour                         & JET$\_$Flavor$\_$Response          &   \pt Correction            \\
     &         & JET$\_$BJES$\_$Response          &   \pt Correction           \\
           &                                 & JET$\_$Flavor$\_$Composition        &    \pt Correction             \\
         	&   &   &          \\
      Resolution         	&                                 & JET$\_$JER$\_$EffectiveNP$\_$NP$\_$1-8          &  Event Weight       \\
        			&   &   &          \\
        			
    \hline

     \end{tabular}
    }
    \caption{\label{Tab:JetsExperimentalSyst} Jet systematics take into account effects of jets calibration method, $\eta$ inter-calibration, high \pt jets, pile-up, and flavor response. They are all diagonalised into effective parameters.}
 \end{center}
\end{table}

\begin{table}[H] 
  \begin{center}
    {\small
    \begin{tabular}{|llc|}
      \hline
     \multicolumn{3}{|c|}{\textbf{Experimental Systematics on b-tagging}} \\
      \hline
      Type     & Origin   & Systematic Name \\
     \hline
     &   &                \\
      Scale Factors & DL1r b-tagger efficiency & DL1r$\_$Continuous$\_$EventWeight$\_$B0-29 \\
      &    on b originated jets in bins of $\eta$  &   \\
      &   &                \\
      &    DL1r b-tagger efficiency & DL1r$\_$Continuous$\_$EventWeight$\_$C0-19  \\
      &    on c originated jets in bins of $\eta$    &     \\
      &   &   \\
      &    DL1r b-tagger efficiency & DL1r$\_$Continuous$\_$EventWeight$\_$Light0-79           \\
      &    on light flavoured originated jets         &   \\
     &     in bins of $\eta$ and \pt      &    \\
         &   &             \\
     &    DL1r b-tagger                        & DL1r$\_$Continuous$\_$EventWeight$\_$extrapolation  \\
     &    extrapolation efficiency    &         DL1r$\_$Continuous$\_$EventWeight$\_$extrapolation$\_$from$\_$charm             \\
     \hline
    \end{tabular}
    }
    \caption{\label{Tab:BTagExperimentalSyst} Summary of experimental systematics to be included for $b$-tagging of jets in the analysis, using the continuous DL1r tagging algorithm. All of the b-tagging related systematics are applied as event weights. From left: type, description, and the name of systematic used in the fit.}
  \end{center}
\end{table}

\paragraph{}
Theoretical uncertainties applied to MC predictions, including cross section, PDF, and scale uncertainties are taken from theory calculations, with the exception of non-prompt and diboson backgrounds. The cross-section uncertainty on tZ is taken from \cite{tZ_paper}. Derivation of the non-prompt background uncertainties, Z+jets and $t\bar{t}$, is explained in detail in Section \ref{sec:fakes}. These normalization uncertainties are chosen so as to account for the complete uncertainty in the non-prompt contribution, and therefore no additional modelling uncertainties are considered for Z+jets and $t\bar{t}$.

The other VV + heavy flavor processes (namely VV+b and VV+charm, which primarily consist of ZZ events) are also poorly understood, because these processes involve the same physics as WZ + heavy flavor, and have also not been measured. Therefore, an arbitrary 50\% uncertainty is applied to those samples, as it is found to have little impact on the significance of the final result: Fits were performed with this uncertainty decreased to 25\% and increased to 75\% and no change in the final result was observed.

The theory uncertainties applied to the predominate background estimates are summarized in Table \ref{tab:xsecUnc}. 

\begin{table}[H]
{\footnotesize
\centering
\begin{tabular}{c|c}
\hline
Process                 & X-section [\%]                \\
\hline
WZ                      & QCD Scale: $^{+3.7}_{-3.4}$ \\                                                                     
                        & PDF($+\alpha_S$): $\pm 3.1$ \\
%\hline
tZ                      & X-sec: $\pm 15.2$    \\ 
%                        & QCD Scale: $^{+5.2}_{-1.3}$ \\
%                        & PDF($+\alpha_S$): $\pm 1.2$ \\
\hline
\ttbar H                & QCD Scale:$^{+5.8}_{-9.2}$    \\
(aMC$@$NLO$+$Pythia8)   & PDF($+\alpha_S$): $\pm$ 3.6   \\
\hline
\ttbar Z                & QCD Scale:$^{+9.6}_{-11.3}$   \\
(aMC$@$NLO$+$Pythia8)   & PDF($+\alpha_S$): $\pm$ 4     \\
\hline
\ttbar W                & QCD Scale:$^{+12.9}_{-11.5}$  \\
(aMC$@$NLO$+$Pythia8)   & PDF($+\alpha _S$): $\pm$ 3.4  \\
\hline
VV + b/charm            & $\pm$ 50                      \\
(Sherpa 2.2.1)          &                               \\
\hline
VV + light              & $\pm$ 6                      \\   
(Sherpa 2.2.1)          &                               \\
\hline
$t\bar{t}$              & $\pm$ 20 \\                          
\hline
Z + jets                & $\pm$ 25 \\
\hline
Others                  & $\pm$ 50 \\
\hline
\hline
\end{tabular}



\caption{Summary of theoretical uncertainties for MC predictions in the analysis.}
\label{tab:xsecUnc}
}
\end{table}

Due to its importance as a background, additional modelling uncertainties are considered for tZ. Alternative tZ samples with variations in scale (DSID 412064-5) and shower modelling (DSID 501046) are included as systematics..

\paragraph{}
The fit involves varying the overall normalization of signal templates over the regions described in Section \ref{subsec:regions}, which are defined by the flavor and number of associated jets at truth-level. The modelling of these template shapes therefore significantly impacts the final result. Additional signal uncertainties, probing the shape of the signal templates as well as the rate of migrations between the number of truth-jets and reconstructed jets, are estimated by comparing estimates from the nominal Sherpa WZ samples with alternative WZ samples generated with \textsc{Powheg}+\textsc{Pythia8} (DSID 361601). Separate systematics are included in the fit for WZ + $b$, WZ + $c$ and WZ + light, where the distribution among each of the fit regions is varied based on the prediction of the Powheg sample.

The variations in the signal templates are shown in Figures \ref{fig:powheg1j} and \ref{fig:powheg2j}. Each of these plots is normalized to unity in order to capture the relevant differences in shape. 

\begin{figure}[H]
    \centering
    \subfigure[]{\includegraphics[width=.53\linewidth]{signalModel/plotsNotZ/bjets1j.pdf}}%                  
    \subfigure[]{\includegraphics[width=.53\linewidth]{signalModel/plotsNotZ/charm1j.pdf}}\\
    \subfigure[]{\includegraphics[width=.53\linewidth]{signalModel/plotsNotZ/light1j.pdf}}%
    \caption{Comparison between Sherpa and Powheg predictions of the distribution of (a) WZ + b, (b) WZ + charm, and (c) WZ + light among the various b-tag WPs used in the 1-jet fit.}
\label{fig:powheg1j}
\end{figure}

\begin{figure}[H]
    \centering
    \subfigure[]{\includegraphics[width=.53\linewidth]{signalModel/plotsNotZ/bjets2j.pdf}}%                     
    \subfigure[]{\includegraphics[width=.53\linewidth]{signalModel/plotsNotZ/charm2j.pdf}}\\                           
    \subfigure[]{\includegraphics[width=.53\linewidth]{signalModel/plotsNotZ/light2j.pdf}}%
    \caption{Comparison between Sherpa and Powheg predictions of the distribution of (a) WZ + b, (b) WZ + charm, and (c) WZ + light among the various b-tag WPs used in the 2-jet fit.}
\label{fig:powheg2j}
\end{figure}

Separate systematics are included in the fit for WZ + b, WZ + charm and WZ + light, where the distribution among each of the fit regions is varied based on the prediction of the Powheg sample.

\paragraph{}
A similar approach is taken to account for uncertainties in migrations between the number of reco and truth jets. The fraction of events with 1 truth jet which fall into the 1 jet bin versus the 2 jet bin at reco level is compared for Sherpa and Powheg. The same is done for events with 2 truth jets. This comparison is shown is figure \ref{fig:migration12}.

\begin{figure}[H]
    \centering
    \subfigure[]{\includegraphics[width=.53\linewidth]{signalModel/truthJets/plots/migrations_bjets.pdf}}%                
    \subfigure[]{\includegraphics[width=.53\linewidth]{signalModel/truthJets/plots/migrations_charm.pdf}}\\    
    \subfigure[]{\includegraphics[width=.53\linewidth]{signalModel/truthJets/plots/migrations_light.pdf}}%
    \caption{Comparison between Sherpa and Powheg predictions for truth jet migrations between the 1 and 2 jet reco bins for (a) WZ + b, (b) WZ + charm, and (c) WZ +\ light}
\label{fig:migration12}
\end{figure}

A systematic is included where events are shifted between the 1-jet and 2-jet regions based on the differences between these two shapes. This is done independently for each of the WZ + b, WZ + charm, and WZ + light templates.

Additional systematics are included to account for the uncertainty in the contamination of 0 jet and 3 or more jet events (at truth level) in the 1 and 2 reco jet bins. Because these events fall outside the scope of this measurement, these events are included as a background. As such, a normalization, rather than a shape, uncertainty is applied for this background.

The number of WZ events with 0-jets and $>=$3-jets in the reconstructed 1-jet and 2-jet regions are compared for Sherpa and Powheg, as seen in figure \ref{fig:overflow}. These differences are taken as separate normalization systematics on the yield of WZ+0-jet and WZ+$>=$3-jet events.

\begin{figure}[H]
    \centering
    \includegraphics[width=.62\linewidth]{signalModel/truthJets/plots/overflow.pdf}
    \caption{Comparison between Sherpa and Powheg predictions for 0 and $>=$3 truth jet contributions in the 1 and 2 jet reco bins}
\label{fig:overflow}
\end{figure} 
