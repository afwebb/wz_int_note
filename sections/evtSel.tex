
Event are required to pass a preselection described in Section \ref{subsec:presel}. Those that pass this preselection are divided into various fit regions described in Section \ref{subsec:regions}, based on the number of jets in the event, and the b-tag score of those jets.

%--------------------------- 
\subsection{Event Preselection}
\label{subsec:presel}
%--------------------------- 

Events are required to include exactly three reconstructed light leptons passing the requirement described in \ref{sec:obj}, which have a total charge of $\pm$1. As the opposite sign lepton is found to be prompt the vast majority of the time \cite{ttH_comb}, it is required to have $p_T > 10$ GeV, while the same sign leptons are required to have $p_T > 20$ GeV to reduce the contribution of non-prompt leptons.

The invariant mass of at least one pair of opposite sign, same flavor leptons is required to fall within 10 GeV of the mass of the Z boson, 91.2 GeV. Events where one of the opposite sign pairs have an invariant mass less than 12 GeV are rejected in order to suppress low mass resonances. %Further, events where the trilepton mass falls within 5 GeV of the Z mass are rejected to remove Z events that include conversions.

An additional requirement is placed on the missing transverse energy, $E^{miss}_T$ > 20 GeV. The transverse mass of the $W$ candidate, defined as $\sqrt{2p_T^{lep}E^{miss}_T*(1-\cos(\phi_{lep}-\phi_{E^{miss}_T}))}$, is required to be greater than 30 GeV. Here $E^{miss}_T$ is the missing transverse energy, and the lepton considered is the lepton not included in the Z-candidate. 

Events are required to have exactly one or two reconstructed. Events with more than two jets are rejected in order to reduce the contribution of backgrounds such as $t\bar{t}Z$ and $t\bar{t}W$, which tend to have higher jet multiplicity. 

The WZ events are split into WZ + $b$, WZ + c, and WZ + light based on the truth flavor of the associated jet in the event, as determined by the presence of a $b$- or $c$-hadron within $R=0.3$ of the jet. In this ordering b-jet supersede charm, which supersides light. That is, WZ + light events contain no charm and no b jets at truth level, WZ + c contain at least one truth charm and no b-jets, and WZ + $b$ contains at least one truth b-jet. 

%---------------------------                                                                                         
\subsection{Fit Regions}
\label{subsec:regions}
%--------------------------- 

Once preselection has been applied, the remaining events are categorized into one of twelve orthogonal regions. The regions used in the fit are summarized in Table \ref{tab:regions}.

\begin{table}[H] 
\centering
\caption{A list of the regions used in the fit and the selection used for each.}
\begin{tabular}{l|l}
\hline\hline
Region & Selection            \\
\hline
\hline
1j, <85\%       & $N_{jets}$ = 1, nJets\_DL1r\_85 = 0            \\
1j, 85\%-77\%   & $N_{jets}$ = 1, nJets\_DL1r\_85 = 1, nJets\_DL1r\_77=0                     \\
1j, 77\%-70\%   & $N_{jets}$ = 1, nJets\_DL1r\_77 = 1, nJets\_DL1r\_70=0                     \\
1j, 70\%-60\%   & $N_{jets}$ = 1, nJets\_DL1r\_70 = 1, nJets\_DL1r\_60=0                      \\
1j, >60\%       & $N_{jets}$ = 1, nJets\_DL1r\_60 = 1, tZ BDT > 0.12 \\
1j tZ CR        & $N_{jets}$ = 1, nJets\_DL1r\_60 = 1, tZ BDT < 0.12 \\
2j, <85\%       & $N_{jets}$ = 2, nJets\_DL1r\_85 = 0                    \\
2j, 85\%-77\%   & $N_{jets}$ = 2, nJets\_DL1r\_85 >= 1, nJets\_DL1r\_77=0                     \\
2j, 77\%-70\%   & $N_{jets}$ = 2, nJets\_DL1r\_77 >= 1, nJets\_DL1r\_70=0                     \\
2j, 70\%-60\%   & $N_{jets}$ = 2, nJets\_DL1r\_70 >= 1, nJets\_DL1r\_60=0                      \\
2j, >60\%       & $N_{jets}$ = 2, nJets\_DL1r\_60 >= 1, tZ BDT > 0.12 \\
2j tZ CR        & $N_{jets}$ = 2, nJets\_DL1r\_60 >= 1, tZ BDT < 0.12 \\
\hline\hline
\end{tabular}
\label{tab:regions}
\end{table}

The working points discussed in Section \ref{sec:obj} are used to separate events into fit regions based on the highest working point reached by a jet in each event. Because the background composition differs significantly based on the number of b-jets, events are further subdivided into 1-jet and 2-jet regions in order to minimize the impact of background uncertainties.

An unfolding procedure is performed to account for differences in the number of reconstructed jets compared to the number of truth jets in each event. In order to account for migration of WZ+1-jet and WZ+2-jet events between the 1-jet and 2-jet bins at reco level, the signal samples are separated based on the number of truth jets. Events with 0 jets or more than 3 jets at truth level, yet fall within one of the categories listed in Table \ref{tab:regions}, are categorized as WZ + other, and treated as background. The composition of the number of truth jets in each reco jet bin is taken from MC, with uncertainties in these estimates described in detail in Section \ref{sec:sys}. 

An additional tZ control region is created based on the BDT described in Section \ref{sec:tZ_bdt}. The region with 1-jet passing the 60\% working point is split in two - a signal enriched region of events with a BDT score greater than 0.12, and a tZ control region including events with less than 0.12. This cutoff is optimized for significance of $WZ$ + b.

%---------------------------
\subsection{Non-Prompt Lepton Estimation}
\label{sec:fakes}
%---------------------------

Two processes that act as sources of non-prompt leptons appear in the analysis: $t\bar{t}$ and $Z$+jet production both produce two prompt leptons, but can meet the selection of this analysis when an additional non-prompt lepton appears in the event. The contribution of these processes is estimated with Monte Carlo simulations, which are validated using non-prompt enriched regions. These validation regions are used to derive correction factors and uncertainties for the non-prompt contribution.

$t\bar{t}$ events can produce two prompt leptons from the decay of each of the tops. These top decays produce two b-quarks, the decay of which can produce additional non-prompt leptons, which occasionally pass the event preselection. In order to validate that the Monte Carlo accurately simulates this process accurately, the MC prediction in a non-prompt $t\bar{t}$ enriched validation region is compared to data.

The $t\bar{t}$ validation region is similar to the preselection region - three leptons meeting the criteria described in Section \ref{sec:evt_selection} are required, and the requirements on $E_T^{miss}$ remain the same. However, the selection requiring that a lepton pair form a Z-candidate are reversed. Events where the invariant mass of any two opposite sign, same flavor leptons falls within 10 GeV of 91.2 GeV are rejected. This ensures the $t\bar{t}$ validation region is orthogonal to the preselection region. 

Further, because the jet multiplicity of $t\bar{t}$ events tends to be higher than WZ + jets, the number of jets in each event is required to be greater than 1. As b-jets are almost invariably produced from top decays, at least one b-tagged jet passing the 70\% DL1r WP in each event is required. 

Data is compared to MC predictions in the region for a variety of kinematic variable, as well as various b-tag WPs. A constant normalization discrepency between data and MC predictions of approximately 10\% is found, which is accounted for by applying a constant correction factor of 0.9 to the $t\bar{t}$ MC prediction. Once this correction factor has been applied, no significant modelling discrepencies, either in terms of shape or overall yield, are found in any of the kinematic distributions considered. As data and MC are found to agree within 20\% for each of the b-tag WPs considered, a 20\% systematic uncertainty on the $t\bar{t}$ prediction is included for the analysis.

Similar to $t\bar{t}$, a $Z$+jets validation region is produced in order to validate the MC predictions. The lepton requirements remain the same as the preselection region. Because no neutrinos are present for this process, the $E_T^{miss}$ cut is reversed, requiring $E_T^{miss}$ < 30 GeV. This also ensures this validation region is orthogonal to the preselection region. Further, the number of jets in each event is required to be greater than or equal to one.

While there is general agreement between data and MC, the shape of the $p_T$ spectrum of the lepton from the W candidate is found to differ. This is the lepton not included in the Z-candidate, and in the case of Z+jets, this lepton is most often the non-prompt lepton. To account for this discrepency, a variable correction factor is applied to Z+jets. $\chi^2$ minimization of the W lepton $p_T$ spectrum is performed to derive a correction factor.

The systematic uncertainty in the Z + jets prediction is evaluated by comparing data to MC for each of the continuous b-tag WPs. For each of the regions considered, the data falls within 25\% of the MC prediction once this correction factor has been applied. Therefore, a 25\% systematic uncertainty is applied to Z + jets in the analysis.

