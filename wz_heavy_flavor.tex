%-------------------------------------------------------------------------------
% This file provides a skeleton ATLAS note.
% \pdfinclusioncopyfonts=1
% This command may be needed in order to get \ell in PDF plots to appear. Found in
% https://tex.stackexchange.com/questions/322010/pdflatex-glyph-undefined-symbols-disappear-from-included-pdf
%-------------------------------------------------------------------------------
% Specify where ATLAS LaTeX style files can be found.
\newcommand*{\ATLASLATEXPATH}{latex/}
% Use this variant if the files are in a central location, e.g. $HOME/texmf.
% \newcommand*{\ATLASLATEXPATH}{}
%-------------------------------------------------------------------------------
%\documentclass[USenglish,texlive=2016,cernpreprint,NOTE,paper=A4]{\ATLASLATEXPATH atlasdoc}
\documentclass[NOTE, atlasdraft=true, texlive=2016, UKenglish]{\ATLASLATEXPATH atlasdoc}
\usepackage{float}
\usepackage{euler}\usepackage{pgf}
%\usepackage[backend=biber]{biblatex}
%\usepackage{natbib}
\usepackage{geometry}
\usepackage{pdflscape}
\DeclareOldFontCommand{\bf}{\normalfont\bfseries}{\mathbf}
% The language of the document must be set: usually UKenglish or USenglish.
% british and american also work!
% Commonly used options:
%  atlasdraft=true|false This document is an ATLAS draft.
%  texlive=YYYY          Specify TeX Live version (2016 is default).
%  coverpage             Create ATLAS draft cover page for collaboration circulation.
%                        See atlas-draft-cover.tex for a list of variables that should be defined.
%  cernpreprint          Create front page for a CERN preprint.
%                        See atlas-preprint-cover.tex for a list of variables that should be defined.
%  NOTE                  The document is an ATLAS note (draft).
%  PAPER                 The document is an ATLAS paper (draft).
%  CONF                  The document is a CONF note (draft).
%  PUB                   The document is a PUB note (draft).
%  BOOK                  The document is of book form, like an LOI or TDR (draft)
%  txfonts=true|false    Use txfonts rather than the default newtx
%  paper=a4|letter       Set paper size to A4 (default) or letter.

%-------------------------------------------------------------------------------
% Extra packages:
%\usepackage{\ATLASLATEXPATH atlaspackage}
\usepackage[subfigure]{\ATLASLATEXPATH atlaspackage}
%\usepackage[biblatex=false]{\ATLASLATEXPATH atlaspackage}
% Commonly used options:
%  biblatex=true|false   Use biblatex (default) or bibtex for the bibliography.
%  backend=bibtex        Use the bibtex backend rather than biber.
%  subfigure|subfig|subcaption  to use one of these packages for figures in figures.
%  minimal               Minimal set of packages.
%  default               Standard set of packages.
%  full                  Full set of packages.
%-------------------------------------------------------------------------------
% Style file with biblatex options for ATLAS documents.
%\usepackage[biblatex=true]{\ATLASLATEXPATH atlasbiblatex}

% Package for creating list of authors and contributors to the analysis.
\usepackage{\ATLASLATEXPATH atlascontribute}

% Useful macros
\usepackage{\ATLASLATEXPATH atlasphysics}
% See doc/atlas_physics.pdf for a list of the defined symbols.
% Default options are:
%   true:  journal, misc, particle, unit, xref
%   false: BSM, heppparticle, hepprocess, hion, jetetmiss, math, process, other, texmf
% See the package for details on the options.

% Files with references for use with biblatex.
% Note that biber gives an error if it finds empty bib files.
\addbibresource{wz_heavy_flavor.bib}
\addbibresource{bib/ATLAS.bib}
\addbibresource{bib/CMS.bib}
\addbibresource{bib/ConfNotes.bib}
\addbibresource{bib/PubNotes.bib}

% Paths for figures - do not forget the / at the end of the directory name.
\graphicspath{{logos/}{figures/}}

% Add you own definitions here (file wz_heavy_flavor-defs.sty).
\usepackage{wz_heavy_flavor-defs}

%-------------------------------------------------------------------------------
% Generic document information
%-------------------------------------------------------------------------------

% Title, abstract and document 
%-------------------------------------------------------------------------------
% This file contains the title, author and abstract.
% It also contains all relevant document numbers used for an ATLAS note.
%-------------------------------------------------------------------------------

% Title
\AtlasTitle{WZ + Heavy Flavor Production in pp collisions at $\sqrt{s}$ = 13 TeV}

% Draft version:
% Should be 1.0 for the first circulation, and 2.0 for the second circulation.
% If given, adds draft version on front page, a 'DRAFT' box on top of each other page, 
% and line numbers.
% Comment or remove in final version.
\AtlasVersion{0.1}

% Abstract - % directly after { is important for correct indentation
\AtlasAbstract{%
        A measurement of WZ produced with an associated heavy flavor jet is performed using 140 $fb^{-1}$ of proton-proton collision data at $\sqrt{s} =$ 13 TeV from the ATLAS experiment at the LHC. The measurement is performed in the fully leptonic decay mode, $WZ\rightarrow l\nu ll$. The cross-section of WZ + b-jets is measured to be $X\pm X\pm X$, while the cross-section of WZ + charm is measured as X, with a correlation of X between the two processes. 
}

% Author - this does not work with revtex (add it after \begin{document})
\author{The ATLAS Collaboration}

% Authors and list of contributors to the analysis
% \AtlasAuthorContributor also adds the name to the author list
% Include package latex/atlascontribute to use this
% Use authblk package if there are multiple authors, which is included by latex/atlascontribute
% \usepackage{authblk}
% Use the following 3 lines to have all institutes on one line
% \makeatletter
% \renewcommand\AB@affilsepx{, \protect\Affilfont}
% \makeatother
% \renewcommand\Authands{, } % avoid ``. and'' for last author
% \renewcommand\Affilfont{\itshape\small} % affiliation formatting
% \AtlasAuthorContributor{First AtlasAuthorContributor}{a}{Author's contribution.}
% \AtlasAuthorContributor{Second AtlasAuthorContributor}{b}{Author's contribution.}
% \AtlasAuthorContributor{Third AtlasAuthorContributor}{a}{Author's contribution.}
% \AtlasContributor{Fourth AtlasContributor}{Contribution to the analysis.}
% \author[a]{First Author}
% \author[a]{Second Author}
% \author[b]{Third Author}
% \affil[a]{One Institution}
% \affil[b]{Another Institution}

% If a special author list should be indicated via a link use the following code:
% Include the two lines below if you do not use atlasstyle:
% \usepackage[marginal,hang]{footmisc}
% \setlength{\footnotemargin}{0.5em}
% Use the following lines in all cases:
% \usepackage{authblk}
% \author{The ATLAS Collaboration%
% \thanks{The full author list can be found at:\newline
%   \url{https://atlas.web.cern.ch/Atlas/PUBNOTES/ATL-PHYS-PUB-2017-007/authorlist.pdf}}
% }

% ATLAS reference code, to help ATLAS members to locate the paper
\AtlasRefCode{GROUP-2017-XX}

% ATLAS note number. Can be an COM, INT, PUB or CONF note
% \AtlasNote{ATLAS-CONF-2017-XXX}
% \AtlasNote{ATL-PHYS-PUB-2017-XXX}
% \AtlasNote{ATL-COM-PHYS-2017-XXX}

% Author and title for the PDF file
\hypersetup{pdftitle={ATLAS document},pdfauthor={The ATLAS Collaboration}}

%-------------------------------------------------------------------------------
% Content
%-------------------------------------------------------------------------------
\begin{document}

\maketitle

\tableofcontents

% List of contributors - print here or after the Bibliography.
%\PrintAtlasContribute{0.30}
\clearpage

%------------------------------------------------------------------------------
\section{Changes and outstanding items}
\label{sec:changes}
%------------------------------------------------------------------------------

\subsection{Changelog}

This is version 4

\subsubsection{Changes relative to v3}
\begin{itemize}
  \item Merged introduction into executive summary, including unblinding details and list of SRs/CRs used
  \item listed ptag used (p4133), and release (AB 21.2.127)
  \item Included table ref{tab:xsecUnc} listing x-sec uncertainties used
  \item Removed selection criteria listed in table \ref{tbl:tightleps} (QMisID, AmbiguityType) that were removed from the analysis
  \item specified variable used to make truth jet flavor determination (HadronConeExclTruthLabelID)
  \item fixed bug in MtLepMet calculation, updated selection/fits to account for this
  \item Included plots of MtLepMet and PtZ, swapped lep 1 and 2 $p_T$ plots for lep W and lep Z plots
  \item updated tZ BDT training to reduce overfitting, updated plots to include error bars, feature importance
  \item updated table \ref{tab:regions} to clarify selection, fix the tZ\_BDT cut used
  \item replace a few broken ntuples which included large weight events
  \item include DL1r distribution for Z+jets and $t\bar{t}$ VRs
  \item Expanded section on facts, included information on derived scale factors from VRs.
\end{itemize}

\subsubsection{Changes relative to v2}

\begin{itemize}
    \item Added alternate VBS samples to include missing b-jet diagrams
    \item Included a section on tZ interference effects, \ref{subsec:interference}. 
    \item Updated to reflect changes for 2018, including the move to PFlow jets, DL1r, updated trigger, and updated AnalysisBase version (now 21.2.127)
    \item Revised fit regions, using seperate 1-jet and 2-jet fits, with all 2-j regions included
    \item updated plots for tZ BDT, added details about the model
    \item Included truth jet information
\end{itemize}

\subsubsection{Changes relative to v1}

\begin{itemize}
    \item Added GRL list
    \item Fixed latex issue in line 92, typo in line 172
    \item Added tables \ref{tbl:selection} and \ref{tbl:tightleps}, summarizing the event and object selection
    \item Added table \ref{tbl:dsids}, which includes the DSID of samples used
    \item Included reference to WZ inclusive paper in introduction
\end{itemize}

\subsection{Outstanding Items}

\begin{itemize}
    \item Unblind, update plots and fits to include data
    \item Include truth jet studies
    \item Add cross-section, significance once unblinded
\end{itemize}



\clearpage

%-------------------------------------------------------------------------------                                     
\section{Executive Summary}
\label{sec:summary}

The production of $WZ$ in association with a heavy flavor jet represents an important background for many major analyses. This includes any process with leptons and b-jets in the final state, such as $t\bar{t}H$, $t\bar{t}W$, and $t\bar{t}Z$. While precise measurements have been made of $WZ$ production \cite{WZ_36}, $WZ$ + heavy flavor remains poorly understood. This is largely because the QCD processes involved in the production of the b-jet make it difficult to simulate accurately. This introduces a large uncertainty for analyses that include this process as a background.

Motivated by its relevance to the $t\bar{t}H$ multilepton analysis, we perform a study of the fully leptonic decay mode of this channel; that is, events where both the W and Z decay leptonically. Because WZ has no associated jets at leading order, while the major backgrounds for this channel tend to have high jet multiplicity, events with more than two jets are rejcted. This gives a final state signature of three leptons and one or two jets.

Events that meet this selection criteria are sorted into pseudo-continuous b-tagging regions based on the DL1r b-tag score of their associated jets. This is done to separate $WZ$ + b-jet events from $WZ$ + charm and $WZ$ + light jets. These regions are fit to data in order make a more accurate estimate of the contribution of $WZ$ + heavy-flavor, where heavy-flavor jets include b-jets and charm jets. The full Run-2 dataset collected by the ATLAS detector, representing 139 $fb^{-1}$ of data from pp collisions at $\sqrt{s} = 13$ TeV, is used for this study.

All backgrounds are accounted for using Monte Carlo, with the simulation of non-prompt lepton backgrounds - Z+jets and $t\bar{t}$ - validated using non-prompt Validation Regions.

Section \ref{sec:data} details the data and Monte Carlo (MC) samples used in the analysis. The reconstruction of various physics objects is described in section \ref{sec:obj}. Section \ref{sec:evt_selection} describes the event selection applied to these samples, along the definitions of the various regions used in the fit. The multivariate analysis techniques used to separate the tZ background from WZ + heavy flavor are described in section \ref{sec:tZ_bdt}. Section \ref{sec:sys} describes the various sources of systematic uncertainties considered in the fit. Finally, the results of the analysis are summarized in section \ref{sec:results}, followed by a brief conclusion in section \ref{sec:conclusion}.

The current state of thee analysis shows blinded results for thee full Run 2 dataset. Regions containing $>$5\% WZ+b events are blinded, and results are from Asimov, MC only fits. In addition to adding some additional information to this note, remaining tasks include performing WZ/tZ interference studies, finalizing the presentation of results, and unblinding.

%-------------------------------------------------------------------------------

%-------------------------------------------------------------------------------
%\section{Introduction}
%\label{sec:intro}
%
The production of $WZ$ in association with a heavy flavor jet represents an important background for many major analyses. This includes any process with multiple leptons and b-jets in the final state, such as $t\bar{t}H$, $t\bar{t}W$, and $t\bar{t}Z$. While precise measurements have been made of inclusive $WZ$ production \cite{WZ_36}, $WZ$ + heavy flavor remains poorly understood. This is largely because the QCD processes involved in the production of the b-jet make it difficult to simulate accurately. This introduces a large uncertainty for analyses that include this process as a background.  

We perform a study of the fully leptonic decay mode of this channel; that is, events where both the W and Z decay leptonically. Because WZ has no associated jets at leading order, while the major backgrounds for this channel tend to have high jet multiplicity, events with more than two jets are rejected. This gives a final state signature of three leptons and one or two jets.

Events that meet a preselection criteria are sorted into regions based on the b-taggin score of their associated jets. This is done to separate $WZ$ + b-jet events from $WZ$ + charm and $WZ$ + light jets. These regions are fit to data in order make a more accurate estimate of the contribution of $WZ$ + heavy-flavor, where heavy-flavor jets include b-jets and charm jets. The full Run-2 dataset collected by the ATLAS detector, representing 139 $fb^{-1}$ of data from pp collisions at $\sqrt{s} = 13$ TeV, is used for this study.

The fiducial volume at particle level is defined based on the number of stable leptons and jets in each event. Three light leptons with total charge $\pm$1 and one or two associated jets are required. Only leptons which do not originate from hadron or $\tau$ decays are considered. The phase space definitions use dressed kinematics of the final state particles. Leptons are dressed by summing the momentum of photons within a cone of $\Delta R < 0.1$ of the lepton to correct the leptons energy. Particle level jets are reconstructed using the anti-$k_t$ algorithm with a radius of $R=0.4$. The kinematic selection applied to these objects is summarized below:

\begin{itemize}
\item Three light leptons with total charge $\pm$1, $|\eta| < 2.5$
\item OS lepton with \pt$>$10 GeV, SS leptons with \pt$>$20 GeV
\item One OSSF lepton pair with $|M(ll)-91.2\ GeV| < 10\ GeV$
\item One or two associated truth jets with $p_T >$25 GeV, $|\eta| < 2.5$, $R<0.4$
%\item At least one truth b-jet or one charm jet
\end{itemize}

The result of the fit is used to extract the cross-section in this fiducial region for WZ + $b$ and WZ + $c$ with one associated jet, and WZ + $b$ and WZ + $c$ with two associated jets, where the number and flavor of the jets is determined at particle level. Events with both charm and b-jets are counted as WZ + $b$. The analysis reports a cross-section measurement of WZ + $b$ and WZ + $c$, along with their correlations, for both 1-jet and 2-jet exclusive regions. Normalization factors, representing how the MC prediction differs from the observed result, are also reported.

Section \ref{sec:data} details the data and Monte Carlo (MC) samples used in the analysis. The reconstruction of various physics objects is described in Section \ref{sec:obj}. Section \ref{sec:evt_selection} describes the event selection applied to these samples, along the definitions of the various regions used in the fit. The multivariate analysis techniques used to separate the tZ background from WZ + heavy flavor are described in Section \ref{sec:tZ_bdt}. Section \ref{sec:sys} describes the various sources of systematic uncertainties considered in the fit. Finally, the results of the analysis are summarized in Section \ref{sec:results}, followed by a brief conclusion in Section \ref{sec:conclusion}.



%-------------------------------------------------------------------------------

%-------------------------------------------------------------------------------
\section{Data and Monte Carlo Samples}
\label{sec:data}

Both data and Monte Carlo samples used in this analysis were prepared in the \verb|xAOD| format, which was used to produce a \verb|DxAOD| sample in the \verb|HIGG8D1| derivation framework. The \verb|HIGG8D1| framework is designed for the $t\bar{t}H$ multi-lepton analysis, which targets events with multiple leptons as well as tau hadrons. This framework skims the dataset to remove unneeded variables as well as entire events. Events are removed from the derivations that do not meet the following selection:

\begin{itemize}
    \item at least two light leptons within a range $|\eta|$<2.6, with leading lepton $p_{T}$ > 15 GeV and subleading lepton $p_{T}$ > 5 GeV
    \item at least one light lepton with $p_{T}$ > 15 GeV within a range $|\eta|$<2.6, and at least two hadronic taus with $p_{T}$ > 15 GeV.
\end{itemize}

Samples were then generated from these \verb|HIGG8D1| derivations with p-tag of p4134 using AnalysisBase version 21.2.127 modified to include custom variables..

\subsection{Data Samples}

The study uses a sample of proton-proton collision data collected by the ATLAS detector from 2015 through 2018 at an energy of $\sqrt{s} = 13$ TeV, which represents an integrated luminosity of 139 $fb^{-1}$. This data set was collected with a bunch crossing rate of 25 ns. All data used in this analysis was verified by data quality checks, having been included in the following Good Run Lists: 
\begin{itemize}
    \item data15\_13TeV.periodAllYear\_DetStatus-v79-repro20-02\_DQDefects-00-02-02\\\_PHYS\_StandardGRL\_All\_Good\_25ns.xml
    \item data16\_13TeV.periodAllYear\_DetStatus-v88-pro20-21\_DQDefects-00-02-04\\\_PHYS\_StandardGRL\_All\_Good\_25ns.xml 
    \item data17\_13TeV.periodAllYear\_DetStatus-v97-pro21-13\_Unknown\_PHYS\_StandardGRL\\\_All\_Good\_25ns\_Triggerno17e33prim.xml 
    \item data18\_13TeV.periodAllYear\_DetStatus-v102-pro22-04\_Unknown\_PHYS\_StandardGRL\\\_All\_Good\_25ns\_Triggerno17e33prim.xml
\end{itemize}

Runs included from the AllYear period containers are included.

\subsection{Monte Carlo Samples}

Several different generators were used to produce Monte Carlo simulations of the signal and background processes. For all samples, the response of the ATLAS detector is simulated using Geant4. The WZ signal samples are simulated using Sherpa 2.2.2 \cite{sherpa}. Specific information about the Monte Carlo samples being used can be found in table \ref{tbl:evgen}. A list of the specific samples used by data set ID is shown in table \ref{tbl:dsids}.

\begin{table}[hbt!]
\begin{center}
\caption{\label{tbl:evgen} The configurations used for event generation of signal and background processes, including the event generator, matrix element (ME) order, parton shower algorithm, and parton distribution functin (PDF). }
 \resizebox{\textwidth}{!}{
\begin{tabular}{llllll}
\hline\hline
Process & Event generator & ME order & Parton Shower & PDF   \\
\hline
$WZ$, $VV$ & \textsc{Sherpa} 2.2.2
& MEPS NLO & \textsc{Sherpa} & CT10 \\
$t Z$ & \textsc{MG5\_aMC} & LO & \textsc{Pythia} 6  & CTEQ6L1  \\
$\ttbar W$ & \textsc{MG5\_aMC} & NLO & \textsc{Pythia} 8 & NNPDF 3.0 NLO \\
& (\textsc{Sherpa} 2.1.1) & (LO multileg) & (\textsc{Sherpa}) & (NNPDF 3.0 NLO)  \\
$\ttbar (Z/\gamma^* \to ll)$ & \textsc{MG5\_aMC} & NLO & \textsc{Pythia} 8 & NNPDF 3.0 NLO  \\
$t\bar{t}H$ & \textsc{MG5\_aMC} & NLO & \textsc{Pythia} 8\ & NNPDF 3.0 NLO \cite{Ball:2014uwa} \\
     & (\textsc{MG5\_aMC}) & (NLO) & (\textsc{Herwig++}) & (CT10 \cite{ct10})  \\
$tHqb$ & \textsc{MG5\_aMC} & LO & \textsc{Pythia} 8 & CT10  \\
$tHW$ & \textsc{MG5\_aMC} & NLO & \textsc{Herwig++}  & CT10  \\
& (\textsc{Sherpa} 2.1.1) & (LO multileg) & (\textsc{Sherpa}) & (NNPDF 3.0 NLO)  \\
$t W Z$ & \textsc{MG5\_aMC} & NLO & \textsc{Pythia} 8 & NNPDF 2.3 LO   \\
$t\bar t t$, $t\bar t t\bar t$ & \textsc{MG5\_aMC} & LO & \textsc{Pythia} 8 & NNPDF 2.3 LO  \\
$t\bar t W^+ W^-$ & \textsc{MG5\_aMC} & LO & \textsc{Pythia} 8 & NNPDF 2.3 LO\\
$\ttbar$ & \textsc{Powheg-BOX v2} \cite{powhegtt} & NLO & \textsc{Pythia} 8 & NNPDF 3.0 NLO  \\
$\ttbar\gamma$ & \textsc{MG5\_aMC} & LO & \textsc{Pythia} 8 & NNPDF 2.3 LO \\
$s$-, $t$-channel, & \textsc{Powheg-BOX v1} \cite{powhegstp}& NLO & \textsc{Pythia} 6 & CT10 \\
 $Wt$ single top & & & &  \\
$qqVV$, $VVV$ & &   \\
$Z \to l^+l^-$ & \textsc{Sherpa} 2.2.1 & MEPS NLO  & \textsc{Sherpa} & NNPDF 3.0 NLO \\
\hline\hline
\end{tabular}
}
\end{center}
\end{table}

\begin{table}[hbt!]
    \centering
    \begin{tabular}{l|l}
        \hline\hline
        Sample & DSID \\
        \hline\hline
        $WZ$ & 364253, 364739-42 \\
        $VV$ & 364250, 364254, 364255, 363355-60, 364890 \\
        $t\bar{t}W$ & 410155 \\
        $t\bar{t}Z$ & 410156, 410157, 410218-20 \\
        low mass $t\bar{t}Z$ & 410276-8 \\
        Rare Top & 410397, 410398, 410399 \\
        single Top & 410658-9, 410644-5 \\
        three Top & 304014 \\
        four Top & 410080 \\
        $t\bar{t}WW$ & 410081 \\
        Z + jets & 364100-41 \\
        low mass Z + jets & 364198-215 \\
        W + jets & 364156-97 \\
        $V\gamma$ & 364500-35 \\
        $tZ$  & 410560 \\
        $tW$  & 410013-4 \\
        $WtZ$ & 410408 \\
        $VVV$ & 364242-9 \\
        $VH$ & 342284-5 \\
        $WtH$ & 341998 \\
        $t\bar{t}\gamma$ & 410389 \\
        $t\bar{t}$ & 410470 \\
        $t\bar{t}H$ & 345873-5, 346343-5 \\
        \hline\hline
    \end{tabular}
    \caption{List of Monte Carlo samples by data set ID used in the analysis.}
    \label{tbl:dsids}
\end{table}

%-------------------------------------------------------------------------------

%-------------------------------------------------------------------------------
\section{Object Reconstruction}
\label{sec:obj}

All regions defined in this analysis share a common lepton, jet, and overall event preselection. The selection applied to each physics object is detailed here; the event preselection, and the selection used to define the various fit regions, is described in Section \ref{sec:evt_selection}.

\subsection{Trigger}

Events are required to be selected by dilepton triggers, as summarized in Table \ref{tbl:trigger}. 

\begin{table}[H]
 \begin{center}
   \begin{tabular}{cc}
     \toprule
                  & Dilepton triggers (2015) \\
     \midrule
      $\mu\mu$ (asymm.)          & \verb!HLT_mu18_mu8noL1! \\
      $ee$ (symm.)               & \verb!HLT_2e12_lhloose_L12EM10VH! \\
      $e\mu,\mu e$ ($\sim$symm.) & \verb!HLT_e17_lhloose_mu14! \\
     \bottomrule
                       & Dilepton triggers (2016) \\
     \midrule
      $\mu\mu$ (asymm.)                   & \verb!HLT_mu22_mu8noL1! \\
      $ee$ (symm.)                        & \verb!HLT_2e17_lhvloose_nod0! \\
      $e\mu,\mu e$ ($\sim$symm.)          & \verb!HLT_e17_lhloose_nod0_mu14! \\
     \bottomrule

                  & Dilepton triggers (2017) \\
     \midrule
      $\mu\mu$ (asymm.)                   & \verb!HLT_mu22_mu8noL1! \\
      $ee$ (symm.)                        & \verb!HLT_2e24_lhvloose_nod0! \\
      $e\mu,\mu e$ ($\sim$symm.)          & \verb!HLT_e17_lhloose_nod0_mu14! \\
     \bottomrule
                  & Dilepton triggers (2018) \\
     \midrule
      $\mu\mu$ (asymm.)                   & \verb!HLT_mu22_mu8noL1! \\
      $ee$ (symm.)                        & \verb!HLT_2e24_lhvloose_nod0! \\
      $e\mu,\mu e$ ($\sim$symm.)          & \verb!HLT_e17_lhloose_nod0_mu14! \\ 
      \bottomrule
   \end{tabular}
   \caption{\label{tbl:trigger} List of lowest $p_{T}$-threshold, un-prescaled dilepton triggers used for 2015-2018 data taking.}
 \end{center}
\end{table}

\subsection{Light leptons}
\label{subsec:leps}

Electron candidates are reconstructed from energy clusters in the electromagnetic calorimeter that are associated with charged particle tracks reconstructed in the inner detector~\cite{ATLAS-CONF-2016-024}. Electron candidates are required to have $\pt > 10$ GeV and $|\eta_\textrm{cluster}| < 2.47$. Muon candidates are reconstructed by combining inner detector tracks with track segments or full tracks in the muon spectrometer \cite{PERF-2014-05}. Muon candidates are required to have $\pt > 10$~GeV and $|\eta| < 2.5$. Candidates in the transition region between different electromagnetic calorimeter components, $1.37 < |\eta_\textrm{cluster}| < 1.52$, are rejected. A multivariate likelihood discriminant combining shower shape and track information is used to distinguish real electrons from hadronic showers (fake electrons). To further reduce the non-prompt electron contribution, the track is required to be consistent with originating from the primary vertex; requirements are imposed on the transverse impact parameter significance ($|d_0|/\sigma_{d_0}<5$) and the longitudinal impact parameter ($|\Delta z_0 \sin \theta_\ell| < 0.5$ mm). Electron candidates are required to pass TightLH identification.
                   
Muon candidates are reconstructed by combining inner detector tracks with track segments or full tracks in the muon spectrometer \cite{PERF-2014-05}. Muon candidates are required to have $\pt > 10$~GeV and $|\eta| < 2.5$. The longitutinal impact parameter is the same for both electrons and muons, while muons are required to pass a slightly tighter transverse imapact parameter, $|d_0|/\sigma_{d_0}<3$. Muons are also required to pass Medium ID requirements. 

%All leptons are required to be isolated, as defined though the standard \verb|PLVLoose| working point supported by combined performance groups. 
Leptons are additionally required to pass a non-prompt BDT selection developed by the $t\bar{t}H$ multilepton/$t\bar{t}W$ analysis group. This BDT and the WPs used are summarized in Appendix \ref{sec:lepMVA}, and described in detail in \cite{ttW_140}. Optimized working points and scale factors for this BDT are taken from that analysis.

%\begin{table}
%\begin{center}
% \begin{tabular}{l|cccccc}
% \hline\hline
% & \multicolumn{3}{c|}{$e$} & \multicolumn{3}{c}{$\mu$} \\
% \hline
% & \multicolumn{1}{c|}{L}  & \multicolumn{1}{c|}{L*} & \multicolumn{1}{c|}{T} & \multicolumn{1}{c|}{L} & \multicolumn{1}{c|}%{L*} & \multicolumn{1}{c|}{T}  \\
%  \hline
%  FixedCutLoose           &  No & \multicolumn{2}{|c|}{Yes} & No & \multicolumn{2}{|c}{Yes} \\
%  \hline
%  Non-prompt lepton BDT   &  \multicolumn{2}{c|}{No} & \multicolumn{1}{c|}{Yes} & \multicolumn{2}{c|}{No} & \multicolumn{1}{%c}{Yes} \\
%  \hline
%  Identification  & \multicolumn{2}{c|}{Loose} & \multicolumn{1}{c|}{Tight} & \multicolumn{2}{c}{Loose} & \multicolumn{1}{c|%}{Medium}\\
%  \hline
%  %Charge mis-assignment veto &  \multicolumn{2}{c|}{No} & Yes & \multicolumn{3}{|c}{N/A} \\ - not needed, only for 2lSS
%  %\hline
%  %ambiguity bit == 0 &  \multicolumn{2}{c|}{No} & Yes & \multicolumn{3}{|c}{N/A} \\
%  %\hline
%  Transverse impact parameter significance  &  \multicolumn{3}{c|}{$<5$} & \multicolumn{3}{c}{$<3$ } \\
%  $|d_0|/\sigma_{d_0}$ & \multicolumn{3}{c|}{} &  \multicolumn{3}{c}{}  \\
%  \hline
%  Longitudinal impact parameter &  \multicolumn{6}{c}{$< 0.5$ mm} \\
%  $|z_0 \sin \theta|$ &  \multicolumn{6}{c}{} \\
%  \hline\hline
% \end{tabular}
%\caption{\label{tbl:tightleps} Loose (L), loose and minimally-isolated (L*), and tight (T) light lepton definitions.}
%\end{center}
%\end{table}


\subsection{Jets}
\label{subsec:jets}
%UPDATE FOR PFLOW
Jets are reconstructed from calibrated topological clusters built from energy deposits in the calorimeters \cite{ATL-PHYS-PUB-2015-015}, using the anti-$k_t$ algorithm with a radius parameter $R=0.4$. Particle Flow, or PFlow, jets are used in the analysis, which are hadronic objects reconstructed using information from both the tracker and the calorimeter. Jets with energy contributions likely arising from noise or detector effects are removed from consideration \cite{ATLAS-CONF-2015-029}, and only jets satisfying $\pt > 25$~GeV and $|\eta| < 2.5$ are used in this analysis.  For jets with $\pt < 60$~GeV and $|\eta| < 2.4$, a jet-track association algorithm is used to confirm that the jet originates from the selected primary vertex, in order to reject jets arising from pileup collisions \cite{PERF-2014-03}.

\subsection{B-tagged Jets}
\label{subsec:bjets}

In order to make a measurement of $WZ$ + heavy flavor it is necessary to distinguish these events from $WZ$ + light jets. For this purpose, the DL1r b-tagging algorithm is used to distinguish heavy flavor jets from lighter ones. The DL1r algorithm uses jet kinematics, particularly jet vertex information, as input for a neural network which assigns each jet a score designed to reflect how likely that jet is to have originated from a b-quark. 

\begin{figure}[H] 
    \centering
    \includegraphics[width=0.54\linewidth]{DL1_output.pdf} 
    \caption{Output distribution of the DL1r algorithm for b-jets, charm jets, and light jets}
    \label{fig:DL1r}
\end{figure}

From the output of the BDT, calibrated working points (WPs) are developed based on the efficiency of truth b-jets at particular values of the DL1r algorithm. The working points used in this analysis are summarized in Table \ref{tab:btag_WPs}. 

\begin{table}[H] 
\begin{center}
\begin{tabular}{|c|ccccc|}
    \hline
       WP &  none & loose & medium & tight & tightest\\
       \hline
     b eff. & - & 85\% & 77\% & 70\% & 60\% \\ 
    \hline
    \end{tabular}    
    \caption{B-tagging Working Points by tightness and b-jet efficiency}
    \label{tab:btag_WPs}
    \end{center}
\end{table}

A tighter WP will accept fewer b-jets, but reject a higher fraction of charm and light jets. Generally, analyses that include b-jets will use a fixed working point, for example, requiring that a jet pass the 70\% threshold. By instead treating these working point as bins, e.g. events with jets that fall between the 85\% and 77\% WPs fall into one bin, while events with jets passing the 60\% WP fall into another, and looking at the full psuedo-continuous DL1r spectrum of the jets, additional information can be gained. The psuedo-continuous b-tag spectrum is used in this case to separate out WZ + b, WZ + charm, and WZ + light. 

\subsection{Missing transverse energy}
\label{subsec:met}

Missing transverse momentum ($E_T^{miss}$) is used as part of the event selection. The missing transverse momentum vector is defined as the inverse of the sum of the transverse momenta of all reconstructed physics objects as well as remaining unclustered energy, the latter of which is estimated from low-\pt tracks associated with the primary vertex but not assigned to a hard object, with object definitions taken from \cite{ATL-PHYS-PUB-2015-027}. Light leptons considered in the $E_T^{miss}$ reconstruction are required to have $p_T$>10 GeV, while jets are required to have $p_T$>20 GeV.

\subsection{Overlap removal}
\label{subsec:overlapremoval}

To avoid double counting objects and remove leptons originating from decays of hadrons, overlap removal is performed in the following order: any electron candidate within $\Delta R = 0.1$ of another electron candidate with higher \pt\ is removed; any electron candidate within $\Delta R = 0.1$ of a muon candidate is removed; any jet within $\Delta R = 0.3$ of an electron candidate is removed; if a muon candidate and a jet lie within $\Delta R = min(0.4, $0.04+10$[GeV]/\pt(muon))$ of each other, the jet is kept and the muon is removed.

This algorithm is applied to the preselected objects. The overlap removal procedure is summarized in Table~\ref{tab:overlap-removal}. 

\begin{table}[h!]
 \begin{center}
   \begin{tabular}{|c|c|c|}
     \hline
                            \textbf{Keep}  &  \textbf{Remove} & \textbf{Cone size ($\Delta$ R)}  \\
         \hline
                        electron        & electron (low \pt)    & 0.1 \\
     \hline
                        muon    & electron      & 0.1 \\
     \hline
                            electron    & jet   & 0.3 \\
         \hline
                        jet             & muon  & min(0.4, $0.04+10$[GeV]/\pt(muon)) \\
         \hline
                        electron        & tau   & 0.2 \\
     \hline
   \end{tabular}
   \caption{\label{tab:overlap-removal} Summary of the overlap removal procedure between electrons, muons, and jets.}
 \end{center}
\end{table}

%-------------------------------------------------------------------------------

%-------------------------------------------------------------------------------
\section{Event Selection and Signal Region Definitions}
\label{sec:evt_selection}

Event are required to pass a preselection described in section \ref{subsec:presel} and summarized in table \ref{tbl:selection}. Those that pass this preselection are divided into various fit regions described in section \ref{subsec:regions}, based on the number of jets in the event, and the b-tag score of those jets.

%--------------------------- 
\subsection{Event Preselection}
\label{subsec:presel}
%--------------------------- 

Events are required to include exactly three reconstructed light leptons passing the requirement described in \ref{subsec:leps}, which have a total charge of $\pm$1. As the opposite sign lepton is found to be prompt the vast majority of the time \cite{ttH_paper}, it is required to be loose and isolated, as defined though the standard \verb|isolationFixedCutLoose| working point supported by combined performance groups. The same sign leptons are required to be very tight, as per the recommended \verb|isolationFixedCutTight|.

The leptons are ordered in the analysis code as 0, 1, and 2. Lepton 0 is the lepton whose charge is opposite the other two. Lepton 1 is the lepton closest to the opposite charge lepton, i.e. the smallest $\Delta R$, leaving lepton 2 as the lepton further from the opposite charge lepton. Lepton 0 is required to have $p_T > 10$ GeV, while the same sign leptons, 1 and 2, are required to have $p_T > 20$ GeV to reduce the contribution of non-prompt leptons.  

The invariant mass of at least one pair of opposite sign, same flavor leptons is required to fall within 10 GeV of the mass of the Z boson, 91.2 GeV. Events where one of the opposite sign pairs have an invariant mass less than 12 GeV are rejected in order to suppress low mass resonances. %Further, events where the trilepton mass falls within 5 GeV of the Z mass are rejected to remove Z events that include conversions.

An additional requirement is placed on the missing transverse energy, $E^{miss}_T$ > 20 GeV, and the transverse mass of the $W$ candidate, $m(E^{miss}_T + l_{other}) > 30$ GeV, where $E^{miss}_T$ is the missing transverse energy, and $l_{other}$ is the lepton not included in the Z-candidate. 

Events are required to have one or two reconstructed jets passing the selection described in section \ref{subsec:jets}. Events with more than two jets are rejected in order to reduce the contribution of backgrounds such as $t\bar{t}Z$ and $t\bar{t}W$, which tend to have higher jet multiplicity. 

\begin{table}[H]
    \centering
    \begin{tabular}{l}
        \hline\hline
        Event Selection\\
        \hline 
        Exactly three leptons with charge $\pm$1 \\
        Two same-charge leptons with $p_T$ $>$ 20 GeV \\
        One opposite charge lepton with $p_T$ $>$ 10 GeV \\
        $m(l^+l^-)$ within 10 GeV of 91.2 GeV \\
        Transverse mass of W-candidate, $m_T(E_T^{miss} + lep_{other})$ $>$ 30 GeV \\
        Missing transverse energy, $E_T^{miss} >$ 20 GeV \\
        One or two jets with $p_T$ $>$ 25 GeV \\
        \hline\hline
    \end{tabular}
    \caption{Summary of the selection applied to events for inclusion in the fit}
    \label{tbl:selection}
\end{table}

The event yields in the preselection region for both data and Monte Carlo are summarized in table \ref{tab:evt_yields}, which shows good agreement between data and Monte Carlo, and demonstrates that this region consists primarily of WZ events. The WZ events are split into WZ + b, WZ + c, and WZ + l based on the truth flavor of the heaviest associated jet in the event. Specifically, this determination is made based on the \verb!HadronConeExclTruthLabelID! of the jet. That is, WZ + l events contain no charm and b jets at truth level, WZ + c contain at least one truth charm and no b-jets, and WZ + b contains at least one truth b-jet. 

\begin{table}[H]
    \centering
        \begin{center}
\begin{tabular}{|c|c|}
\hline
Process & Events \\
\hline 
  $WZ + b$   & 143 $\pm$ 43 \\
  $WZ + c$   & 930 $\pm$ 280 \\
  $WZ + l$   & 6250 $\pm$ 1900 \\
  Other VV   & 16.6 $\pm$ 4.1 \\
  ZZ   & 460 $\pm$ 55 \\
  $t\bar{t}W$   & 14.8 $\pm$ 2.1 \\
  $t\bar{t}Z$   & 96 $\pm$ 14 \\
  Single top   & 0.1 $\pm$ 0.2 \\
  Three top   & 0.003 $\pm$ 0.002 \\
  Four top   & 0.01 $\pm$ 0.01 \\
  $t\bar{t}WW$   & 0.20 $\pm$ 0.04 \\
  $Z+\text{jets}$   & 320 $\pm$ 70 \\
  $V+\gamma$   & 75 $\pm$ 43 \\
  $tZ$   & 164 $\pm$ 37 \\
  $tW$   & 4.9 $\pm$ 1.2 \\
  $WtZ$   & 21 $\pm$ 11 \\
  $VVV$   & 23 $\pm$ 11 \\
  $VH$   & 67 $\pm$ 6 \\
  $t\bar{t}$   & 84 $\pm$ 9 \\
  $t\bar{t}H$   & 3.6 $\pm$ 0.4 \\
\hline
  Total  & 8700 $\pm$ 2300 \\
\hline
  Data   & 8696 \\

  %$WZ + b$   & 167.6 $\pm$ 6.5 \\
  %$WZ + c$   & 1080 $\pm$ 40 \\
  %$WZ + l$   & 7220 $\pm$ 310 \\
  %Other VV   & 850 $\pm$ 140 \\
  %$t\bar{t}W$   & 16.8 $\pm$ 2.3 \\
  %$t\bar{t}Z$   & 115 $\pm$ 17 \\
  %Single top   & 0.10 $\pm$ 0.45 \\
  %Three top   & 0.01 $\pm$ 0.01 \\
  %Four top   & 0.02 $\pm$ 0.01 \\
  %$t\bar{t}WW$   & 0.23 $\pm$ 0.05 \\
  %$Z+\text{jets}$   & 600 $\pm$ 260 \\
  %$V+\gamma$   & 37 $\pm$ 54 \\
  %$tZ$   & 190 $\pm$ 70 \\
  %$tW$   & 5.5 $\pm$ 1.2 \\
  %$WtZ$   & 25.8 $\pm$ 1.1 \\
  %$VVV$   & 26.2 $\pm$ 0.9 \\
  %$VH$   & 94 $\pm$ 7 \\
  %$t\bar{t}$   & 110 $\pm$ 8 \\
  %$t\bar{t}H$   & 4.3 $\pm$ 0.5 \\
%\hline
%  Total  & 10600 $\pm$ 530 \\
%\hline
%  Data   & 10574 \\
\hline 
\end{tabular} 
\caption{Event yields in the preselection region at 139.0 $fb^{-1}$. Includes the full set of systematic uncertainties} 
\end{center} 


    %\caption{Data and MC yields after the event selection requiring three leptons, one or two jets, $E^{miss}_T$ > 20 GeV, and $m(E^{miss}_T + l_{other}) > 30$ GeV selection has been applied.}
    \label{tab:evt_yields}
\end{table}

Here Other $VV$ represents diboson processes other than WZ, and consists predominantly of $ZZ\rightarrow llll$ events where one of the leptons is not reconstructed.

Simulations are further validated by comparing the kinematic distributions of the Monte Carlo with data, which are shown in figures \ref{kin:inclusive}. Here, bins with 5\% or more WZ+b are blinded.

%textbf{There is some discrepancies between data and MC, particularly in the low MET and low lepton $p_T$ regions, which are being investigated. This is suspected to be the result of underestimating the fake contribution, possibly because of several missing Z+jets simulation files.}

\begin{figure}[H]
    \centering
    \textbf{WZ Fit Region - Inclusive}\\
    \subfigure[]{\includegraphics[width=.29\linewidth]{regions/plots_inclusive/Plots/lead_jetPt.png}}%
    \subfigure[]{\includegraphics[width=.29\linewidth]{regions/plots_inclusive/Plots/lep_Pt_0.png}}%
    \subfigure[]{\includegraphics[width=.29\linewidth]{regions/plots_inclusive/Plots/lep_Pt_1.png}}\\      
    \subfigure[]{\includegraphics[width=.29\linewidth]{regions/plots_inclusive/Plots/lep_Pt_2.png}}%
    \subfigure[]{\includegraphics[width=.29\linewidth]{regions/plots_inclusive/Plots/MET.png}}%
    \subfigure[]{\includegraphics[width=.29\linewidth]{regions/plots_inclusive/Plots/Mll01.png}}\\
    \subfigure[]{\includegraphics[width=.29\linewidth]{regions/plots_inclusive/Plots/Mll02.png}}%
    \subfigure[]{\includegraphics[width=.29\linewidth]{regions/plots_inclusive/Plots/Mll12.png}}%
    \caption{Comparisons between data and MC distributions in the preselection region for the $p_T$ of (a) the leading jet, (b) lepton 0, (c) lepton 1, (d) lepton 2, (e) the missing transverse energy, and (f) the invariant mass of leptons 0 and 1, (g) the invariant mass of leptons 0 and 2, and (h) the invariant mass of leptons 1 and 2.}
    \label{kin:inclusive}
\end{figure}

%---------------------------                                                                                         
\subsection{Fit Regions}
\label{subsec:regions}
%--------------------------- 

Once preselection has been applied, the remaining events are categorized into one of twelve orthogonal regions. The regions used in the fit are summarized in table \ref{tab:regions}.

\begin{table}[h]
\centering
\caption{A list of the regions used in the fit and the selection used for each.}
\begin{tabular}{l|l}
\hline\hline
Region & Selection            \\
\hline
\hline
%1j, <85\%       & $N_{jets}$ = 1, jet DL1r score < 85\% WP            \\
%1j, 85\%-77\%   & $N_{jets}$ = 1, 85\% < jet DL1r score < 77\% WP                    \\
%1j, 77\%-70\%   & $N_{jets}$ = 1, 77\% < jet DL1r score < 70\% WP                    \\
%1j, 70\%-60\%   & $N_{jets}$ = 1, 70\% < jet DL1r score < 60\% WP                    \\
%1j, >60\%       & $N_{jets}$ = 1, jet DL1r score > 85\% WP, tZ BDT score > 0.03 \\
%1j tZ CR        & $N_{jets}$ = 1, jet DL1r > 85\% WP, tZ BDT score < 0.03 \\
%2j, <85\%       & $N_{jets}$ = 2, jet DL1r score < 85\% WP                    \\
%2j, 85\%-77\%   & $N_{jets}$ = 2, 85\% WP < jet DL1r score < 77\% WP                 \\
%2j, 77\%-70\%   & $N_{jets}$ = 2, 77\% WP < jet DL1r score < 70\% WP                 \\
%2j, 70\%-60\%   & $N_{jets}$ = 2, 70\% < jet DL1r score < 60\% WP                     \\
%2j, >60\%       & $N_{jets}$ = 2, jet DL1r score > 85\% WP, tZ BDT score > 0.03 \\
%2j tZ CR        & $N_{jets}$ = 2, jet DL1r score > 85\% WP, tZ BDT score < 0.03 \\
1j, <85\%       & $N_{jets}$ = 1, nJets\_DL1r\_85 = 0            \\
1j, 85\%-77\%   & $N_{jets}$ = 1, nJets\_DL1r\_85 = 1, nJets\_DL1r\_77=0                     \\
1j, 77\%-70\%   & $N_{jets}$ = 1, nJets\_DL1r\_77 = 1, nJets\_DL1r\_70=0                     \\
1j, 70\%-60\%   & $N_{jets}$ = 1, nJets\_DL1r\_70 = 1, nJets\_DL1r\_60=0                      \\
1j, >60\%       & $N_{jets}$ = 1, nJets\_DL1r\_60 = 1, tZ BDT > 0.725 \\
1j tZ CR        & $N_{jets}$ = 1, nJets\_DL1r\_60 = 1, tZ BDT < 0.725 \\
2j, <85\%       & $N_{jets}$ = 2, nJets\_DL1r\_85 = 0                    \\
2j, 85\%-77\%   & $N_{jets}$ = 2, nJets\_DL1r\_85 >= 1, nJets\_DL1r\_77=0                     \\
2j, 77\%-70\%   & $N_{jets}$ = 2, nJets\_DL1r\_77 >= 1, nJets\_DL1r\_70=0                     \\
2j, 70\%-60\%   & $N_{jets}$ = 2, nJets\_DL1r\_70 >= 1, nJets\_DL1r\_60=0                      \\
2j, >60\%       & $N_{jets}$ = 2, nJets\_DL1r\_60 >= 1, tZ BDT > 0.725 \\
2j tZ CR        & $N_{jets}$ = 2, nJets\_DL1r\_60 >= 1, tZ BDT < 0.725 \\
\hline\hline
\end{tabular}
\label{tab:regions}
\end{table}

The working points discussed in section \ref{subsec:bjets} are used to separate events into fit regions based on the highest working point reached by a jet in each event. Because the background composition differs significantly based on the number of b-jets, events are further subdivided into 1 jet and 2 jet regions in order to minimize the impact of background uncertainties.

An additional tZ control region is created based on the BDT described in section \ref{sec:tZ_bdt}. The region with 1-jet passing the 60\% working point is split in two - a signal enriched region of events with a BDT score greater than 0.03, and a tZ control region including events with less than 0.03. This cutoff is arrived at by performing an Asimov fit with a variety of cutoffs, and selecting the value that produces the highest significance for the measurement of $WZ$ + b.

The modeling in each region is validated by comparing data and MC predictions for various kinematic distributions. These plot are shown in figures \ref{kin:WP_1j_not85}-\ref{kin:tZ_CR_2j}.

\begin{figure}[H]
    \centering
    \textbf{WZ Fit Region - 1j $<$ 85\% WP}\\
    \subfigure[]{\includegraphics[width=.29\linewidth]{regions/plots_not_85/Plots/lead_jetPt.png}}%
    \subfigure[]{\includegraphics[width=.29\linewidth]{regions/plots_not_85/Plots/lep_Pt_0.png}}%
    \subfigure[]{\includegraphics[width=.29\linewidth]{regions/plots_not_85/Plots/lep_Pt_1.png}}\\
    \subfigure[]{\includegraphics[width=.29\linewidth]{regions/plots_not_85/Plots/lep_Pt_2.png}}%
    \subfigure[]{\includegraphics[width=.29\linewidth]{regions/plots_not_85/Plots/MET.png}}%
    \subfigure[]{\includegraphics[width=.29\linewidth]{regions/plots_not_85/Plots/Mll01.png}}\\
    \caption{Comparisons between the data and MC distributions in the preselection region for the $p_T$ of (a) the leading jet, (b) lepton 0, (c) lepton 1, (d) lepton 2, (e) the missing transverse energy, and (f) the invariant mass of lepton 0 and 1.}
    \label{kin:WP_1j_not85}
\end{figure}

\begin{figure}[H]
    \centering
    \textbf{WZ Fit Region - 1j 77-85\% WP}\\
    \subfigure[]{\includegraphics[width=.29\linewidth]{regions/plots_1j_77_85/Plots/lead_jetPt.png}}%
    \subfigure[]{\includegraphics[width=.29\linewidth]{regions/plots_1j_77_85/Plots/lep_Pt_0.png}}%
    \subfigure[]{\includegraphics[width=.29\linewidth]{regions/plots_1j_77_85/Plots/lep_Pt_1.png}}\\
    \subfigure[]{\includegraphics[width=.29\linewidth]{regions/plots_1j_77_85/Plots/lep_Pt_2.png}}%
    \subfigure[]{\includegraphics[width=.29\linewidth]{regions/plots_1j_77_85/Plots/MET.png}}%
    \subfigure[]{\includegraphics[width=.29\linewidth]{regions/plots_1j_77_85/Plots/Mll01.png}}\\
    \caption{Comparisons between the data and MC distributions in the preselection region for the $p_T$ of (a) the leading jet, (b) lepton 0, (c) lepton 1, (d) lepton 2, (e) the missing transverse energy, and (f) the invariant mass of lepton 0 and 1.}
    \label{kin:WP_1j_77_85}
\end{figure}

\begin{figure}[H]
    \centering
    \textbf{WZ Fit Region - 1j 70-77\% WP}\\
    \subfigure[]{\includegraphics[width=.29\linewidth]{regions/plots_1j_70_77/Plots/lead_jetPt.png}}%
    \subfigure[]{\includegraphics[width=.29\linewidth]{regions/plots_1j_70_77/Plots/lep_Pt_0.png}}%
    \subfigure[]{\includegraphics[width=.29\linewidth]{regions/plots_1j_70_77/Plots/lep_Pt_1.png}}\\
    \subfigure[]{\includegraphics[width=.29\linewidth]{regions/plots_1j_70_77/Plots/lep_Pt_2.png}}%
    \subfigure[]{\includegraphics[width=.29\linewidth]{regions/plots_1j_70_77/Plots/MET.png}}%
    \subfigure[]{\includegraphics[width=.29\linewidth]{regions/plots_1j_70_77/Plots/Mll01.png}}\\
    \caption{Comparisons between the data and MC distributions in the preselection region for the $p_T$ of (a) the leading jet, (b) lepton 0, (c) lepton 1, (d) lepton 2, (e) the missing transverse energy, and (f) the invariant mass of lepton 0 and 1.}
    \label{kin:WP_1j_70_77}   
\end{figure}

\begin{figure}[H]
    \centering
    \textbf{WZ Fit Region - 1j 60-70\% WP}\\
    \subfigure[]{\includegraphics[width=.29\linewidth]{regions/plots_1j_60_70/Plots/lead_jetPt.png}}%
    \subfigure[]{\includegraphics[width=.29\linewidth]{regions/plots_1j_60_70/Plots/lep_Pt_0.png}}%
    \subfigure[]{\includegraphics[width=.29\linewidth]{regions/plots_1j_60_70/Plots/lep_Pt_1.png}}\\
    \subfigure[]{\includegraphics[width=.29\linewidth]{regions/plots_1j_60_70/Plots/lep_Pt_2.png}}%
    \subfigure[]{\includegraphics[width=.29\linewidth]{regions/plots_1j_60_70/Plots/MET.png}}%
    \subfigure[]{\includegraphics[width=.29\linewidth]{regions/plots_1j_60_70/Plots/Mll01.png}}\\
    \caption{Comparisons between the data and MC distributions in the preselection region for the $p_T$ of (a) the leading jet, (b) lepton 0, (c) lepton 1, (d) lepton 2, (e) the missing transverse energy, and (f) the invariant mass of lepton 0 and 1.}
    \label{kin:WP_1j_60_70}
\end{figure}

\begin{figure}[H]
    \centering
    \textbf{WZ Fit Region - 1j 60\% WP}\\
    \subfigure[]{\includegraphics[width=.29\linewidth]{regions/plots_1j_60/Plots/lead_jetPt.png}}%
    \subfigure[]{\includegraphics[width=.29\linewidth]{regions/plots_1j_60/Plots/lep_Pt_0.png}}%
    \subfigure[]{\includegraphics[width=.29\linewidth]{regions/plots_1j_60/Plots/lep_Pt_1.png}}\\
    \subfigure[]{\includegraphics[width=.29\linewidth]{regions/plots_1j_60/Plots/lep_Pt_2.png}}%
    \subfigure[]{\includegraphics[width=.29\linewidth]{regions/plots_1j_60/Plots/MET.png}}%
    \subfigure[]{\includegraphics[width=.29\linewidth]{regions/plots_1j_60/Plots/Mll01.png}}\\
    \caption{Comparisons between the data and MC distributions in the preselection region for the $p_T$ of (a) the leading jet, (b) lepton 0, (c) lepton 1, (d) lepton 2, (e) the missing transverse energy, and (f) the invariant mass of lepton 0 and 1.}
    \label{kin:WP_1j_60}    
\end{figure}

\begin{figure}[H]
    \centering
    \textbf{WZ Fit Region - tZ-CR}\\
    \subfigure[]{\includegraphics[width=.29\linewidth]{regions/plots_tZ_CR/Plots/lead_jetPt.png}}%
    \subfigure[]{\includegraphics[width=.29\linewidth]{regions/plots_tZ_CR/Plots/lep_Pt_0.png}}%
    \subfigure[]{\includegraphics[width=.29\linewidth]{regions/plots_tZ_CR/Plots/lep_Pt_1.png}}\\
    \subfigure[]{\includegraphics[width=.29\linewidth]{regions/plots_tZ_CR/Plots/lep_Pt_2.png}}%
    \subfigure[]{\includegraphics[width=.29\linewidth]{regions/plots_tZ_CR/Plots/MET.png}}%
    \subfigure[]{\includegraphics[width=.29\linewidth]{regions/plots_tZ_CR/Plots/Mll01.png}}\\
    \caption{Comparisons between the data and MC distributions in the preselection region for the $p_T$ of (a) the leading jet, (b) lepton 0, (c) lepton 1, (d) lepton 2, (e) the missing transverse energy, and (f) the invariant mass of lepton 0 and 1.}
    \label{kin:tZ_CR_1j}
\end{figure}

\begin{figure}[H]
    \centering
    \textbf{WZ Fit Region - 2j $<$ 85\% WP}\\
    \subfigure[]{\includegraphics[width=.29\linewidth]{regions/plots_not_85_2j/Plots/lead_jetPt.png}}%
    \subfigure[]{\includegraphics[width=.29\linewidth]{regions/plots_not_85_2j/Plots/lep_Pt_0.png}}%
    \subfigure[]{\includegraphics[width=.29\linewidth]{regions/plots_not_85_2j/Plots/lep_Pt_1.png}}\\
    \subfigure[]{\includegraphics[width=.29\linewidth]{regions/plots_not_85_2j/Plots/lep_Pt_2.png}}%
    \subfigure[]{\includegraphics[width=.29\linewidth]{regions/plots_not_85_2j/Plots/MET.png}}%
    \subfigure[]{\includegraphics[width=.29\linewidth]{regions/plots_not_85_2j/Plots/Mll01.png}}\\
    \caption{Comparisons between the data and MC distributions in the preselection region for the $p_T$ of (a) the leading jet, (b) lepton 0, (c) lepton 1, (d) lepton 2, (e) the missing transverse energy, and (f) the invariant mass of lepton 0 and 1.}
    \label{kin:WP_2j_not85}
\end{figure}

\begin{figure}[H]
    \centering
    \textbf{WZ Fit Region - 2j 77-85\% WP}\\
    \subfigure[]{\includegraphics[width=.29\linewidth]{regions/plots_2j_77_85/Plots/lead_jetPt.png}}%
    \subfigure[]{\includegraphics[width=.29\linewidth]{regions/plots_2j_77_85/Plots/lep_Pt_0.png}}%
    \subfigure[]{\includegraphics[width=.29\linewidth]{regions/plots_2j_77_85/Plots/lep_Pt_1.png}}\\
    \subfigure[]{\includegraphics[width=.29\linewidth]{regions/plots_2j_77_85/Plots/lep_Pt_2.png}}%
    \subfigure[]{\includegraphics[width=.29\linewidth]{regions/plots_2j_77_85/Plots/MET.png}}%
    \subfigure[]{\includegraphics[width=.29\linewidth]{regions/plots_2j_77_85/Plots/Mll01.png}}\\
    \caption{Comparisons between the data and MC distributions in the preselection region for the $p_T$ of (a) the leading jet, (b) lepton 0, (c) lepton 1, (d) lepton 2, (e) the missing transverse energy, and (f) the invariant mass of lepton 0 and 1.}
    \label{kin:WP_2j_77_85}
\end{figure}

\begin{figure}[H]
    \centering
    \textbf{WZ Fit Region - 2j 70-77\% WP}\\
    \subfigure[]{\includegraphics[width=.29\linewidth]{regions/plots_2j_70_77/Plots/lead_jetPt.png}}%
    \subfigure[]{\includegraphics[width=.29\linewidth]{regions/plots_2j_70_77/Plots/lep_Pt_0.png}}%
    \subfigure[]{\includegraphics[width=.29\linewidth]{regions/plots_2j_70_77/Plots/lep_Pt_1.png}}\\
    \subfigure[]{\includegraphics[width=.29\linewidth]{regions/plots_2j_70_77/Plots/lep_Pt_2.png}}%
    \subfigure[]{\includegraphics[width=.29\linewidth]{regions/plots_2j_70_77/Plots/MET.png}}%
    \subfigure[]{\includegraphics[width=.29\linewidth]{regions/plots_2j_70_77/Plots/Mll01.png}}\\
    \caption{Comparisons between the data and MC distributions in the preselection region for the $p_T$ of (a) the leading jet, (b) lepton 0, (c) lepton 1, (d) lepton 2, (e) the missing transverse energy, and (f) the invariant mass of lepton 0 and 1.}
    \label{kin:WP_2j_70_77}
\end{figure}

\begin{figure}[H]
    \centering
    \textbf{WZ Fit Region - 2j 60-70\% WP}\\
    \subfigure[]{\includegraphics[width=.29\linewidth]{regions/plots_2j_60_70/Plots/lead_jetPt.png}}%
    \subfigure[]{\includegraphics[width=.29\linewidth]{regions/plots_2j_60_70/Plots/lep_Pt_0.png}}%
    \subfigure[]{\includegraphics[width=.29\linewidth]{regions/plots_2j_60_70/Plots/lep_Pt_1.png}}\\
    \subfigure[]{\includegraphics[width=.29\linewidth]{regions/plots_2j_60_70/Plots/lep_Pt_2.png}}%
    \subfigure[]{\includegraphics[width=.29\linewidth]{regions/plots_2j_60_70/Plots/MET.png}}%
    \subfigure[]{\includegraphics[width=.29\linewidth]{regions/plots_2j_60_70/Plots/Mll01.png}}\\
    \caption{Comparisons between the data and MC distributions in the preselection region for the $p_T$ of (a) the leading jet, (b) lepton 0, (c) lepton 1, (d) lepton 2, (e) the missing transverse energy, and (f) the invariant mass of lepton 0 and 1.}
    \label{kin:WP_2j_60_70}
\end{figure}

\begin{figure}[H]
    \centering
    \textbf{WZ Fit Region - 2j 60\% WP}\\
    \subfigure[]{\includegraphics[width=.29\linewidth]{regions/plots_2j_60/Plots/lead_jetPt.png}}%
    \subfigure[]{\includegraphics[width=.29\linewidth]{regions/plots_2j_60/Plots/lep_Pt_0.png}}%
    \subfigure[]{\includegraphics[width=.29\linewidth]{regions/plots_2j_60/Plots/lep_Pt_1.png}}\\
    \subfigure[]{\includegraphics[width=.29\linewidth]{regions/plots_2j_60/Plots/lep_Pt_2.png}}%
    \subfigure[]{\includegraphics[width=.29\linewidth]{regions/plots_2j_60/Plots/MET.png}}%
    \subfigure[]{\includegraphics[width=.29\linewidth]{regions/plots_2j_60/Plots/Mll01.png}}\\
    \caption{Comparisons between the data and MC distributions in the preselection region for the $p_T$ of (a) the leading jet, (b) lepton 0, (c) lepton 1, (d) lepton 2, (e) the missing transverse energy, and (f) the invariant mass of lepton 0 and 1.}
    \label{kin:WP_2j_60}
\end{figure}

\begin{figure}[H]
    \textbf{WZ Fit Region - tZ-CR-2j}\\
    \subfigure[]{\includegraphics[width=.29\linewidth]{regions/plots_tZ_CR_2j/Plots/lead_jetPt.png}}%
    \subfigure[]{\includegraphics[width=.29\linewidth]{regions/plots_tZ_CR_2j/Plots/lep_Pt_0.png}}%
    \subfigure[]{\includegraphics[width=.29\linewidth]{regions/plots_tZ_CR_2j/Plots/lep_Pt_1.png}}\\
    \subfigure[]{\includegraphics[width=.29\linewidth]{regions/plots_tZ_CR_2j/Plots/lep_Pt_2.png}}%
    \subfigure[]{\includegraphics[width=.29\linewidth]{regions/plots_tZ_CR_2j/Plots/MET.png}}%
    \subfigure[]{\includegraphics[width=.29\linewidth]{regions/plots_tZ_CR_2j/Plots/Mll01.png}}\\
    \caption{Comparisons between the data and MC distributions in the preselection region for the $p_T$ of (a) the leading jet, (b) lepton 0, (c) lepton 1, (d) lepton 2, (e) the missing transverse energy, and (f) the invariant mass of lepton 0 and 1.}
    \label{kin:tZ_CR_2j}
\end{figure}

%---------------------------
\subsection{Non-Prompt Lepton Estimation}
\label{sec:fakes}
%---------------------------

Two processes act as sources of non-prompt leptons appear in the analysis: $t\bar{t}$ and $Z$+jet production both produce two prompt leptons, and each contribute to the 3l region when an additional non-prompt lepton appears in the event. The contribution of these processes is estimated with Monte Carlo simulations, which are validated using enriched validation regions.

\subsubsection{$t\bar{t}$ Validation}

$t\bar{t}$ events can produce two prompt leptons from the decay of each of the tops. These top decays produce two b-quarks, the decay of which can produce additional non-prompt leptons, which occasionally pass the event preselection. In order to validate that the Monte Carlo accurately simulates this process accurately, the MC prediction in a non-prompt $t\bar{t}$ enriched validation region is compared to data.

The $t\bar{t}$ validation region is similar to the preselection region - three leptons meeting the criteria described in section \ref{sec:evt_selection} are required, and the requirements on $E_T^{miss}$ remain the same. However, the selection requiring a lepton pair form a Z-candidate are reversed. Events where the invariant mass of any two opposite sign, same flavor leptons falls within 10 GeV of 91.2 GeV are rejected. This ensures the $t\bar{t}$ validation region is orthogonal to the preselection region. 

Further, because the jet multiplicity of $t\bar{t}$ events tends to be higher than WZ, the number of jets in each event is required to be greater than 1. As b-jets are almost invariably produced from top decays, at least one b-tagged jet passing the 70\% DL1r WP in each event is required. Various kinematic plots of this region are shown in figure \ref{fig:ttbar_noScale}.

\begin{figure}[H]
    \centering
    \subfigure[]{\includegraphics[width=0.29\textwidth]{ttbar/noScale/lead_jetPt.png}}%                             
    \subfigure[]{\includegraphics[width=0.29\textwidth]{ttbar/noScale/MET.png}}%
    \subfigure[]{\includegraphics[width=0.29\textwidth]{ttbar/noScale/lep_Pt_0.png}}\\
    \subfigure[]{\includegraphics[width=0.29\textwidth]{ttbar/noScale/lep_Pt_1.png}}%
    \subfigure[]{\includegraphics[width=0.29\textwidth]{ttbar/noScale/lep_Pt_2.png}}%                                                 
    \subfigure[]{\includegraphics[width=0.29\textwidth]{ttbar/noScale/Mll01.png}}\\
    \subfigure[]{\includegraphics[width=0.29\textwidth]{ttbar/noScale/Mll02.png}}%
    \subfigure[]{\includegraphics[width=0.29\textwidth]{ttbar/noScale/nJets_OR.png}}%                                       
    \subfigure[]{\includegraphics[width=0.29\textwidth]{ttbar/noScale/nJets_OR_DL1r_70.png}}\\
    \caption{Comparisons between the data and MC distributions in the $t\bar{t}$ validation region for (a) the $p_T$ of the leading jet, (b) the missing transverse energy, (c) the $p_T$ of lepton 0, (d) $p_T$ of lepton 1, (e) $p_T$ of lepton 2, (f) the invariant mass of leptons 0 and 1, (g) the invariant mass of leptons 0 and 2, (h) the number of jets, (i) the number of b-tagged jets.}
    \label{fig:ttbar_noScale}
\end{figure}

The shape of each distribution agrees quite well between data and MC, with a constant offset between the two. This is accounted for by applying a constant correction factor of 0.883 to the $t\bar{t}$ MC prediction. Plots showing the kinematics of the $t\bar{t}$ VR after this correction factor has been applied are shown in figure \ref{fig:ttbar_withScale}.

\begin{figure}[H]
    \centering
    \subfigure[]{\includegraphics[width=0.29\textwidth]{ttbar/noScale/lead_jetPt.png}}%                                      
    \subfigure[]{\includegraphics[width=0.29\textwidth]{ttbar/noScale/MET.png}}%                                             
    \subfigure[]{\includegraphics[width=0.29\textwidth]{ttbar/noScale/lep_Pt_0.png}}\\                                       
    \subfigure[]{\includegraphics[width=0.29\textwidth]{ttbar/noScale/lep_Pt_1.png}}%                                        
    \subfigure[]{\includegraphics[width=0.29\textwidth]{ttbar/noScale/lep_Pt_2.png}}%                            
    \subfigure[]{\includegraphics[width=0.29\textwidth]{ttbar/noScale/Mll01.png}}\\                                          
    \subfigure[]{\includegraphics[width=0.29\textwidth]{ttbar/noScale/Mll02.png}}%                                          
    \subfigure[]{\includegraphics[width=0.29\textwidth]{ttbar/noScale/nJets_OR.png}}%                                       
    \subfigure[]{\includegraphics[width=0.29\textwidth]{ttbar/noScale/nJets_OR_DL1r_70.png}}\\                             
    \caption{Comparisons between the data and MC distributions in the $t\bar{t}$ validation region after the correction factor has been applied for (a) the $p_T$ of the leading jet, (b) the missing transverse energy, (c) the $p_T$ of lepton 0, (d) $p_T$ of lepton 1, (e) $p_T$ of lepton 2, (f) the invariant mass of leptons 0 and 1, (g) the invariant mass of leptons 0 and 2, (h) the number of jets, (i) the number of b-tagged jets.}                                                                                                                     \label{fig:ttbar_noScale}
\end{figure}

The modeling is further validated by looking at the yield in the $t\bar{t}$ VR for each DL1r WP, giving a clearer correspondence to the signal regions used in the fit. Each region shown in figure \ref{fig:ttbar_summary} requires one or more jets pass the listed WP, with no jets passing the next highest WP.

\begin{figure}[H]
   \centering
   \includegraphics[width=0.9\textwidth]{ttbar/Summary.png}   
   \caption{Data and MC comparisons for each DL1r WP for both 1-jet and 2-jet regions, after the $t\bar{t}$ VR selection and correction factor have been applied}
   \label{fig:ttbar_summary}
\end{figure}

As data and MC are found to agree within 10\% for each of these working points, a 10\% systematic uncertainty on the $t\bar{t}$ prediction is included for the analysis.

\subsubsection{$Z$+jets Validation}

Similar to $t\bar{t}$, a non-prompt $Z$+jets validation region is produced in order to validate the MC predictions. The lepton requirements remain the same as the preselection region. Because no neutrinos are present for this process, the $E_T^{miss}$ cut is reversed, requiring $E_T^{miss}$ < 30 GeV. This also ensures this validation region is orthogonal to the preselection region. Further, the number of jets in each event is required to be greater than or equal to one. Various kinematic plots of this region are shown below. The general agreement between data and MC in each of these suggests that the non-prompt contribution of $Z$+jets is well modeled by Monte Carlo.

\begin{figure}[H]
    \subfigure[]{\includegraphics[width=0.29\textwidth]{zjets/noScale/lead_jetPt.png}}%                          
    \subfigure[]{\includegraphics[width=0.29\textwidth]{zjets/noScale/DRll01.png}}%
    \subfigure[]{\includegraphics[width=0.29\textwidth]{zjets/noScale/lep_Pt_0.png}}\\
    \subfigure[]{\includegraphics[width=0.29\textwidth]{zjets/noScale/lep_Pt_1.png}}%
    \subfigure[]{\includegraphics[width=0.29\textwidth]{zjets/noScale/lep_Pt_2.png}}%                                      
    \subfigure[]{\includegraphics[width=0.29\textwidth]{zjets/noScale/Mll01.png}}\\
    \subfigure[]{\includegraphics[width=0.29\textwidth]{zjets/noScale/Mll02.png}}%
    \subfigure[]{\includegraphics[width=0.29\textwidth]{zjets/noScale/nJets_OR.png}}%                            
    \subfigure[]{\includegraphics[width=0.29\textwidth]{zjets/noScale/DRll02.png}}\\
    \caption{Comparisons between the data and MC distributions in the $Z$+jets validation region for (a) the $p_T$ of the leading jet, (b) $\Delta R$ between leptons 0 and 1, (c) the $p_T$ of lepton 0, (d) $p_T$ of lepton 1, (e) $p_T$ of lepton 2, (f) the invariant mass of leptons 0 and 1, (g) the invariant mass of leptons 0 and 2, (h) the number of jets, (i) $\Delta R$ between leptons 0 and 2. Includes only statistical uncertainties}%(i) the number of b-tagged jets.}
    \label{fig:zjets_noScale}
\end{figure}

While there is general agreement between data and MC within statistical uncertainty, the shape of the $p_T$ spectrum of lepton 2 is found to differ. To account for this discrepency, a variable correction factor is applied to Z+jets. $\chi^2$ minimization of the lepton 2 $p_T$ spectrum is performed to derive a correction factor of $1.53 - 6.6*10^{-6} (lep\_Pt\_2)$. Kinematic plots of the Z + jets validation region after this correction factor has been apllied are shown in figure \ref{fig:zjets_withScale}.

\begin{figure}[H]
    \subfigure[]{\includegraphics[width=0.29\textwidth]{zjets/withScale/lead_jetPt.png}}%              
    \subfigure[]{\includegraphics[width=0.29\textwidth]{zjets/withScale/DRll01.png}}%                            
    \subfigure[]{\includegraphics[width=0.29\textwidth]{zjets/withScale/lep_Pt_0.png}}\\
    \subfigure[]{\includegraphics[width=0.29\textwidth]{zjets/withScale/lep_Pt_1.png}}%                           
    \subfigure[]{\includegraphics[width=0.29\textwidth]{zjets/withScale/lep_Pt_2.png}}%                             
    \subfigure[]{\includegraphics[width=0.29\textwidth]{zjets/withScale/Mll01.png}}\\                                 
    \subfigure[]{\includegraphics[width=0.29\textwidth]{zjets/withScale/Mll02.png}}%                                   
    \subfigure[]{\includegraphics[width=0.29\textwidth]{zjets/withScale/nJets_OR.png}}%                               
    \subfigure[]{\includegraphics[width=0.29\textwidth]{zjets/withScale/DRll02.png}}\\
    \caption{Comparisons between the data and MC distributions in the $Z$+jets validation region after the correction factor has been applied for (a) the $p_T$ of the leading jet, (b) $\Delta R$ between leptons 0 and 1, (c) the $p_T$ of lepton 0, (d) $p_T$ of lepton 1, (e) $p_T$ of lepton 2, (f) the invariant mass of leptons 0 and 1, (g) the invariant mass of leptons 0 and 2, (h) the number of jets, (i) $\Delta R$ between leptons 0 and 2}%(i) the number of b-tagged jets.}                 
    \label{fig:zjets_noScale}
\end{figure}

The modeling is further validated by looking at the yield in the Z+jets VR for each DL1r WP, giving a clearer correspondence to the signal regions used in the fit. Each region shown in figure \ref{fig:ttbar_summary} requires one or more jets pass the listed WP, with no jets passing the next highest WP.                                                                    

                                                                                                                             
\begin{figure}[H]                                                                                                            
   \centering
   \includegraphics[width=0.9\textwidth]{zjets/Summary.png}
   \caption{Data and MC comparisons for each DL1r WP for both 1-jet and 2-jet regions, after the Z+jets VR selection and correction factor have been applied}                                                                                             
   \label{fig:ttbar_summary}
\end{figure}

For each of the working points considered, the data falls within 20\% of the MC prediction once this correction factor has been applied. Therefore, a 20\% systematic uncertainty is applied to Z + jets in the analysis.

%-------------------------------------------------------------------------------

%------------------------------------------------------------------------------- 
\section{tZ Interference Studies and Separation Multivariate Analysis}
\label{sec:tZ_bdt}

%Because it includes an on-shell Z boson as well as a b-jet and W from the top decay, tZ production represents an identical final state  to WZ + b-jet. This implies the possibility of matrix level interference between these two processes not accounted for in the Monte Carlo simulations, which consider the two processes independently. Truth level studies are performed in order to estimate the impact of these interference effects.

An important process to consider in this analysis is tZ: the top almost always decays into a W boson and b-quark, and when both the W and Z decay leptonically, this gives three leptons and a heavy flavor jet in the final state. Because tZ can produce a final state identical to the signal, it represents a predominant background in the most signal enriched regions. That is, the region with one jet passing the 60\% DL1r WP. Therefore, a boosted decision tree (BDT) algorithm is trained using XGBoost \cite{xgboost_cite} to separate $WZ$ + heavy flavor from tZ using kinematic quantities. The result of this BDT is used to create a tZ enriched region in the fit, reducing its impact on the measurement of WZ + heavy flavor.

The kinematic variables used as inputs to train this BDT include the invariant mass of the reconstructed top candidate, the $p_T$ of each of the leptons and associated jets, the  invariant mass of each combination of lepton pairs, $E_T^{miss}$, the distance between each combination of leptons, $\Delta R (ll)$, and the distance between each lepton and the jet, $\Delta R (lj)$.

Here the top candidate is reconstructed based on the procedure described in section 6.1 of \cite{ttZ_paper}. Broadly, the mass of the top quark candidate is reconstructed from the jet, the lepton not included in the Z-candidate, and a reconstructed neutrino. In the case that there is one jet in the event, there is only possible b-jet candidate. For events with two jets, the jet with the highest DL1r score is used.
 
The training samples included only events meeting the requirements of the 1-jet, >60\% region, i.e. passing all the selection described in section \ref{sec:evt_selection} and having exactly one jet which passes the tightest (60\%) DL1r working point.A sample of 20,000 background (tZ) and signal (WZ+b) Monte Carlo events are used to train the BDT. And additional 5,000 events are reserved for testing the model, in order to prevent over-fitting. A total of 750 decision trees with a maximum depth of 6 branches are used to build the model. These parameters are chosen empirically, by training several models with different parameters and selecting the one that gave the best separation for the test sample. The results of the BDT training are shown in figure \ref{fig:tZ_bdt}. 
\begin{figure}[H] 
\center
    \subfigure[]{\includegraphics[width=.7\linewidth]{tZ_bdt_ab127/plots_nlo/xgb_score.png}}%
    \caption{Distribution of the BDT response for WZ+$b$ (blue) and tZ (orange) events, for both training and testing samples.}
    \label{fig:tZ_bdt}
\end{figure}

The relative important of each input feature in the model, measured by how often they appeared in the decision trees, is shown in figure \ref{fig:tZ_fImp}.

\begin{figure}[H] 
\center
        \includegraphics[width=0.9\linewidth]{tZ_bdt/feature_importance.png}
        \caption{Relative importance of each input feature in the model.}
        \label{fig:tZ_fImp}
\end{figure}

A BDT score of 0.12 is selected as a cutoff, where events with scores higher than this form a signal enriched region, and events with scores lower than this form a tZ control region. This cutoff is selected by varying the value of this cutoff in stat-only Asimov fits, and selecting the value that minimizes the statistical uncertainty on WZ + b.



%------------------------------------------------------------------------------- 

%-------------------------------------------------------------------------------
%\section{Signal Region Definitions}
%\label{sec:signal_region}
%
The regions used in the fit are summarized in table \ref{tab:regions}.

\begin{table}[h]
\centering
\caption{A list of the regions used in the fit and the selection used for each.}
\begin{tabular}{l|l}
\hline\hline
Region & Selection  	      \\
\hline
\hline
1j, <85\%	& $N_{jets}$ = 1, jet MV2c10 < 85\%		      \\
1j, 85\%-77\%	& $N_{jets}$ = 1, 85\% < jet MV2c10 < 77\% 		      \\
1j, 77\%-70\%	& $N_{jets}$ = 1, 77\% < jet MV2c10 < 70\% 		      \\
1j, 70\%-60\%	& $N_{jets}$ = 1, 70\% < jet MV2c10 < 60\% 		      \\
1j, >60\%	& $N_{jets}$ = 1, jet MV2c10 > 85\%, tZ BDT score > 0.03 \\
tZ CR	& $N_{jets}$ = 1, jet MV2c10 > 85\% WP, tZ BDT score < 0.03 \\
2j, <85\%	& $N_{jets}$ = 2, jet MV2c10 score < 85\% WP		      \\
2j, 85\%-77\%	& $N_{jets}$ = 2, 85\% WP < jet MV2c10 score < 77\% WP		      \\
2j, 77\%-70\%	& $N_{jets}$ = 2, 77\% WP < jet MV2c10 score < 70\% WP		      \\
\hline\hline
\end{tabular}
\label{tab:regions}
\end{table}

The working points discussed in section \ref{subsec:bjets} are used to separate events into fit regions based on the highest working point reached by a jet in each event. Because the background composition differs significantly based on the number of b-jets, events are further subdivided into 1 jet and 2 jet regions in order to minimize the impact of background uncertainties. 

The two jet regions which fall within the tightest MV2c10 working points, between 70\%-60\% and 60\%, are left out of the fit. These regions are found to be dominated by leptonically decaying $t\bar{t}$ with an additional fake lepton. Because these regions add little statistical significance to the measurement while introducing large systematic uncertainties because of the prescence of a fake lepton, these regions are not included.

An additional tZ control region is created based on the BDT described in section \ref{sec:tZ_bdt}. The region with 1-jet passing the 60\% working point is split in two - a signal enriched region of events with a BDT score greater than 0.03, and a tZ control region including events with less than 0.03. This cutoff is arrived at by performing an Asimov fit with a variety of cutoffs, and selecting the value that produces the highest significance for the measurement of $WZ$ + b. 


%-------------------------------------------------------------------------------

%-------------------------------------------------------------------------------
\section{Systematic Uncertainties}
\label{sec:sys}

The systematic uncertainties that are considered are summarized in table \ref{tab:SystSummary}. These are implemented in the fit either as a normalization factors or as a shape variation or both in the signal and background estimations. The numerical impact of each of these uncertainties is outlined in section \ref{sec:results}.

\begin{table}[h]
\centering
\caption{Sources of systematic uncertainty considered in the analysis.}
\begin{tabular}{lr}
\hline\hline
Systematic uncertainty & Components  	      \\
\hline
\hline
Luminosity	& 1		      \\
Pileup reweighting 	& 1		      \\
\textbf {Physics Objects}     	&		      \\
\ \ Electron                               	& 6		      \\
\ \ Muon	& 15		      \\
\ \ Jet energy scale and resolution  	& 28                  \\
\ \ Jet vertex fraction  	& 1		      \\
\ \ Jet flavor tagging   	& 131		      \\
\ \ $E^{miss}_T$  	& 3		      \\
\hline
Total (Experimental)        & 186		     \\
\hline
\hline
\textbf {Background Modeling}          	&		      \\
\ \ Cross section                 	& 24		      \\
\ \ Renormalization and factorization scales 	& 10		      \\
\ \ Parton shower and hadronization model       	& 2		      \\
\ \ Shower tune				& 4		      \\
\hline
Total (Signal and background modeling)       & 40		     \\
\hline\hline
Total (Overall)                             & 226	      \\
\hline\hline
\end{tabular}
\label{tab:SystSummary}
\end{table}

The uncertainty in the combined integrated luminosity is derived from a calibration of the luminosity scale performed in August 2015 and May 2016 \cite{lumi}.

The experimental uncertainties are related to the reconstruction and identification of light leptons and b-tagging of jets, and to the reconstruction of $E^{miss}_T$. The sources which contribute to the uncertainty in the jet energy scale \cite{jes} are decomposed into uncorrelated components and treated as independent sources in the analysis. 

The uncertainties in the b-tagging efficiencies measured in dedicated calibration analyses \cite{btag_cal} are also decomposed into uncorrelated components. The large number of components for b-tagging is due to the calibration of the distribution of the MVA discriminant.  

The systematic uncertainties associated with the signal and background processes are accounted for by varying the cross-section of each process within its uncertainty.

The full list of systematic uncertainties considered in the analysis is summarized in tables
\ref{Tab:LeptonExperimentalSyst}, \ref{Tab:JetsExperimentalSyst} and \ref{Tab:BTagExperimentalSyst}.

\hspace{-1in}\begin{table}[H]
  \begin{center}
    {\small
    \begin{tabular}{|llcl|}
      \hline
      \multicolumn{4}{|c|}{\bf Experimental Systematics on Leptons and $E_T^{miss}$} \\
     % \hline
      Type     & Description  & Systematics Name & Application \\
     \hline
     \hline
     \multicolumn{4}{|c|}{\bf{Trigger}}\\
     \hline
    Scale Factors    & Trigger Efficiency        & lepSFTrigTight$\_$MU(EL)$\_$SF$\_$Trigger$\_$STAT(SYST)    & Event Weight      \\
      \hline
      \multicolumn{4}{|c|}{\bf{Muons}} \\
      \hline
      Efficiencies   & Reconstruction and        & lepSFObjTight$\_$MU$\_$SF$\_$ID$\_$STAT(SYST)              & Event Weight       \\
     & Identification    &       &        \\
      & Isolation                 &       lepSFObjTight$\_$MU$\_$SF$\_$Isol$\_$STAT(SYST)            & Event Weight       \\
         & Track To Vertex   	 & lepSFObjTight$\_$MU$\_$SF$\_$TTVA$\_$STAT(SYST )           & Event Weight       \\
    & Association  		 &   							      &           \\
     \pt Scale   & \pt Scale & MUONS$\_$SCALE    & \pt Correction     \\
     &   &   &           \\
      Resolution     & Inner Detector            & MUONS$\_$ID        					      & \pt Correction     \\
         & Energy Resolution      	 &     &         \\
    & Muon Spectrometer    	 & MUONS$\_$MS      & \pt Correction     \\
     & Energy Resolution         &       &        \\
     &   &   &         \\
     \hline
     \multicolumn{4}{|c|}{\bf{Electrons}}\\
     \hline
     Efficiencies    & Reconstruction       	 & lepSFObjTight$\_$EL$\_$SF$\_$ID  			      & Event Weight   	    \\
     & Identification   & lepSFObjTight$\_$EL$\_$SF$\_$Reco       		      & Event Weight            \\
        & Isolation                 & lepSFObjTight$\_$EL$\_$SF$\_$Isol      		      & Event Weight        \\
       &   &   &          \\
     Scale Factor    & Energy  Scale             & EG$\_$SCALE$\_$ALL  					      & Energy Correction    \\
         	     &   &   &          \\
     Resolution      & Energy Resolution  	 & EG$\_$RESOLUTION$\_$ALL      			      & Energy Correction     \\
         	     &   &   &             \\
     \hline
     \multicolumn{4}{|c|}{\bf{$E_T^{miss}$}}\\
     \hline
     Soft Tracks Terms         &             Resolution                   &      MET$\_$SoftTrk$\_$ResoPerp       &   \pt Correction  \\
                               &             Resolution                   &      MET$\_$SoftTrk$\_$ResoPara        &    \pt Correction    \\
                               &             Scale                        &      MET$\_$SoftTrk$\_$ScaleUp         &   \pt Correction     \\
                               &             Scale                        &      MET$\_$SoftTrk$\_$ScaleDown         &   \pt Correction     \\

     \hline
     
    \end{tabular}
   }
   \caption{\label{Tab:LeptonExperimentalSyst} Summary of experimental systematics considered for leptons and $E_T^{miss}$. Includes type, description, name of systematic as used in the fit, and mode of application. The mode of application indicates the systematic evaluation, e.g. as an  overall event re-weighting (Event Weight) or rescaling (\pt Correction).}
  \end{center}
\end{table}


\begin{table}[H]
  \begin{center}
    {\small
    \begin{tabular}{|llcc|}
      \hline
      \multicolumn{4}{|c|}{\bf Experimental Systematics on Jets} \\
      \hline
      Type     & Origin   & Systematics Name  & Application \\
      \hline
      Jet Vertex Tagger         &     & JVT      &        Event Weight          \\
     	&   &   &     \\
      Energy Scale              & Calibration Method              & JET$\_$21NP$\_$           &      \pt Correction         \\
       &   & JET$\_$EffectiveNP$\_$1-19     &    \pt Correction  \\
       &   &   &       \\
        & $\eta$ inter-calibration        & JET$\_$EtaIntercalibration$\_$Modelling    & \pt Correction          \\
     &                                 & JET$\_$EtaIntercalibration$\_$NonClosure   & \pt Correction      \\
     &                                 & JET$\_$EtaIntercalibration$\_$TotalStat    & \pt Correction      \\
    &   &   &        \\
     & High \pt jets                   & JET$\_$SingleParticle$\_$HighPt         &     \pt Correction             \\
     	&   &   &           \\
        & Pile-Up                         & JET$\_$Pileup$\_$OffsetNPV            &     \pt Correction             \\
        &       & JET$\_$Pileup$\_$OffsetMu             &     \pt Correction               \\
        &        & JET$\_$Pileup$\_$PtTerm       &     \pt Correction         \\
        &                                         & JET$\_$Pileup$\_$RhoTopology      &     \pt Correction             \\
    	&   &   &            \\
          & Non Closure                     & JET$\_$PunchThrough$\_$MC15    & \pt Correction    \\
    	&   &   &       \\
         & Flavour                         & JET$\_$Flavor$\_$Response          &   \pt Correction            \\
     &         & JET$\_$BJES$\_$Response          &   \pt Correction           \\
           &                                 & JET$\_$Flavor$\_$Composition        &    \pt Correction             \\
         	&   &   &          \\
      Resolution         	&                                 & JET$\_$JER$\_$SINGLE$\_$NP          &  Event Weight       \\
        			&   &   &          \\
        			
    \hline

     \end{tabular}
    }
    \caption{\label{Tab:JetsExperimentalSyst} Jet systematics take into account effects of jets calibration method, $\eta$ inter-calibration, high \pt jets, pile-up, and flavor response. They are all diagonalised into effective parameters.}
 \end{center}
\end{table}

\begin{table}[H]
  \begin{center}
    {\small
    \begin{tabular}{|llc|}
      \hline
     \multicolumn{3}{|c|}{\bf Experimental Systematics on b-tagging} \\
      \hline
      Type     & Origin   & Systematic Name \\
     \hline
     &   &                \\
      Scale Factors & DL1r b-tagger efficiency & DL1r$\_$Continuous$\_$EventWeight$\_$B0-29 \\
      &    on b originated jets in bins of $\eta$  &   \\
      &   &                \\
      &    DL1r b-tagger efficiency & DL1r$\_$Continuous$\_$EventWeight$\_$C0-19  \\
      &    on c originated jets in bins of $\eta$    &     \\
      &   &   \\
      &    DL1r b-tagger efficiency & DL1r$\_$Continuous$\_$EventWeight$\_$Light0-79           \\
      &    on light flavoured originated jets         &   \\
     &     in bins of $\eta$ and \pt      &    \\
         &   &             \\
     &    DL1r b-tagger                        & DL1r$\_$Continuous$\_$EventWeight$\_$extrapolation  \\
     &    extrapolation efficiency    &         DL1r$\_$Continuous$\_$EventWeight$\_$extrapolation$\_$from$\_$charm             \\
     \hline
    \end{tabular}
    }
    \caption{\label{Tab:BTagExperimentalSyst} Summary of experimental systematics to be included for $b$-tagging of jets in the analysis, using the continuous DL1r tagging algorithm. All of the b-tagging related systematics are applied as event weights. From left: type, description, and the name of systematic used in the fit.}
  \end{center}
\end{table}


%-------------------------------------------------------------------------------

%-------------------------------------------------------------------------------
\section{Results}
\label{sec:results}

A separate maximum-likelihood fit is performed over the 1-jet and 2-jet fit regions in order to extract the best-fit value of the WZ + $b$ and WZ + $c$ jet contributions. The $WZ$ + b, $WZ$ + charm and $WZ$ + light contributions are allowed to float, with the remaining background contributions are held fixed. \textbf{The current fit strategy treats the WZ + $b$ contribution as the parameter of interest, with the normalization of the $WZ$ + charm and the $WZ$ + light contributions taken as systematic uncertainties. This could however be adjusted, depending on whether it is decided the goal of the analysis should be to measure WZ + $b$ specifically or WZ + heavy flavor overall.} The result of the fit is used to extract the cross-section of $WZ$ + heavy-flavor production.

A maximum likelihood fit to data is performed simultaneously in the regions described in Section \ref{sec:evt_selection}. The parameters $\mu_{WZ+b}$, $\mu_{WZ+charm}$, $\mu_{WZ+light}$, where $\mu = \sigma_{observed}/\sigma_{SM} $, are extracted from the fit.

%While in the nominal fit events with an intermediate top quark are treated as a separate process, an alternative version of the measurement is performed which considers tZ as part of the signal.

The Asimov fit for 1-jet events gives an expected $\mu$ value of $1.00^{+0.37}_{-0.35}(stat)^{+0.24}_{-0.22}(sys)$ for $WZ$ + b. The fitted cross-section modifiers for $WZ$ + charm and $WZ$ + light are $1.00 \pm 0.17 \pm 0.15$ and $1.00 \pm 0.04 \pm 0.07 $, respectively.

The expected cross-section of WZ + $b$ with 1 associated jet obtained from the fit is $1.74^{+0.65}_{-0.61}(stat)^{+0.42}_{-0.37}(sys)$ fb with an expected significance of 2.2$\sigma$. The expected cross-sectin of WZ + $c$ is measured to be $14.6 \pm 2.5 (stat) \pm 2.1 (sys)$ fb, with a correlation of -0.23. 


For 2-jet events, the fit gives an expected $\mu$ value of $1.00^{+0.34}_{-0.33}(stat)^{+0.29}_{-0.28}(sys)$ for $WZ$ + b. The fitted cross-section modifiers for $WZ$ + charm and $WZ$ + light are $1.00 \pm 0.17 \pm 0.15$ and $1.00 \pm 0.04 \pm 0.08 $, respectively.

The expected $WZ$ + b cross-section in the 2-jet region is $2.46^{+0.83}_{-0.81}(stat)^{+0.0.73}_{-0.68}(sys)$ fb with an expected significance of 2.6$\sigma$. The 2-jet expected cross-section of $WZ$ + charm is $12.7 \pm 2.2 (stat) \pm 2.0 (sys)$ fb, and the correlation between WZ + $c$ and WZ + $b$ is -0.26. 

\subsection{1-jet Fit Results}

\textbf{The results of the fit are currently blinded.} 

The pre-fit yields in each of the regions used in the fit are shown in Table \ref{tab:prefitYield1j}, and summarized in Figure \ref{fig:prefitSummary_1j}.

\hspace{-1in}\begin{table}[H]
\begin{adjustwidth}{-3em}{-3em}
\small
\begin{tabular}{|l|c|c|c|c|c|c|}
\hline 
Sample & {1j, $<$85\% WP} & {1j, 77\%-85\% WP} & {1j, 70\%-77\% WP} & {1j, 60\%-70\% WP} & {1j, 60\%+ WP} & {tZ CR}\\
\hline 
  $WZ + b - 1j$   & 8.1 $\pm$ 1.6 & 4.7 $\pm$ 0.5 & 4.6 $\pm$ 0.4 & 5.1 $\pm$ 0.4 & 18.1 $\pm$ 2.4 & 5.0 $\pm$ 0.6 \\ 
  $WZ + c - 1j$   & 260 $\pm$ 22 & 81 $\pm$ 6 & 43.1 $\pm$ 3.6 & 25.8 $\pm$ 2.6 & 9.4 $\pm$ 1.8 & 2.9 $\pm$ 0.6 \\ 
  $WZ + l - 1j$   & 3090 $\pm$ 250 & 91 $\pm$ 13 & 17 $\pm$ 3 & 4.9 $\pm$ 1.6 & 1.3 $\pm$ 0.4 & 0.2 $\pm$ 0.1 \\ 
  $WZ + b - 2j$   & 1.10 $\pm$ 0.37 & 0.44 $\pm$ 0.11 & 0.39 $\pm$ 0.06 & 0.62 $\pm$ 0.14 & 2.1 $\pm$ 0.5 & 0.59 $\pm$ 0.14 \\
  $WZ + c - 2j$   & 21 $\pm$ 5 & 5.6 $\pm$ 1.2 & 3.0 $\pm$ 0.7 & 2.0 $\pm$ 0.5 & 0.70 $\pm$ 0.20 & 0.30 $\pm$ 0.08\\
  $WZ + l - 2j$   & 250 $\pm$ 60 & 5.7 $\pm$ 1.6 & 0.73 $\pm$ 0.53 & 0.31 $\pm$ 0.15 & 0.07 $\pm$ 0.06 & 0.01 $\pm$ 0.01 \\
  $WZ - Other$   & 13 $\pm$ 5 & 1.4 $\pm$ 0.4 & 0.42 $\pm$ 0.08 & 0.2 $\pm$ 0.01 & 0.30 $\pm$ 0.05 & 0.67 $\pm$ 0.15 \\
  Other VV   & 6.2 $\pm$ 0.6 & 0.2 $\pm$ 0.4 & 0.2 $\pm$ 0.04 & 0.07 $\pm$ 0.1 & 0.1 $\pm$ 0.1 & 0.1 $\pm$ 0.2 \\ 
  ZZ   & 336 $\pm$ 26 & 17.8 $\pm$ 2.1 & 4.3 $\pm$ 0.6 & 1.7 $\pm$ 0.5 & 0.36 $\pm$ 0.08 & 0.10 $\pm$ 0.03 \\ 
  $t\bar{t}W$   & 1.1 $\pm$ 0.2 & 0.2 $\pm$ 0.1 & 0.3 $\pm$ 0.1 & 0.4 $\pm$ 0.1 & 1.5 $\pm$ 0.3 & 0.7 $\pm$ 0.2 \\ 
  $t\bar{t}Z$   & 6.8 $\pm$ 1.2 & 1.4 $\pm$ 0.3 & 1.0 $\pm$ 0.2 & 1.2 $\pm$ 0.2 & 4.4 $\pm$ 0.8 & 3.2 $\pm$ 0.6 \\ 
  $Z+\text{jets}$   & 169 $\pm$ 38 & 8.9 $\pm$ 1.9 & 3.7 $\pm$ 0.8 & 3.3 $\pm$ 0.7 & 3.2 $\pm$ 0.7 & 0.8 $\pm$ 0.17 \\ 
  $V+\gamma$   & 45 $\pm$ 28 & 1.9 $\pm$ 2.4 & 0.1 $\pm$ 0.1 & 0.02 $\pm$ 0.01 & 1.0 $\pm$ 0.9 & 0.02 $\pm$ 0.03 \\ 
  $tZ$   & 31.8 $\pm$ 4.3 & 6.4 $\pm$ 1.1 & 5.3 $\pm$ 0.8 & 7.2 $\pm$ 1.1 & 11.8 $\pm$ 2.0 & 33.9 $\pm$ 4.5 \\ 
  $tW$   & 1.4 $\pm$ 0.8 & 0.2 $\pm$ 0.5 & 0.0 $\pm$ 0.2 & 0.7 $\pm$ 0.6 & 0.26 $\pm$ 0.42 & 0.39 $\pm$ 0.41 \\ 
  $WtZ$   & 2.3 $\pm$ 1.2 & 0.6 $\pm$ 0.3 & 0.3 $\pm$ 0.21 & 0.27 $\pm$ 0.2 & 1.1 $\pm$ 0.7 & 0.6 $\pm$ 0.5 \\ 
  $VVV$   & 12.4 $\pm$ 0.5 & 0.93 $\pm$ 0.06 & 0.35 $\pm$ 0.03 & 0.13 $\pm$ 0.02 & 0.14 $\pm$ 0.03 & 0.02 $\pm$ 0.01 \\ 
  $VH$   & 40 $\pm$ 6 & 2.6 $\pm$ 1.4 & 0.9 $\pm$ 0.8 & 0.7 $\pm$ 0.8 & 0.5 $\pm$ 0.6 & 0.0 $\pm$ 0.0 \\ 
  $t\bar{t}$   & 12.1 $\pm$ 1.6 & 2.9 $\pm$ 0.6 & 2.5 $\pm$ 0.5 & 2.8 $\pm$ 0.5 & 11.2 $\pm$ 1.4 & 10.9 $\pm$ 1.5 \\ 
  $t\bar{t}H$   & 0.24 $\pm$ 0.03 & 0.05 $\pm$ 0.01 & 0.04 $\pm$ 0.01 & 0.06 $\pm$ 0.01 & 0.20 $\pm$ 0.03 & 0.13 $\pm$ 0.02 \\ 
\hline 
  Total  & 5010 $\pm$ 260 & 227 $\pm$ 24 & 88 $\pm$ 12 & 57 $\pm$ 8 & 76 $\pm$ 16 & 53 $\pm$ 8 \\ 
\hline 
\end{tabular} 


\label{tab:prefitYield1j}
\caption{Pre-fit yields in each of the 1-jet fit regions.}
\end{adjustwidth}
\end{table}

\begin{figure}[H]
  \center
  \includegraphics[width=.9\linewidth]{1j/Summary.png}
  \caption{Pre-fit summary of the 1-jet fit regions.}
  \label{fig:prefitSummary_1j}
\end{figure}

%\begin{figure}[H]
%    \center
%    \includegraphics[width=.35\linewidth]{1j/not_85.png}%                                        
%    \includegraphics[width=.35\linewidth]{1j/WP_1b_77_85.png}%                                           
%    \includegraphics[width=.35\linewidth]{1j/WP_1b_70_77.png}\\
%    \includegraphics[width=.35\linewidth]{1j/WP_1b_60_70.png}%                                     
%    \includegraphics[width=.35\linewidth]{1j/WP_1b_60.png}%                                      
%    \includegraphics[width=.35\linewidth]{1j/tZ_CR.png}\\
%    \caption{Data/MC yields in each of the 1-jet regions before the fit has been performed.}
 %   \label{fig:prefit_1j}
%\end{figure}

The post-fit yields in each region are summarized in Figure \ref{tab:postfitYield1j}.

\hspace{-1in}\begin{table}[H]
\begin{adjustwidth}{-3em}{-3em}
\begin{tabular}{|l|c|c|c|c|c|c|}
\hline 
 & {1j, $<$85\% WP} & {1j, 77\%-85\% WP} & {1j, 70\%-77\% WP} & {1j, 60\%-70\% WP} & {1j, $>$60\% WP} & {tZ CR}\\
\hline 
  $WZ + b$   & 11.15 $\pm$ 4.94 & 4.68 $\pm$ 2.00 & 4.61 $\pm$ 1.99 & 5.14 $\pm$ 2.14 & 24.28 $\pm$ 10.11 & 6.01 $\pm$ 2.50 \\ 
  $WZ + c$   & 318.25 $\pm$ 61.13 & 80.74 $\pm$ 14.38 & 43.26 $\pm$ 7.65 & 25.82 $\pm$ 4.87 & 9.40 $\pm$ 2.31 & 2.86 $\pm$ 0.72 \\ 
  $WZ + l$   & 4017.60 $\pm$ 131.71 & 90.52 $\pm$ 11.37 & 17.30 $\pm$ 2.81 & 4.94 $\pm$ 1.58 & 1.25 $\pm$ 0.42 & 0.23 $\pm$ 0.13 \\ 
  Other VV + b   & 0.09 $\pm$ 0.09 & 0.01 $\pm$ 0.22 & 0.01 $\pm$ 0.01 & 0.07 $\pm$ 0.06 & 0.04 $\pm$ 0.10 & 0.03 $\pm$ 0.04 \\ 
  Other VV + c   & 0.67 $\pm$ 0.38 & 0.08 $\pm$ 0.12 & 0.03 $\pm$ 0.03 & 0.07 $\pm$ 0.06 & 0.04 $\pm$ 0.05 & 0.01 $\pm$ 0.01 \\ 
  Other VV + l   & 6.20 $\pm$ 0.64 & 0.19 $\pm$ 0.07 & 0.02 $\pm$ 0.01 & 0.01 $\pm$ 0.01 & 0.01 $\pm$ 0.01 & 0.01 $\pm$ 0.01 \\ 
  ZZ + b   & 1.33 $\pm$ 0.72 & 0.69 $\pm$ 0.39 & 0.49 $\pm$ 0.27 & 1.14 $\pm$ 0.62 & 3.59 $\pm$ 1.89 & 1.04 $\pm$ 0.57 \\ 
  ZZ + c   & 10.27 $\pm$ 5.33 & 2.40 $\pm$ 1.27 & 1.45 $\pm$ 0.80 & 1.01 $\pm$ 0.54 & 0.56 $\pm$ 0.33 & 0.12 $\pm$ 0.09 \\ 
  ZZ + l  & 336.48 $\pm$ 55.34 & 17.81 $\pm$ 3.07 & 4.33 $\pm$ 0.79 & 1.66 $\pm$ 0.35 & 0.36 $\pm$ 0.12 & 0.10 $\pm$ 0.05 \\ 
  $t\bar{t}W$   & 1.09 $\pm$ 0.21 & 0.23 $\pm$ 0.07 & 0.28 $\pm$ 0.08 & 0.38 $\pm$ 0.10 & 1.46 $\pm$ 0.29 & 0.71 $\pm$ 0.15 \\ 
  $t\bar{t}Z$   & 6.83 $\pm$ 1.16 & 1.41 $\pm$ 0.25 & 0.99 $\pm$ 0.18 & 1.20 $\pm$ 0.23 & 4.39 $\pm$ 0.73 & 3.21 $\pm$ 0.54 \\ 
  rare Top   & 0.14 $\pm$ 0.04 & 0.04 $\pm$ 0.02 & 0.04 $\pm$ 0.02 & 0.08 $\pm$ 0.03 & 0.14 $\pm$ 0.04 & 0.15 $\pm$ 0.05 \\ 
  $t\bar{t}WW$   & 0.04 $\pm$ 0.03 & 0.01 $\pm$ 0.02 & 0.01 $\pm$ 0.01 & 0.01 $\pm$ 0.01 & 0.01 $\pm$ 0.02 & 0.01 $\pm$ 0.01 \\ 
  $Z+\text{jets}$   & 168.84 $\pm$ 37.57 & 8.87 $\pm$ 1.92 & 3.73 $\pm$ 0.78 & 3.30 $\pm$ 0.68 & 3.18 $\pm$ 0.71 & 0.80 $\pm$ 0.17 \\ 
  $W+\text{jets}$   & 0.01 $\pm$ 0.01 & 0.01 $\pm$ 0.01 & 0.01 $\pm$ 0.01 & 0.01 $\pm$ 0.01 & 0.01 $\pm$ 0.01 & 0.01 $\pm$ 0.01 \\ 
  $V+\gamma$   & 45.67 $\pm$ 58.19 & 1.98 $\pm$ 2.41 & 0.13 $\pm$ 0.12 & 0.02 $\pm$ 0.15 & 1.01 $\pm$ 0.91 & 0.02 $\pm$ 0.03 \\ 
  $tZ$   & 24.25 $\pm$ 3.97 & 5.49 $\pm$ 1.01 & 4.12 $\pm$ 0.78 & 5.89 $\pm$ 1.05 & 10.73 $\pm$ 1.79 & 23.32 $\pm$ 3.73 \\ 
  $tW$   & 1.37 $\pm$ 0.82 & 0.18 $\pm$ 0.26 & 0.01 $\pm$ 0.12 & 0.67 $\pm$ 0.64 & 0.26 $\pm$ 0.42 & 0.39 $\pm$ 0.41 \\ 
  $WtZ$   & 2.33 $\pm$ 1.20 & 0.55 $\pm$ 0.31 & 0.27 $\pm$ 0.16 & 0.27 $\pm$ 0.17 & 1.05 $\pm$ 0.56 & 0.61 $\pm$ 0.34 \\ 
  $VVV$   & 12.38 $\pm$ 0.45 & 0.93 $\pm$ 0.06 & 0.35 $\pm$ 0.03 & 0.13 $\pm$ 0.02 & 0.14 $\pm$ 0.03 & 0.02 $\pm$ 0.01 \\ 
  $VH$   & 40.19 $\pm$ 6.22 & 2.56 $\pm$ 1.41 & 0.90 $\pm$ 0.77 & 0.74 $\pm$ 0.75 & 0.45 $\pm$ 0.63 & 0.01 $\pm$ 0.01 \\ 
  $t\bar{t}$   & 12.14 $\pm$ 1.58 & 2.88 $\pm$ 0.55 & 2.46 $\pm$ 0.50 & 2.75 $\pm$ 0.53 & 11.22 $\pm$ 1.47 & 10.92 $\pm$ 1.44 \\ 
  $t\bar{t}H$   & 0.24 $\pm$ 0.03 & 0.05 $\pm$ 0.01 & 0.04 $\pm$ 0.01 & 0.06 $\pm$ 0.01 & 0.20 $\pm$ 0.03 & 0.13 $\pm$ 0.02 \\ 
\hline 
  Total  & 5099.12 $\pm$ 113.23 & 226.54 $\pm$ 12.02 & 86.71 $\pm$ 6.24 & 56.65 $\pm$ 4.35 & 76.05 $\pm$ 8.71 & 52.53 $\pm$ 4.15 \\ 
\hline 
\end{tabular} 

\label{tab:postfitYield1j}
\caption{Post-fit yields in each of the 1-jet fit regions.}                                                                 
\end{adjustwidth} 
\end{table}

A post-fit summary plot of the 1-jet fitted regions is shown in Figure \ref{fig:fit_results_1j}: 

\begin{figure}[H]
    \center
    \includegraphics[width=.9\linewidth]{1j/Summary_postFit.png}
    \caption{Post-fit summary of the 1-jet fit regions.}
    \label{fig:fit_results_1j}
\end{figure}

As described in Section \ref{sec:sys}, there are 226 systematic uncertainties that are considered as NPs in the fit. These NPs are constrained by Gaussian or log-normal probability density functions. The latter are used for normalisation factors to ensure that they are always positive. The expected number of signal and background events are functions of the likelihood. The prior for each NP is added as a penalty term, decreasing the likelihood as it is shifted away from its nominal value. 

The impact of each NP is calculated by performing the fit with the parameter of interest held fixed, varied from its fitted value by its uncertainty, and calculating $\Delta\mu$ relative to the baseline fit.  The impact of the most significant sources of systematic uncertainties is summarized in Table \ref{tab:systematics_1j}. 

\begin{table}[H]
    \centering
    \begin{tabular}{l|cc}
        \hline\hline
        Uncertainty Source & \multicolumn{2}{c}{$\Delta \mu$ }  \\
        \hline
        WZ + light cross-section & 0.13 & -0.12 \\
        WZ + $c$ cross-section & -0.10 & 0.12 \\
        Jet Energy Scale & 0.08 & 0.13 \\
        tZ cross-section & -0.10 & 0.10 \\
        Jet Energy Resolution & -0.10 & 0.10 \\
        Luminosity & -0.08 & 0.09 \\
        Other Diboson + b cross-section & -0.07 & 0.07 \\
        Flavor tagging & 0.05 & 0.05 \\
        $t\bar{t}$ cross-section & -0.05 & 0.05 \\
        WZ cross-section - QCD scale & -0.04 & 0.03 \\
        \hline
        Total Systematic Uncertainty & 0.28 & 0.32 \\
        
        %WZ + charm cross-section & -0.1966 & 0.2171 \\
        %tZ cross-section & -0.1521 & 0.1518 \\
        %WZ + light cross-section & 0.1485 & -0.1411 \\
        %Other VV + b cross-section & -0.1115 & 0.1163 \\
        %Flavor Tagging & 0.0955 & 0.0957 \\
        %Jet Energy Scale & 0.0613 & 0.081 \\
        %$t\bar{t}$ cross-section & -0.0662 & 0.0654 \\
        %Luminosity & -0.0609 & 0.0655 \\
        %Z + jets cross-section & -0.0284 & 0.0284 \\
        %Other VV + charm cross-sction & 0.0207 & -0.0202 \\
        %Muon Trigger Scale Factor & 0.019 & 0.0209 \\
        %\hline
        %Total Systematic Uncertainty & 0.3511 & 0.3679 \\
        \hline\hline
    \end{tabular}
    \caption{Summary of the most significant sources of systematic uncertainty on the measurement of $WZ+b$ with exactly one associated jet.}
    \label{tab:systematics_1j}
\end{table}

The ranking and impact of those nuisance parameters with the largest contribution to the overall uncertainty is shown in Figure \ref{fig:ranking_1j}.

\begin{figure}[H]
    \centering
    \includegraphics[width=0.7\linewidth]{1j/Ranking.png}
    \caption{Impact of systematic uncertainties on the signal-strength of $WZ$ + b for events with exactly one jet}
    \label{fig:ranking_1j}
\end{figure}

The large impact of the Jet Energy Scale and Jet Flavor Tagging is unsurprising, as the shape of the fit regions depends heavily on the modeling of the jets. The other major sources of uncertainty come from background modelling and cross-section uncertainty. The pie charts in Figure \ref{fig:pie_chart_1j} show that for the modelling uncertainties that contribute most correspond to the most significant backgrounds. %The pileup-reweighting and luminosity play a significant role as well, because, as shown in Figure \ref{fig:pie_chart_1j}, the signal purity is relatively small in each of the fit regions. This means that a small scaling in the background contribution resulting from a change in luminosity or pileup corresponds to a significant change in the best fit signal contribution. 

\begin{figure}[H]
    \centering
    \includegraphics[width=0.7\linewidth]{1j/PieChart_postFit.png}
    \caption{Post-fit background composition of the fit regions.}
    \label{fig:pie_chart_1j}
\end{figure}

The correlations between these nuisance parameters are summarized in Figure \ref{fig:corr_mat_1j}. 

\begin{figure}[H]
    \centering
    \includegraphics[width=1.0\linewidth]{1j/CorrMatrix.png}
    \caption{Correlations between nuisance parameters}
    \label{fig:corr_mat_1j}
\end{figure}

The negative correlations between $\mu_{WZ+charm}$ and $\mu_{WZ+b}$ and $\mu_{WZ+light}$ are expected: $WZ$ + charm is present in both the $WZ$ + b and $WZ$ + light enriched regions, therefore increasing the fraction of charm requires increasing the fraction of $WZ$ + b and $WZ$ + light. This reasoning also explains the positive correlation between $\mu_{WZ+b}$ and $\mu_{WZ+light}$. 

Two of the major backgrounds in the region with the highest purity of $WZ$ + b are tZ and Other VV + b, explaining the negative correlations between $\mu_{WZ+b}$ and the tZ cross section, and the VV + b cross section.

The high correlation between the luminosity and $\mu_{WZ+light}$ arises from the fact that the uncertainty on $\mu_{WZ+light}$ is very low (around 4\%). Small changes in luminosity cause a change in the yield of $WZ$ + light that is large compared to its uncertainty, producing a large correlation between these two parameters. 

%The other significant correlation present is between ATLAS FTAG L0 and $\mu_{WZ + light}$ and $\mu_{WZ+ charm}$. This shifts the   

%\begin{figure}[H]
%    \centering
%    \includegraphics[width=0.63\linewidth]{1j/SignalRegions.png}
%    \caption{Significance, given by $S/\sqrt{B}$ of the signal in each of the fit regions.}
%    \label{fig:signalRegions}
%\end{figure}

%\newpage
%\begin{landscape}
%\begin{table}[htbp]
%\begin{center}
%\caption{\label{tbl:yields} Background, signal and observed yields in the twelve analysis categories in 36.1 %\ifb\ of data at $\sqrt{s} =$ 13~TeV. Uncertainties in the background estimates due to systematic effects and %t.}
% \resizebox{1.3\textwidth}{!}{
%        \begin{tabular}{|c|c|c|c|c|c|c|c|c|c|}
\hline 
 & $<$85 WP & 1b, 77-85 WP & 1b, 70-77 WP & 1b, 60-70 WP & 1b, $>$60 WP & 2b, 77-85 WP & 2b, 70-77 WP & 2b, 60-70 WP & 2b, $>$60 WP\\
\hline 
  $t\bar{t}W$   & \num[round-mode=figures,round-precision=3]{2.27655} $\pm$ \num[round-mode=figures,round-precision=3]{2.30791} & \num[round-mode=figures,round-precision=3]{1.3571} $\pm$ \num[round-mode=figures,round-precision=3]{1.35835} & \num[round-mode=figures,round-precision=3]{0.941663} $\pm$ \num[round-mode=figures,round-precision=3]{0.969334} & \num[round-mode=figures,round-precision=3]{1.99269} $\pm$ \num[round-mode=figures,round-precision=3]{1.98343} & \num[round-mode=figures,round-precision=3]{10.3804} $\pm$ \num[round-mode=figures,round-precision=3]{10.3282} & \num[round-mode=figures,round-precision=3]{1.50549} $\pm$ \num[round-mode=figures,round-precision=3]{1.50488} & \num[round-mode=figures,round-precision=3]{1.16646} $\pm$ \num[round-mode=figures,round-precision=3]{1.16209} & \num[round-mode=figures,round-precision=3]{1.37188} $\pm$ \num[round-mode=figures,round-precision=3]{1.36002} & \num[round-mode=figures,round-precision=3]{3.85472} $\pm$ \num[round-mode=figures,round-precision=3]{3.818} \\ 
  $t\bar{t}Z$   & \num[round-mode=figures,round-precision=3]{5.59481} $\pm$ \num[round-mode=figures,round-precision=3]{0.952561} & \num[round-mode=figures,round-precision=3]{1.85638} $\pm$ \num[round-mode=figures,round-precision=3]{0.457615} & \num[round-mode=figures,round-precision=3]{1.61823} $\pm$ \num[round-mode=figures,round-precision=3]{0.288506} & \num[round-mode=figures,round-precision=3]{2.22864} $\pm$ \num[round-mode=figures,round-precision=3]{0.409013} & \num[round-mode=figures,round-precision=3]{14.0042} $\pm$ \num[round-mode=figures,round-precision=3]{3.50867} & \num[round-mode=figures,round-precision=3]{1.86738} $\pm$ \num[round-mode=figures,round-precision=3]{0.362445} & \num[round-mode=figures,round-precision=3]{1.14193} $\pm$ \num[round-mode=figures,round-precision=3]{0.396228} & \num[round-mode=figures,round-precision=3]{1.15182} $\pm$ \num[round-mode=figures,round-precision=3]{0.425086} & \num[round-mode=figures,round-precision=3]{2.7131} $\pm$ \num[round-mode=figures,round-precision=3]{0.684245} \\ 
  $VV + b$   & \num[round-mode=figures,round-precision=3]{9.26027} $\pm$ \num[round-mode=figures,round-precision=3]{5.07692} & \num[round-mode=figures,round-precision=3]{5.31041} $\pm$ \num[round-mode=figures,round-precision=3]{2.6598} & \num[round-mode=figures,round-precision=3]{5.63769} $\pm$ \num[round-mode=figures,round-precision=3]{3.04505} & \num[round-mode=figures,round-precision=3]{4.7021} $\pm$ \num[round-mode=figures,round-precision=3]{2.56084} & \num[round-mode=figures,round-precision=3]{30.7357} $\pm$ \num[round-mode=figures,round-precision=3]{14.5166} & \num[round-mode=figures,round-precision=3]{1.89107} $\pm$ \num[round-mode=figures,round-precision=3]{1.01403} & \num[round-mode=figures,round-precision=3]{1.02904} $\pm$ \num[round-mode=figures,round-precision=3]{0.505366} & \num[round-mode=figures,round-precision=3]{1.37878} $\pm$ \num[round-mode=figures,round-precision=3]{0.681346} & \num[round-mode=figures,round-precision=3]{1.23094} $\pm$ \num[round-mode=figures,round-precision=3]{0.856462} \\ 
  $VV + c$   & \num[round-mode=figures,round-precision=3]{228.417} $\pm$ \num[round-mode=figures,round-precision=3]{51.3311} & \num[round-mode=figures,round-precision=3]{62.547} $\pm$ \num[round-mode=figures,round-precision=3]{13.4738} & \num[round-mode=figures,round-precision=3]{31.8} $\pm$ \num[round-mode=figures,round-precision=3]{7.84773} & \num[round-mode=figures,round-precision=3]{23.5735} $\pm$ \num[round-mode=figures,round-precision=3]{5.96178} & \num[round-mode=figures,round-precision=3]{23.9244} $\pm$ \num[round-mode=figures,round-precision=3]{6.99115} & \num[round-mode=figures,round-precision=3]{4.56139} $\pm$ \num[round-mode=figures,round-precision=3]{1.46901} & \num[round-mode=figures,round-precision=3]{1.23133} $\pm$ \num[round-mode=figures,round-precision=3]{1.07988} & \num[round-mode=figures,round-precision=3]{0.116983} $\pm$ \num[round-mode=figures,round-precision=3]{0.164252} & \num[round-mode=figures,round-precision=3]{--} $\pm$ \num[round-mode=figures,round-precision=3]{--} \\ 
  $VV + l$   & \num[round-mode=figures,round-precision=3]{2240.34} $\pm$ \num[round-mode=figures,round-precision=3]{85.2546} & \num[round-mode=figures,round-precision=3]{114.763} $\pm$ \num[round-mode=figures,round-precision=3]{11.1222} & \num[round-mode=figures,round-precision=3]{31.8071} $\pm$ \num[round-mode=figures,round-precision=3]{5.36505} & \num[round-mode=figures,round-precision=3]{16.8257} $\pm$ \num[round-mode=figures,round-precision=3]{3.80662} & \num[round-mode=figures,round-precision=3]{6.47758} $\pm$ \num[round-mode=figures,round-precision=3]{2.25164} & \num[round-mode=figures,round-precision=3]{2.36682} $\pm$ \num[round-mode=figures,round-precision=3]{1.23121} & \num[round-mode=figures,round-precision=3]{0.357081} $\pm$ \num[round-mode=figures,round-precision=3]{0.365519} & \num[round-mode=figures,round-precision=3]{0.0167576} $\pm$ \num[round-mode=figures,round-precision=3]{0.0232322} & \num[round-mode=figures,round-precision=3]{--} $\pm$ \num[round-mode=figures,round-precision=3]{--} \\ 
  $t\bar{t}$   & \num[round-mode=figures,round-precision=3]{27.761} $\pm$ \num[round-mode=figures,round-precision=3]{357.388} & \num[round-mode=figures,round-precision=3]{--} $\pm$ \num[round-mode=figures,round-precision=3]{--} & \num[round-mode=figures,round-precision=3]{--} $\pm$ \num[round-mode=figures,round-precision=3]{--} & \num[round-mode=figures,round-precision=3]{0.000302665} $\pm$ \num[round-mode=figures,round-precision=3]{2.1126e-05} & \num[round-mode=figures,round-precision=3]{36.0985} $\pm$ \num[round-mode=figures,round-precision=3]{407.045} & \num[round-mode=figures,round-precision=3]{0.000302665} $\pm$ \num[round-mode=figures,round-precision=3]{2.1126e-05} & \num[round-mode=figures,round-precision=3]{0.000302665} $\pm$ \num[round-mode=figures,round-precision=3]{2.1126e-05} & \num[round-mode=figures,round-precision=3]{0.000302665} $\pm$ \num[round-mode=figures,round-precision=3]{2.1126e-05} & \num[round-mode=figures,round-precision=3]{0.48602} $\pm$ \num[round-mode=figures,round-precision=3]{1.7566} \\ 
  $t\bar{t}+\gamma$   & \num[round-mode=figures,round-precision=3]{0.480558} $\pm$ \num[round-mode=figures,round-precision=3]{1.05386} & \num[round-mode=figures,round-precision=3]{0.000320031} $\pm$ \num[round-mode=figures,round-precision=3]{0.000152476} & \num[round-mode=figures,round-precision=3]{0.197334} $\pm$ \num[round-mode=figures,round-precision=3]{1.00686} & \num[round-mode=figures,round-precision=3]{0.209157} $\pm$ \num[round-mode=figures,round-precision=3]{1.06389} & \num[round-mode=figures,round-precision=3]{0.887418} $\pm$ \num[round-mode=figures,round-precision=3]{2.72954} & \num[round-mode=figures,round-precision=3]{0.492239} $\pm$ \num[round-mode=figures,round-precision=3]{2.48655} & \num[round-mode=figures,round-precision=3]{0.000320031} $\pm$ \num[round-mode=figures,round-precision=3]{0.000152476} & \num[round-mode=figures,round-precision=3]{0.000320031} $\pm$ \num[round-mode=figures,round-precision=3]{0.000152476} & \num[round-mode=figures,round-precision=3]{0.241039} $\pm$ \num[round-mode=figures,round-precision=3]{1.23072} \\ 
  Single top t-chan   & \num[round-mode=figures,round-precision=3]{0.000302869} $\pm$ \num[round-mode=figures,round-precision=3]{1.86387e-05} & \num[round-mode=figures,round-precision=3]{0.000302869} $\pm$ \num[round-mode=figures,round-precision=3]{1.86387e-05} & \num[round-mode=figures,round-precision=3]{0.000302869} $\pm$ \num[round-mode=figures,round-precision=3]{1.86387e-05} & \num[round-mode=figures,round-precision=3]{0.000302869} $\pm$ \num[round-mode=figures,round-precision=3]{1.86387e-05} & \num[round-mode=figures,round-precision=3]{0.000302869} $\pm$ \num[round-mode=figures,round-precision=3]{1.86387e-05} & \num[round-mode=figures,round-precision=3]{0.000302869} $\pm$ \num[round-mode=figures,round-precision=3]{1.86387e-05} & \num[round-mode=figures,round-precision=3]{0.000302869} $\pm$ \num[round-mode=figures,round-precision=3]{1.86387e-05} & \num[round-mode=figures,round-precision=3]{0.000302869} $\pm$ \num[round-mode=figures,round-precision=3]{1.86387e-05} & \num[round-mode=figures,round-precision=3]{0.000302869} $\pm$ \num[round-mode=figures,round-precision=3]{1.86387e-05} \\ 
  Single top s-chan   & \num[round-mode=figures,round-precision=3]{0.000302869} $\pm$ \num[round-mode=figures,round-precision=3]{1.86387e-05} & \num[round-mode=figures,round-precision=3]{0.000302869} $\pm$ \num[round-mode=figures,round-precision=3]{1.86387e-05} & \num[round-mode=figures,round-precision=3]{0.000302869} $\pm$ \num[round-mode=figures,round-precision=3]{1.86387e-05} & \num[round-mode=figures,round-precision=3]{0.000302869} $\pm$ \num[round-mode=figures,round-precision=3]{1.86387e-05} & \num[round-mode=figures,round-precision=3]{0.000302869} $\pm$ \num[round-mode=figures,round-precision=3]{1.86387e-05} & \num[round-mode=figures,round-precision=3]{0.000302869} $\pm$ \num[round-mode=figures,round-precision=3]{1.86387e-05} & \num[round-mode=figures,round-precision=3]{0.000302869} $\pm$ \num[round-mode=figures,round-precision=3]{1.86387e-05} & \num[round-mode=figures,round-precision=3]{0.000302869} $\pm$ \num[round-mode=figures,round-precision=3]{1.86387e-05} & \num[round-mode=figures,round-precision=3]{0.000302869} $\pm$ \num[round-mode=figures,round-precision=3]{1.86387e-05} \\ 
  $Wt$   & \num[round-mode=figures,round-precision=3]{0.362598} $\pm$ \num[round-mode=figures,round-precision=3]{0.654777} & \num[round-mode=figures,round-precision=3]{0.000302787} $\pm$ \num[round-mode=figures,round-precision=3]{1.86339e-05} & \num[round-mode=figures,round-precision=3]{0.108486} $\pm$ \num[round-mode=figures,round-precision=3]{0.584642} & \num[round-mode=figures,round-precision=3]{0.000302787} $\pm$ \num[round-mode=figures,round-precision=3]{1.86339e-05} & \num[round-mode=figures,round-precision=3]{0.274423} $\pm$ \num[round-mode=figures,round-precision=3]{0.503357} & \num[round-mode=figures,round-precision=3]{0.000302787} $\pm$ \num[round-mode=figures,round-precision=3]{1.86339e-05} & \num[round-mode=figures,round-precision=3]{0.000302787} $\pm$ \num[round-mode=figures,round-precision=3]{1.86339e-05} & \num[round-mode=figures,round-precision=3]{0.000302787} $\pm$ \num[round-mode=figures,round-precision=3]{1.86339e-05} & \num[round-mode=figures,round-precision=3]{0.000302787} $\pm$ \num[round-mode=figures,round-precision=3]{1.86339e-05} \\ 
  Three top   & \num[round-mode=figures,round-precision=3]{0.0003606} $\pm$ \num[round-mode=figures,round-precision=3]{0.0017759} & \num[round-mode=figures,round-precision=3]{0.000302946} $\pm$ \num[round-mode=figures,round-precision=3]{0.000150851} & \num[round-mode=figures,round-precision=3]{0.000302946} $\pm$ \num[round-mode=figures,round-precision=3]{0.000150851} & \num[round-mode=figures,round-precision=3]{0.000302946} $\pm$ \num[round-mode=figures,round-precision=3]{0.000150851} & \num[round-mode=figures,round-precision=3]{0.000612547} $\pm$ \num[round-mode=figures,round-precision=3]{0.00110042} & \num[round-mode=figures,round-precision=3]{0.000302946} $\pm$ \num[round-mode=figures,round-precision=3]{0.000150851} & \num[round-mode=figures,round-precision=3]{0.000302946} $\pm$ \num[round-mode=figures,round-precision=3]{0.000150851} & \num[round-mode=figures,round-precision=3]{0.000302946} $\pm$ \num[round-mode=figures,round-precision=3]{0.000150851} & \num[round-mode=figures,round-precision=3]{0.000773585} $\pm$ \num[round-mode=figures,round-precision=3]{0.00117832} \\ 
  Four top   & \num[round-mode=figures,round-precision=3]{0.000302952} $\pm$ \num[round-mode=figures,round-precision=3]{0.000150853} & \num[round-mode=figures,round-precision=3]{0.000302952} $\pm$ \num[round-mode=figures,round-precision=3]{0.000150853} & \num[round-mode=figures,round-precision=3]{0.000302952} $\pm$ \num[round-mode=figures,round-precision=3]{0.000150853} & \num[round-mode=figures,round-precision=3]{0.000302952} $\pm$ \num[round-mode=figures,round-precision=3]{0.000150853} & \num[round-mode=figures,round-precision=3]{0.000302952} $\pm$ \num[round-mode=figures,round-precision=3]{0.000150853} & \num[round-mode=figures,round-precision=3]{0.000302952} $\pm$ \num[round-mode=figures,round-precision=3]{0.000150853} & \num[round-mode=figures,round-precision=3]{0.000302952} $\pm$ \num[round-mode=figures,round-precision=3]{0.000150853} & \num[round-mode=figures,round-precision=3]{0.000302952} $\pm$ \num[round-mode=figures,round-precision=3]{0.000150853} & \num[round-mode=figures,round-precision=3]{0.00126998} $\pm$ \num[round-mode=figures,round-precision=3]{0.00683589} \\ 
  $t\bar{t}WW$   & \num[round-mode=figures,round-precision=3]{0.00022652} $\pm$ \num[round-mode=figures,round-precision=3]{0.00351054} & \num[round-mode=figures,round-precision=3]{0.000302883} $\pm$ \num[round-mode=figures,round-precision=3]{3.64287e-05} & \num[round-mode=figures,round-precision=3]{0.00440409} $\pm$ \num[round-mode=figures,round-precision=3]{0.0238607} & \num[round-mode=figures,round-precision=3]{0.0129946} $\pm$ \num[round-mode=figures,round-precision=3]{0.0319054} & \num[round-mode=figures,round-precision=3]{0.0062717} $\pm$ \num[round-mode=figures,round-precision=3]{0.0322067} & \num[round-mode=figures,round-precision=3]{0.00440409} $\pm$ \num[round-mode=figures,round-precision=3]{0.0238607} & \num[round-mode=figures,round-precision=3]{0.000302883} $\pm$ \num[round-mode=figures,round-precision=3]{3.64287e-05} & \num[round-mode=figures,round-precision=3]{0.000302883} $\pm$ \num[round-mode=figures,round-precision=3]{3.64287e-05} & \num[round-mode=figures,round-precision=3]{0.00571661} $\pm$ \num[round-mode=figures,round-precision=3]{0.0304709} \\ 
  $V+\text{jets}$   & \num[round-mode=figures,round-precision=3]{15.2819} $\pm$ \num[round-mode=figures,round-precision=3]{20.5512} & \num[round-mode=figures,round-precision=3]{0.373017} $\pm$ \num[round-mode=figures,round-precision=3]{0.281037} & \num[round-mode=figures,round-precision=3]{0.334766} $\pm$ \num[round-mode=figures,round-precision=3]{0.214858} & \num[round-mode=figures,round-precision=3]{0.364992} $\pm$ \num[round-mode=figures,round-precision=3]{0.306013} & \num[round-mode=figures,round-precision=3]{2.61101} $\pm$ \num[round-mode=figures,round-precision=3]{1.33608} & \num[round-mode=figures,round-precision=3]{0.141037} $\pm$ \num[round-mode=figures,round-precision=3]{0.202839} & \num[round-mode=figures,round-precision=3]{0.000291241} $\pm$ \num[round-mode=figures,round-precision=3]{0.000116691} & \num[round-mode=figures,round-precision=3]{0.000291241} $\pm$ \num[round-mode=figures,round-precision=3]{0.000116691} & \num[round-mode=figures,round-precision=3]{0.0279002} $\pm$ \num[round-mode=figures,round-precision=3]{0.152993} \\ 
  low mass $V+\text{jets}$   & \num[round-mode=figures,round-precision=3]{0.000291241} $\pm$ \num[round-mode=figures,round-precision=3]{0.000116691} & \num[round-mode=figures,round-precision=3]{0.000291241} $\pm$ \num[round-mode=figures,round-precision=3]{0.000116691} & \num[round-mode=figures,round-precision=3]{0.000291241} $\pm$ \num[round-mode=figures,round-precision=3]{0.000116691} & \num[round-mode=figures,round-precision=3]{0.000291241} $\pm$ \num[round-mode=figures,round-precision=3]{0.000116691} & \num[round-mode=figures,round-precision=3]{0.000291241} $\pm$ \num[round-mode=figures,round-precision=3]{0.000116691} & \num[round-mode=figures,round-precision=3]{0.000291241} $\pm$ \num[round-mode=figures,round-precision=3]{0.000116691} & \num[round-mode=figures,round-precision=3]{0.000291241} $\pm$ \num[round-mode=figures,round-precision=3]{0.000116691} & \num[round-mode=figures,round-precision=3]{0.000291241} $\pm$ \num[round-mode=figures,round-precision=3]{0.000116691} & \num[round-mode=figures,round-precision=3]{0.000291241} $\pm$ \num[round-mode=figures,round-precision=3]{0.000116691} \\ 
  $V+\text{jets}$   & \num[round-mode=figures,round-precision=3]{0.000302868} $\pm$ \num[round-mode=figures,round-precision=3]{1.10053e-05} & \num[round-mode=figures,round-precision=3]{0.000302868} $\pm$ \num[round-mode=figures,round-precision=3]{1.10053e-05} & \num[round-mode=figures,round-precision=3]{0.000302868} $\pm$ \num[round-mode=figures,round-precision=3]{1.10053e-05} & \num[round-mode=figures,round-precision=3]{0.000302868} $\pm$ \num[round-mode=figures,round-precision=3]{1.10053e-05} & \num[round-mode=figures,round-precision=3]{0.000302868} $\pm$ \num[round-mode=figures,round-precision=3]{1.10053e-05} & \num[round-mode=figures,round-precision=3]{0.000302868} $\pm$ \num[round-mode=figures,round-precision=3]{1.10053e-05} & \num[round-mode=figures,round-precision=3]{0.000302868} $\pm$ \num[round-mode=figures,round-precision=3]{1.10053e-05} & \num[round-mode=figures,round-precision=3]{0.000302868} $\pm$ \num[round-mode=figures,round-precision=3]{1.10053e-05} & \num[round-mode=figures,round-precision=3]{0.000302868} $\pm$ \num[round-mode=figures,round-precision=3]{1.10053e-05} \\ 
  $V+\gamma$   & \num[round-mode=figures,round-precision=3]{40.5736} $\pm$ \num[round-mode=figures,round-precision=3]{14.4393} & \num[round-mode=figures,round-precision=3]{1.10864} $\pm$ \num[round-mode=figures,round-precision=3]{0.529551} & \num[round-mode=figures,round-precision=3]{1.45353} $\pm$ \num[round-mode=figures,round-precision=3]{0.889563} & \num[round-mode=figures,round-precision=3]{0.112777} $\pm$ \num[round-mode=figures,round-precision=3]{0.585685} & \num[round-mode=figures,round-precision=3]{0.356533} $\pm$ \num[round-mode=figures,round-precision=3]{0.273389} & \num[round-mode=figures,round-precision=3]{0.000302868} $\pm$ \num[round-mode=figures,round-precision=3]{1.10053e-05} & \num[round-mode=figures,round-precision=3]{0.000302868} $\pm$ \num[round-mode=figures,round-precision=3]{1.10053e-05} & \num[round-mode=figures,round-precision=3]{0.000302868} $\pm$ \num[round-mode=figures,round-precision=3]{1.10053e-05} & \num[round-mode=figures,round-precision=3]{0.000302868} $\pm$ \num[round-mode=figures,round-precision=3]{1.10053e-05} \\ 
  $tZ$   & \num[round-mode=figures,round-precision=3]{11.5641} $\pm$ \num[round-mode=figures,round-precision=3]{1.10741} & \num[round-mode=figures,round-precision=3]{3.83557} $\pm$ \num[round-mode=figures,round-precision=3]{0.344865} & \num[round-mode=figures,round-precision=3]{3.22506} $\pm$ \num[round-mode=figures,round-precision=3]{0.290839} & \num[round-mode=figures,round-precision=3]{4.41135} $\pm$ \num[round-mode=figures,round-precision=3]{0.403135} & \num[round-mode=figures,round-precision=3]{27.7444} $\pm$ \num[round-mode=figures,round-precision=3]{2.48965} & \num[round-mode=figures,round-precision=3]{1.56275} $\pm$ \num[round-mode=figures,round-precision=3]{0.159151} & \num[round-mode=figures,round-precision=3]{0.745832} $\pm$ \num[round-mode=figures,round-precision=3]{0.0819564} & \num[round-mode=figures,round-precision=3]{0.665967} $\pm$ \num[round-mode=figures,round-precision=3]{0.0689025} & \num[round-mode=figures,round-precision=3]{1.31404} $\pm$ \num[round-mode=figures,round-precision=3]{0.136992} \\ 
  $WtZ$   & \num[round-mode=figures,round-precision=3]{3.34641} $\pm$ \num[round-mode=figures,round-precision=3]{1.63649} & \num[round-mode=figures,round-precision=3]{1.01967} $\pm$ \num[round-mode=figures,round-precision=3]{0.532459} & \num[round-mode=figures,round-precision=3]{0.912386} $\pm$ \num[round-mode=figures,round-precision=3]{0.464022} & \num[round-mode=figures,round-precision=3]{0.7321} $\pm$ \num[round-mode=figures,round-precision=3]{0.373417} & \num[round-mode=figures,round-precision=3]{4.21335} $\pm$ \num[round-mode=figures,round-precision=3]{2.05616} & \num[round-mode=figures,round-precision=3]{0.458102} $\pm$ \num[round-mode=figures,round-precision=3]{0.246527} & \num[round-mode=figures,round-precision=3]{0.221432} $\pm$ \num[round-mode=figures,round-precision=3]{0.13283} & \num[round-mode=figures,round-precision=3]{0.248025} $\pm$ \num[round-mode=figures,round-precision=3]{0.138019} & \num[round-mode=figures,round-precision=3]{0.162435} $\pm$ \num[round-mode=figures,round-precision=3]{0.100658} \\ 
  $VVV$   & \num[round-mode=figures,round-precision=3]{6.22694} $\pm$ \num[round-mode=figures,round-precision=3]{3.09521} & \num[round-mode=figures,round-precision=3]{0.473923} $\pm$ \num[round-mode=figures,round-precision=3]{0.254746} & \num[round-mode=figures,round-precision=3]{0.10713} $\pm$ \num[round-mode=figures,round-precision=3]{0.0617345} & \num[round-mode=figures,round-precision=3]{0.117184} $\pm$ \num[round-mode=figures,round-precision=3]{0.065154} & \num[round-mode=figures,round-precision=3]{0.0592735} $\pm$ \num[round-mode=figures,round-precision=3]{0.03952} & \num[round-mode=figures,round-precision=3]{0.0291393} $\pm$ \num[round-mode=figures,round-precision=3]{0.0270894} & \num[round-mode=figures,round-precision=3]{0.00323562} $\pm$ \num[round-mode=figures,round-precision=3]{0.00432544} & \num[round-mode=figures,round-precision=3]{0.00030518} $\pm$ \num[round-mode=figures,round-precision=3]{0.000151427} & \num[round-mode=figures,round-precision=3]{0.00030518} $\pm$ \num[round-mode=figures,round-precision=3]{0.000151427} \\ 
  $VH$   & \num[round-mode=figures,round-precision=3]{23.9715} $\pm$ \num[round-mode=figures,round-precision=3]{6.0978} & \num[round-mode=figures,round-precision=3]{0.852682} $\pm$ \num[round-mode=figures,round-precision=3]{1.3381} & \num[round-mode=figures,round-precision=3]{0.309145} $\pm$ \num[round-mode=figures,round-precision=3]{1.5312} & \num[round-mode=figures,round-precision=3]{0.764878} $\pm$ \num[round-mode=figures,round-precision=3]{2.34321} & \num[round-mode=figures,round-precision=3]{0.000302868} $\pm$ \num[round-mode=figures,round-precision=3]{1.10053e-05} & \num[round-mode=figures,round-precision=3]{0.000302868} $\pm$ \num[round-mode=figures,round-precision=3]{1.10053e-05} & \num[round-mode=figures,round-precision=3]{0.000302868} $\pm$ \num[round-mode=figures,round-precision=3]{1.10053e-05} & \num[round-mode=figures,round-precision=3]{0.000302868} $\pm$ \num[round-mode=figures,round-precision=3]{1.10053e-05} & \num[round-mode=figures,round-precision=3]{0.000302868} $\pm$ \num[round-mode=figures,round-precision=3]{1.10053e-05} \\ 
  $tHjb$   & \num[round-mode=figures,round-precision=3]{0.0419189} $\pm$ \num[round-mode=figures,round-precision=3]{0.0144183} & \num[round-mode=figures,round-precision=3]{0.0093033} $\pm$ \num[round-mode=figures,round-precision=3]{0.00855833} & \num[round-mode=figures,round-precision=3]{0.000302911} $\pm$ \num[round-mode=figures,round-precision=3]{3.57751e-05} & \num[round-mode=figures,round-precision=3]{0.00324645} $\pm$ \num[round-mode=figures,round-precision=3]{0.00803724} & \num[round-mode=figures,round-precision=3]{0.0372004} $\pm$ \num[round-mode=figures,round-precision=3]{0.0112587} & \num[round-mode=figures,round-precision=3]{0.00311466} $\pm$ \num[round-mode=figures,round-precision=3]{0.00945198} & \num[round-mode=figures,round-precision=3]{0.00207475} $\pm$ \num[round-mode=figures,round-precision=3]{0.00777246} & \num[round-mode=figures,round-precision=3]{0.00169069} $\pm$ \num[round-mode=figures,round-precision=3]{0.00892229} & \num[round-mode=figures,round-precision=3]{0.00637422} $\pm$ \num[round-mode=figures,round-precision=3]{0.00823961} \\ 
  $WtH$   & \num[round-mode=figures,round-precision=3]{0.0261264} $\pm$ \num[round-mode=figures,round-precision=3]{0.00997852} & \num[round-mode=figures,round-precision=3]{0.000808187} $\pm$ \num[round-mode=figures,round-precision=3]{0.00273078} & \num[round-mode=figures,round-precision=3]{0.000302862} $\pm$ \num[round-mode=figures,round-precision=3]{2.95736e-05} & \num[round-mode=figures,round-precision=3]{0.000302862} $\pm$ \num[round-mode=figures,round-precision=3]{2.95736e-05} & \num[round-mode=figures,round-precision=3]{0.0200644} $\pm$ \num[round-mode=figures,round-precision=3]{0.0131099} & \num[round-mode=figures,round-precision=3]{0.00251338} $\pm$ \num[round-mode=figures,round-precision=3]{0.0136003} & \num[round-mode=figures,round-precision=3]{0.000302862} $\pm$ \num[round-mode=figures,round-precision=3]{2.95736e-05} & \num[round-mode=figures,round-precision=3]{0.00208648} $\pm$ \num[round-mode=figures,round-precision=3]{0.011255} & \num[round-mode=figures,round-precision=3]{0.00349508} $\pm$ \num[round-mode=figures,round-precision=3]{0.0091464} \\ 
  $t\bar{t}H$ (SM)   & \num[round-mode=figures,round-precision=3]{0.17707} $\pm$ \num[round-mode=figures,round-precision=3]{0.0544537} & \num[round-mode=figures,round-precision=3]{0.0737209} $\pm$ \num[round-mode=figures,round-precision=3]{0.0228372} & \num[round-mode=figures,round-precision=3]{0.0427629} $\pm$ \num[round-mode=figures,round-precision=3]{0.0429804} & \num[round-mode=figures,round-precision=3]{0.0848586} $\pm$ \num[round-mode=figures,round-precision=3]{0.0242696} & \num[round-mode=figures,round-precision=3]{0.477672} $\pm$ \num[round-mode=figures,round-precision=3]{0.0574756} & \num[round-mode=figures,round-precision=3]{0.0502371} $\pm$ \num[round-mode=figures,round-precision=3]{0.0127665} & \num[round-mode=figures,round-precision=3]{0.0321943} $\pm$ \num[round-mode=figures,round-precision=3]{0.0282768} & \num[round-mode=figures,round-precision=3]{0.0350634} $\pm$ \num[round-mode=figures,round-precision=3]{0.040962} & \num[round-mode=figures,round-precision=3]{0.0894654} $\pm$ \num[round-mode=figures,round-precision=3]{0.050545} \\ 
\hline 
  Total  & \num[round-mode=figures,round-precision=3]{2615.7} $\pm$ \num[round-mode=figures,round-precision=3]{363.733} & \num[round-mode=figures,round-precision=3]{193.584} $\pm$ \num[round-mode=figures,round-precision=3]{12.9399} & \num[round-mode=figures,round-precision=3]{78.5024} $\pm$ \num[round-mode=figures,round-precision=3]{8.45702} & \num[round-mode=figures,round-precision=3]{56.1389} $\pm$ \num[round-mode=figures,round-precision=3]{6.48792} & \num[round-mode=figures,round-precision=3]{158.311} $\pm$ \num[round-mode=figures,round-precision=3]{407.731} & \num[round-mode=figures,round-precision=3]{14.9387} $\pm$ \num[round-mode=figures,round-precision=3]{3.31296} & \num[round-mode=figures,round-precision=3]{5.93483} $\pm$ \num[round-mode=figures,round-precision=3]{1.50872} & \num[round-mode=figures,round-precision=3]{4.99329} $\pm$ \num[round-mode=figures,round-precision=3]{1.18476} & \num[round-mode=figures,round-precision=3]{10.1404} $\pm$ \num[round-mode=figures,round-precision=3]{3.96314} \\ 
\hline 
  Data   & 2605 & 187 & 67 & 70 & 109 & 18 & 4 & 6 & 17 \\ 
\hline 
\end{tabular} 



%        }
%\end{center} 
%\end{table} 
%\end{landscape}
%\newpage

%\begin{figure}[H]
%    \centering
%    \includegraphics[width=0.9\linewidth]{1j/NormFactors.png}
%    \caption{Normalization factors for WZ+b/c/light}
%    \label{fig:yields}
%\end{figure}

%\begin{table}[H]
%    \centering
%    \begin{tabular}{c}
%         $\mu_{WZ + b} = 1.26^{+0.52}_{-0.48}^{+0.37}_{-0.31}$ \\
%         $\mu_{WZ+c} = 1.20 \pm 0.22 \pm 0.14 $\\ 
%         $\mu_{WZ + l} = 1.04 \pm 0.04 \pm 0.03 $\\\\
%        WZ + $b$: $124.07^{+51.2}_{-43.32}(stat)^{+36.43}_{-35.45}(s%ys) = 124.07^{+62.84}_{-55.98}$ fb \\\\
%        WZ + c: $726.56 \pm 133.2(stat) \pm 84.77(sys) = 726.56 %\pm 157.89$ fb \\\\
%        WZ + hf: %$850.63^{+118.7}_{-119.63}(stat)^{+75.31}_{-75.35}(sys) = %850.63^{+140.57}_{-141.38}$ fb \\\\
%    \end{tabular}
%    \caption{Caption}
%    \label{tab:systematics}
%\end{table}


%\begin{figure}[H]
%    \centering
%    \includegraphics[width=0.8\linewidth]{1j/SignalRegions.png}
%    \caption{Significance of the fit regions.}
%    \label{fig:pie_chart}
%\end{figure}

\subsection{2-jet Fit Results}

\textbf{The results of the fit are currently blinded.} 

Pre-fit yields in each of the 2-jet fit regions are shown in Figure \ref{tab:prefitYield2j}.

\begin{table}[H]
\begin{adjustwidth}{-3em}{-3em}
\documentclass[10pt]{article}
\usepackage{siunitx}
\sisetup{separate-uncertainty,table-format=6.3(6)}  % hint: modify table-format to best fit your tables
\usepackage[margin=0.1in,landscape,papersize={210mm,350mm}]{geometry}
\begin{document}
\begin{table}[htbp]
\begin{center}
\begin{tabular}{|l|S|S|S|S|S|S|}
\hline 
 & {2j, <85% WP} & {2j, 77%-85% WP} & {2j, 70%-77% WP} & {2j, 60-70 WP} & {2j, >60 WP} & {tZ CR 2j}\\
\hline 
  $WZ + b$   & 13.1121 \pm 1.64912 & 6.72633 \pm 0.469748 & 5.79022 \pm 0.422737 & 8.04309 \pm 0.560004 & 31.0916 \pm 2.12415 & 13.5945 \pm 0.92489 \\ 
  $WZ + b$   & 0.648506 \pm 0.0233492 & 0.379656 \pm 0.012239 & 0.366086 \pm 0.0118504 & 0.515992 \pm 0.0165608 & 2.14359 \pm 0.069585 & 1.67052 \pm 0.0536194 \\ 
  $WZ + c$   & 260.427 \pm 19.5987 & 77.1446 \pm 6.02341 & 41.0337 \pm 3.42487 & 26.8615 \pm 2.64804 & 10.8721 \pm 1.56289 & 4.82243 \pm 0.623711 \\ 
  $WZ + c$   & 12.6431 \pm 0.406141 & 3.81658 \pm 0.122327 & 1.9428 \pm 0.0625431 & 1.23046 \pm 0.0398104 & 0.392877 \pm 0.0127938 & 0.191963 \pm 0.00745947 \\ 
  $WZ + l$   & 1863.26 \pm 146.844 & 90.0772 \pm 13.879 & 17.593 \pm 3.1674 & 5.76845 \pm 1.40941 & 1.39342 \pm 0.385532 & 0.25449 \pm 0.151956 \\ 
  $WZ + l$   & 64.3406 \pm 2.12557 & 3.70303 \pm 0.178364 & 0.707517 \pm 0.0606722 & 0.283953 \pm 0.037058 & 0.0574868 \pm 0.0154251 & 0.0251147 \pm 0.0108612 \\ 
  Other VV   & 0.118294 \pm 0.113356 & 0.103623 \pm 0.0785318 & 0.0040485 \pm 0.00402898 & 0.0498154 \pm 0.0488896 & 0.130107 \pm 0.121055 & 0.076211 \pm 0.0675379 \\ 
  Other VV   & 1.00665 \pm 0.53115 & 0.295662 \pm 0.180558 & 0.159605 \pm 0.0882379 & 0.0504455 \pm 0.0504839 & 0.101917 \pm 0.0946648 & 0.01154 \pm 0.00942916 \\ 
  Other VV   & 6.62773 \pm 0.632648 & 0.349085 \pm 0.0737831 & 0.0924561 \pm 0.0341413 & 0.011578 \pm 0.00693331 & 0.00160229 \pm 0.00227466 & 0.00267678 \pm 0.0028377 \\ 
  ZZ   & 1.41467 \pm nan & 0.913732 \pm nan & 0.741413 \pm nan & 1.66039 \pm nan & 3.92695 \pm nan & 2.41578 \pm nan \\ 
  ZZ   & 9.92438 \pm nan & 2.78016 \pm nan & 1.70464 \pm nan & 1.19773 \pm nan & 0.47733 \pm nan & 0.214587 \pm nan \\ 
  ZZ   & 135.413 \pm nan & 11.3293 \pm nan & 2.71159 \pm nan & 1.03942 \pm nan & 0.133004 \pm nan & 0.0902197 \pm nan \\ 
  $t\bar{t}W$   & 0.846634 \pm 0.177117 & 0.427294 \pm 0.109506 & 0.536693 \pm 0.11567 & 0.735696 \pm 0.149367 & 4.246 \pm 0.623863 & 3.88586 \pm 0.593088 \\ 
  $t\bar{t}Z$   & 14.6888 \pm 2.23462 & 5.564 \pm 0.846518 & 4.4493 \pm 0.711863 & 6.5275 \pm 1.04821 & 25.4272 \pm 4.01392 & 21.9396 \pm 3.38298 \\ 
  $t\bar{t}ll low mass$   & 0.0048381 \pm 0.0206088 & 0.00334186 \pm 0.0131255 & 0.0111711 \pm 0.0172486 & 0.0534208 \pm 0.0237593 & 0.0989173 \pm 0.0386884 & 0.145079 \pm 0.0425929 \\ 
  rare Top   & 0.139637 \pm 0.0440634 & 0.0729123 \pm 0.0329065 & 0.029893 \pm 0.0203857 & 0.0895686 \pm 0.0338315 & 0.370734 \pm 0.0717209 & 0.601697 \pm 0.0923253 \\ 
  Single top   & 1e-06 \pm 1.00051e-06 & 1e-06 \pm 1.00051e-06 & 1e-06 \pm 1.00051e-06 & 1e-06 \pm 1.00051e-06 & 1e-06 \pm 1.00051e-06 & 1e-06 \pm 1.00051e-06 \\ 
  Single top t   & 1e-06 \pm 1.00051e-06 & 1e-06 \pm 1.00051e-06 & 1e-06 \pm 1.00051e-06 & 1e-06 \pm 1.00051e-06 & 1e-06 \pm 1.00051e-06 & 1e-06 \pm 1.00051e-06 \\ 
  Single top s   & 1e-06 \pm 1.00051e-06 & 1e-06 \pm 1.00051e-06 & 1e-06 \pm 1.00051e-06 & 1e-06 \pm 1.00051e-06 & 1e-06 \pm 1.00051e-06 & 1e-06 \pm 1.00051e-06 \\ 
  Three top   & 0.000795654 \pm 0.000903456 & 1e-06 \pm 1.11849e-06 & 0.00053819 \pm 0.000807836 & 1e-06 \pm 1.11849e-06 & 0.000291778 \pm 0.000438077 & 0.000524748 \pm 0.000591065 \\ 
  Four top   & 1e-06 \pm 1.00051e-06 & 1e-06 \pm 1.00051e-06 & 1e-06 \pm 1.00051e-06 & 0.00169972 \pm 0.00241044 & 0.00419253 \pm 0.0042639 & 0.0064058 \pm 0.00540073 \\ 
  $t\bar{t}WW$   & 0.0438161 \pm 0.030403 & 0.0157705 \pm 0.0159263 & 0.00990743 \pm 0.0142084 & 1e-06 \pm 1.00051e-06 & 0.0349213 \pm 0.0277333 & 0.00981125 \pm 0.0139011 \\ 
  $Z+\text{jets}$   & 110.035 \pm 22.9657 & 9.63548 \pm 2.01155 & 2.04667 \pm 0.497545 & 1.55601 \pm 0.355075 & 5.07271 \pm 1.08496 & 1.54024 \pm 0.323615 \\ 
  low mass $Z+\text{jets}$   & 0.427955 \pm 0.462239 & 1e-06 \pm 1.02153e-06 & 1e-06 \pm 1.02153e-06 & 0.0645592 \pm 0.0923138 & 1e-06 \pm 1.02153e-06 & 1e-06 \pm 1.02153e-06 \\ 
  $W+\text{jets}$   & 1e-06 \pm 1.00051e-06 & 1e-06 \pm 1.00051e-06 & 1e-06 \pm 1.00051e-06 & 1e-06 \pm 1.00051e-06 & 1e-06 \pm 1.00051e-06 & 1e-06 \pm 1.00051e-06 \\ 
  $V+\gamma$   & 25.3364 \pm 18.7754 & 0.528283 \pm 0.22681 & 0.14496 \pm 0.134163 & 0.134186 \pm 0.139545 & 1e-06 \pm 0.0185128 & 0.0477795 \pm 0.0676476 \\ 
  $tZ$   & 15.8871 \pm 2.87367 & 6.88883 \pm 1.31159 & 5.12346 \pm 0.988274 & 8.00082 \pm 1.50967 & 18.7182 \pm 3.21327 & 36.3981 \pm 6.08448 \\ 
  $tW$   & 0.887016 \pm 0.680988 & 0.148044 \pm 0.267482 & 1e-06 \pm 0.152686 & 1e-06 \pm 1.73235e-06 & 0.775575 \pm 0.579287 & 0.190468 \pm 0.210139 \\ 
  $WtZ$   & 4.87787 \pm 2.47722 & 1.5333 \pm 0.800178 & 1.10464 \pm 0.586653 & 1.28169 \pm 0.676501 & 4.63867 \pm 2.36029 & 3.29961 \pm 1.68465 \\ 
  $VVV$   & 7.4218 \pm 0.29773 & 1.00142 \pm 0.0658458 & 0.361227 \pm 0.0348817 & 0.186629 \pm 0.0266548 & 0.134407 \pm 0.0286498 & 0.0385356 \pm 0.0127406 \\ 
  $VH$   & 19.5288 \pm 4.22681 & 2.75762 \pm 1.64058 & 0.683558 \pm 0.685587 & 0.0956421 \pm 0.157776 & 1e-06 \pm 1.00051e-06 & 1e-06 \pm 1.00051e-06 \\ 
  $t\bar{t}$   & 0.693775 \pm 0.426937 & 0.0907891 \pm 0.0896705 & 0.048401 \pm 0.059168 & 0.151307 \pm 0.133312 & 0.75843 \pm 0.457278 & 2.28894 \pm 1.24499 \\ 
  $t\bar{t}$   & 6.82241 \pm 1.00648 & 2.36225 \pm 0.495812 & 1.74535 \pm 0.403559 & 3.29853 \pm 0.602378 & 8.42598 \pm 1.17436 & 13.6197 \pm 1.7419 \\ 
  $t\bar{t}H$   & 0.401927 \pm 0.0509697 & 0.19376 \pm 0.0275383 & 0.15542 \pm 0.021747 & 0.229821 \pm 0.0301329 & 0.944683 \pm 0.106107 & 1.02694 \pm 0.115019 \\ 
\hline 
  Total  & 2576.99 \pm nan & 228.842 \pm nan & 89.2983 \pm nan & 69.1199 \pm nan & 120.373 \pm nan & 108.409 \pm nan \\ 
\hline 
\end{tabular} 
\caption{Yields of the analysis} 
\end{center} 
\end{table} 
\end{document}

\label{tab:prefitYield2j}
\caption{Pre-fit yields in each of the 2-jet fit regions.}                                     
\end{adjustwidth}
\end{table}

\begin{figure}[H]
  \center                                                                                                                    
  \includegraphics[width=.9\linewidth]{2j/Summary.png}
  \caption{Pre-fit summary of the 2-jet fit regions.}                                                                        
  \label{fig:prefitSummary_2j}
\end{figure}

%\begin{figure}[H]
%    \center
%    \includegraphics[width=.35\linewidth]{2j/not_85_2j.png}%                                                             
%    \includegraphics[width=.35\linewidth]{2j/WP_2b_77_85.png}%                                                
%    \includegraphics[width=.35\linewidth]{2j/WP_2b_70_77.png}\\
%    \includegraphics[width=.35\linewidth]{2j/WP_2b_60_70.png}%                                                       
%    \includegraphics[width=.35\linewidth]{2j/WP_2b_60.png}%                                                     
%    \includegraphics[width=.35\linewidth]{2j/tZ_CR_2j.png}\\
%    \caption{Data/MC yields in each of the regions in the 2-jet fit before the fit has been performed.}
%    \label{fig:prefit_2j}
%\end{figure}

The post-fit yields in each region are summarized in Figure \ref{tab:postfitYield2j}.

\begin{table}[H]
\begin{adjustwidth}{-3em}{-3em}
\begin{tabular}{|l|c|c|c|c|c|c|}
\hline 
 & {2j, $<$85\% WP} & {2j, 77\%-85\% WP} & {2j, 70\%-77\% WP} & {2j, 60\%-70\% WP} & {2j, $>$60\% WP} & {tZ CR 2j}\\
\hline 
  $WZ + b$   & 13 $\pm$ 6 & 6.7 $\pm$ 2.9 & 5.8 $\pm$ 2.5 & 8.0 $\pm$ 3.5 & 31 $\pm$ 13 & 14 $\pm$ 5 \\ 
  $WZ + c$   & 260 $\pm$ 60 & 77 $\pm$ 15 & 41 $\pm$ 8 & 26 $\pm$ 5 & 10.9 $\pm$ 2.4 & 4.8 $\pm$ 1.1 \\ 
  $WZ + l$   & 1860 $\pm$ 90 & 90 $\pm$ 12 & 17.6 $\pm$ 2.8 & 5.8 $\pm$ 1.3 & 1.4 $\pm$ 0.4 & 0.3 $\pm$ 0.2 \\ 
  $WZ + b - 1j$   & 3.4 $\pm$ 0.6 & 1.52 $\pm$ 0.35 & 1.58 $\pm$ 0.23 & 1.95 $\pm$ 0.39 & 6.7 $\pm$ 1.1 & 1.9 $\pm$ 0.6 \\
  $WZ + c - 1j$   & 56 $\pm$ 14 & 17.6 $\pm$ 4.0 & 8.6 $\pm$ 2.2 & 6.3 $\pm$ 1.8 & 3.0 $\pm$ 0.9 & 0.7 $\pm$ 0.2 \\
  $WZ + l - 1j$   & 427 $\pm$ 120 & 24 $\pm$ 7 & 4.7 $\pm$ 2.3 & 1.6 $\pm$ 0.7 & 0.3 $\pm$ 0.2 & 0.01 $\pm$ 0.01 \\
  $WZ - Other$   & 129 $\pm$ 29 & 6.1 $\pm$ 4.6 & 1.2 $\pm$ 0.3 & 0.3 $\pm$ 0.2 & 2.9 $\pm$ 0.5 & 3.6 $\pm$ 0.6 \\
  Other VV   & 7.6 $\pm$ 0.6 & 0.3 $\pm$ 0.3 & 0.3 $\pm$ 0.1 & 0.1 $\pm$ 0.06 & 0.03 $\pm$ 0.02 & 0.1 $\pm$ 0.1 \\ 
  ZZ  & 145 $\pm$ 30 & 11.3 $\pm$ 4.4 & 2.7 $\pm$ 1.6 & 1.0 $\pm$ 0.3 & 4.0 $\pm$ 0.1 & 2.4 $\pm$ 0.1 \\ 
  $t\bar{t}W$   & 0.8 $\pm$ 0.2 & 0.4 $\pm$ 0.1 & 0.54 $\pm$ 0.12 & 0.74 $\pm$ 0.15 & 4.3 $\pm$ 0.6 & 3.9 $\pm$ 0.6 \\ 
  $t\bar{t}Z$   & 14.7 $\pm$ 2.2 & 5.6 $\pm$ 0.8 & 4.5 $\pm$ 0.7 & 6.5 $\pm$ 1.0 & 25.4 $\pm$ 3.9 & 21.9 $\pm$ 3.3 \\ 
  rare Top   & 0.14 $\pm$ 0.04 & 0.07 $\pm$ 0.03 & 0.03 $\pm$ 0.02 & 0.09 $\pm$ 0.03 & 0.4 $\pm$ 0.1 & 0.6 $\pm$ 0.1 \\ 
  $t\bar{t}WW$   & 0.04 $\pm$ 0.03 & 0.02 $\pm$ 0.02 & 0.01 $\pm$ 0.01 & 0.01 $\pm$ 0.01 & 0.03 $\pm$ 0.03 & 0.01 $\pm$ 0.01 \\ 
  $Z+\text{jets}$   & 110 $\pm$ 23 & 9.6 $\pm$ 2.0 & 2.1 $\pm$ 0.5 & 1.6 $\pm$ 0.4 & 5.1 $\pm$ 1.1 & 1.5 $\pm$ 0.3 \\ 
  $W+\text{jets}$   & 0.0 $\pm$ 0.0 & 0.0 $\pm$ 0.0 & 0.0 $\pm$ 0.0 & 0.0 $\pm$ 0.0 & 0.0 $\pm$ 0.0 & 0.0 $\pm$ 0.0 \\ 
  $V+\gamma$   & 25 $\pm$ 19 & 0.5 $\pm$ 0.2 & 0.1 $\pm$ 0.1 & 0.13 $\pm$ 0.14 & 0.0 $\pm$ 0.02 & 0.05 $\pm$ 0.07 \\ 
  $tZ$   & 15.9 $\pm$ 2.7 & 6.9 $\pm$ 1.2 & 5.1 $\pm$ 0.9 & 8.0 $\pm$ 1.4 & 18.7 $\pm$ 3.0 & 36 $\pm$ 6 \\ 
  $tW$   & 0.1 $\pm$ 0.7 & 0.2 $\pm$ 0.3 & 0.0 $\pm$ 0.1 & 0.0 $\pm$ 0.0 & 0.8 $\pm$ 0.6 & 0.2 $\pm$ 0.2 \\ 
  $WtZ$   & 4.9 $\pm$ 2.5 & 1.5 $\pm$ 0.8 & 1.1 $\pm$ 0.6 & 1.3 $\pm$ 0.7 & 4.6 $\pm$ 2.3 & 3.3 $\pm$ 1.7 \\ 
  $VVV$   & 7.4 $\pm$ 0.3 & 1.0 $\pm$ 0.1 & 0.36 $\pm$ 0.03 & 0.19 $\pm$ 0.03 & 0.13 $\pm$ 0.03 & 0.04 $\pm$ 0.01 \\ 
  $VH$   & 19 $\pm$ 4 & 2.8 $\pm$ 1.6 & 0.7 $\pm$ 0.7 & 0.1 $\pm$ 0.2 & 0.0 $\pm$ 0.0 & 0.0 $\pm$ 0.0 \\ 
  $t\bar{t}$   & 6.8 $\pm$ 1.0 & 2.4 $\pm$ 0.5 & 1.8 $\pm$ 0.4 & 3.3 $\pm$ 0.6 & 8.4 $\pm$ 1.2 & 13.6 $\pm$ 1.7 \\ 
  $t\bar{t}H$   & 0.40 $\pm$ 0.05 & 0.19 $\pm$ 0.03 & 0.16 $\pm$ 0.02 & 0.23 $\pm$ 0.03 & 0.94 $\pm$ 0.11 & 1.03 $\pm$ 0.11 \\ 
\hline 
  Total  & 2580 $\pm$ 60 & 229 $\pm$ 11 & 89 $\pm$ 6 & 69.1 $\pm$ 4.1 & 120 $\pm$ 10 & 108 $\pm$ 6 \\ 
\hline 
\end{tabular} 


\label{tab:postfitYield2j}
\caption{Post-fit yields in each of the 2-jet fit regions.}                                                         
\end{adjustwidth}
\end{table}

%\begin{figure}[H]
%    \center
%    \includegraphics[width=.35\linewidth]{2j/not_85_2j_postFit.png}%                                                      
%    \includegraphics[width=.35\linewidth]{2j/WP_2b_77_85_postFit.png}%                                               
%    \includegraphics[width=.35\linewidth]{2j/WP_2b_70_77_postFit.png}\\
%    \includegraphics[width=.35\linewidth]{2j/WP_2b_60_70_postFit.png}%                                             
%    \includegraphics[width=.35\linewidth]{2j/WP_2b_60_postFit.png}%                                                  
%    \includegraphics[width=.35\linewidth]{2j/tZ_CR_2j_postFit.png}\\
%    \caption{Data/MC results in each of the regions in the 2-jet fit after the fit has been performed.}
%    \label{fig:fit_regions_2j}
%\end{figure}

A post-fit summary of the fitted regions is shown in Figure \ref{fig:fit_results_2j}: 

\begin{figure}[H]
    \center
    \includegraphics[width=.9\linewidth]{2j/Summary_postFit.png}
    \caption{Post-fit summary of the fit over 2-jet regions.}
    \label{fig:fit_results_2j}
\end{figure}

%As described in section \ref{sec:sys}, there are 226 systematic uncertainties that are considered as NPs in the fit. These NPs are constrained by Gaussian or log-normal probability density functions. The latter are used for normalisation factors to ensure that they are always positive. The expected numbers
%of signal and background events are functions of the likelihood. The prior for each NP is added as a penalty term, decreasing the likelihood as it is shifted away from its nominal value. 

%The impact of each systematic uncertainty is calculated by performing the fit with the parameter of interest held fixed, varied from its fitted value by its uncertainty, and calculating $\delta\mu$ relative to the baseline fit.  
The same set of systematic uncertainties consider for the 1-jet fit are included in the 2-jet fit as well. The impact of the most significant systematic uncertainties is summarized in Table \ref{tab:systematics_2j}. 

\begin{table}[H]
    \centering
    \begin{tabular}{l|cc}
        \hline\hline
        Uncertainty Source & \multicolumn{2}{c}{$\Delta \mu$ }  \\
        \hline
        WZ + c cross-section & -0.1 & 0.14 \\
        Luminosity & -0.11 & 0.12 \\
        tZ cross-section & -0.11 & 0.11 \\
        Jet Energy Scale & -0.11 & 0.11 \\
        ttZ cross-section - QCD scale & -0.08 & 0.09 \\
        WZ + l cross-section & 0.08 & -0.07 \\
        WtZ cross-section & -0.07 & 0.07 \\
        Flavor tagging  & 0.05 & 0.05 \\
        Other VV + b cross-section & -0.05 & 0.05 \\
        Jet Energy Resolution & -0.04 & 0.04 \\
        \hline
        Total & 0.29 & 0.31 \\

        %WZ + charm cross-section & -0.1966 & 0.2171 \\
        %tZ cross-section & -0.1521 & 0.1518 \\
        %WZ + light cross-section & 0.1485 & -0.1411 \\
        %Other VV + b cross-sction & -0.1115 & 0.1163 \\
        %Flavor Tagging & 0.0955 & 0.0957 \\
        %Jet Energy Scale & 0.0613 & 0.081 \\
        %$t\bar{t}$ cross-section & -0.0662 & 0.0654 \\
        %Luminosity & -0.0609 & 0.0655 \\
        %Z + jets cross-section & -0.0284 & 0.0284 \\
        %Other VV + charm cross-sction & 0.0207 & -0.0202 \\
        %Muon Trigger Scale Factor & 0.019 & 0.0209 \\
        %\hline
        %Total Systematic Uncertainty & 0.3511 & 0.3679 \\
        \hline\hline
    \end{tabular}
    \caption{Summary of the most significant sources of systematic uncertainty on the measurement of $WZ+b$ 2-jet events.}
    \label{tab:systematics_2j}
\end{table}

The ranking and impact of those nuisance parameters with the largest contribution to the overall uncertainty is shown in Figure \ref{fig:ranking_2j}.

\begin{figure}[H]
    \centering
    \includegraphics[width=0.7\linewidth]{2j/Ranking.png}
    \caption{Impact of systematic uncertainties on the signal-strength of $WZ$ + b in 2-jet events.}
    \label{fig:ranking_2j}
\end{figure}

The large impact of the Jet Energy Scale and Jet Flavor Tagging is unsurprising, as the shape of the fit regions depends heavily on the modeling of the jets. The other major sources of uncertainty come from background modelling and cross-section uncertainty. The pie charts in Figure \ref{fig:pie_chart_2j} show that for the modelling uncertainties that contribute most correspond to the most significant backgrounds. %The pileup-reweighting and luminosity play a significant role as well, because, as shown in Figure \ref{fig:pie_chart}, the signal purity is relatively small in each of the fit regions. This means that a small scaling in the background contribution resulting from a change in luminosity or pileup corresponds to a significant change in the best fit signal contribution. 

\begin{figure}[H]
    \centering
    \includegraphics[width=0.7\linewidth]{2j/PieChart_postFit.png}
    \caption{Post-fit background composition of the 2-jet fit regions.}
    \label{fig:pie_chart_2j}
\end{figure}

The correlations between these nuisance parameters are summarized in Figure \ref{fig:corr_mat_2j}. 

\begin{figure}[H]
    \centering
    \includegraphics[width=1.0\linewidth]{2j/CorrMatrix.png}
    \caption{Correlations between nuisance parameters in the 2-jet fit}
    \label{fig:corr_mat_2j}
\end{figure}

As in the 1-jet case, no significant, unexpected correlations are found between nuisance parameters.


%The negative correlations between $\mu_{WZ+charm}$ and $\mu_{WZ+b}$ and $\mu_{WZ+light}$ are expected: $WZ$ + charm is present in both the $WZ$ + b and $WZ$ + light enriched regions, therefore increasing the fraction of charm requires increasing the fraction of $WZ$ + b and $WZ$ + light. This reasoning also explains the positive correlation between $\mu_{WZ+b}$ and $\mu_{WZ+light}$. 

%Two of the major backgrounds in the region with the highest purity of $WZ$ + b are tZ and Other VV + b, explaining the negative correlations between $\mu_{WZ+b}$ and the tZ cross Section, and the VV + b cross Section.

%The high correlation between the luminosity and $\mu_{WZ+light}$ arises from the fact that the uncertainty on $\mu_{WZ+light}$ is very low (around 4\%). Small changes in luminosity cause a change in the yield of $WZ$ + light that is large compared to its uncertainty, producing a large correlation between these two parameters.

%The other significant correlation present is between ATLAS FTAG L0 and $\mu_{WZ + light}$ and $\mu_{WZ+ charm}$. This shifts the   

%\begin{figure}[H]
%    \centering
%    \includegraphics[width=0.63\linewidth]{2j/SignalRegions.png}
%    \caption{Significance, given by $S/\sqrt{B}$ of the signal in each of the fit regions.}
%    \label{fig:signalRegions}
%\end{figure}

%\newpage
%\begin{landscape}
%\begin{table}[htbp]
%\begin{center}
%\caption{\label{tbl:yields} Background, signal and observed yields in the twelve analysis categories in 36.1 %\ifb\ of data at $\sqrt{s} =$ 13~TeV. Uncertainties in the background estimates due to systematic effects and %t.}
% \resizebox{1.3\textwidth}{!}{
%        \begin{tabular}{|c|c|c|c|c|c|c|c|c|c|}
\hline 
 & $<$85 WP & 1b, 77-85 WP & 1b, 70-77 WP & 1b, 60-70 WP & 1b, $>$60 WP & 2b, 77-85 WP & 2b, 70-77 WP & 2b, 60-70 WP & 2b, $>$60 WP\\
\hline 
  $t\bar{t}W$   & \num[round-mode=figures,round-precision=3]{2.27655} $\pm$ \num[round-mode=figures,round-precision=3]{2.30791} & \num[round-mode=figures,round-precision=3]{1.3571} $\pm$ \num[round-mode=figures,round-precision=3]{1.35835} & \num[round-mode=figures,round-precision=3]{0.941663} $\pm$ \num[round-mode=figures,round-precision=3]{0.969334} & \num[round-mode=figures,round-precision=3]{1.99269} $\pm$ \num[round-mode=figures,round-precision=3]{1.98343} & \num[round-mode=figures,round-precision=3]{10.3804} $\pm$ \num[round-mode=figures,round-precision=3]{10.3282} & \num[round-mode=figures,round-precision=3]{1.50549} $\pm$ \num[round-mode=figures,round-precision=3]{1.50488} & \num[round-mode=figures,round-precision=3]{1.16646} $\pm$ \num[round-mode=figures,round-precision=3]{1.16209} & \num[round-mode=figures,round-precision=3]{1.37188} $\pm$ \num[round-mode=figures,round-precision=3]{1.36002} & \num[round-mode=figures,round-precision=3]{3.85472} $\pm$ \num[round-mode=figures,round-precision=3]{3.818} \\ 
  $t\bar{t}Z$   & \num[round-mode=figures,round-precision=3]{5.59481} $\pm$ \num[round-mode=figures,round-precision=3]{0.952561} & \num[round-mode=figures,round-precision=3]{1.85638} $\pm$ \num[round-mode=figures,round-precision=3]{0.457615} & \num[round-mode=figures,round-precision=3]{1.61823} $\pm$ \num[round-mode=figures,round-precision=3]{0.288506} & \num[round-mode=figures,round-precision=3]{2.22864} $\pm$ \num[round-mode=figures,round-precision=3]{0.409013} & \num[round-mode=figures,round-precision=3]{14.0042} $\pm$ \num[round-mode=figures,round-precision=3]{3.50867} & \num[round-mode=figures,round-precision=3]{1.86738} $\pm$ \num[round-mode=figures,round-precision=3]{0.362445} & \num[round-mode=figures,round-precision=3]{1.14193} $\pm$ \num[round-mode=figures,round-precision=3]{0.396228} & \num[round-mode=figures,round-precision=3]{1.15182} $\pm$ \num[round-mode=figures,round-precision=3]{0.425086} & \num[round-mode=figures,round-precision=3]{2.7131} $\pm$ \num[round-mode=figures,round-precision=3]{0.684245} \\ 
  $VV + b$   & \num[round-mode=figures,round-precision=3]{9.26027} $\pm$ \num[round-mode=figures,round-precision=3]{5.07692} & \num[round-mode=figures,round-precision=3]{5.31041} $\pm$ \num[round-mode=figures,round-precision=3]{2.6598} & \num[round-mode=figures,round-precision=3]{5.63769} $\pm$ \num[round-mode=figures,round-precision=3]{3.04505} & \num[round-mode=figures,round-precision=3]{4.7021} $\pm$ \num[round-mode=figures,round-precision=3]{2.56084} & \num[round-mode=figures,round-precision=3]{30.7357} $\pm$ \num[round-mode=figures,round-precision=3]{14.5166} & \num[round-mode=figures,round-precision=3]{1.89107} $\pm$ \num[round-mode=figures,round-precision=3]{1.01403} & \num[round-mode=figures,round-precision=3]{1.02904} $\pm$ \num[round-mode=figures,round-precision=3]{0.505366} & \num[round-mode=figures,round-precision=3]{1.37878} $\pm$ \num[round-mode=figures,round-precision=3]{0.681346} & \num[round-mode=figures,round-precision=3]{1.23094} $\pm$ \num[round-mode=figures,round-precision=3]{0.856462} \\ 
  $VV + c$   & \num[round-mode=figures,round-precision=3]{228.417} $\pm$ \num[round-mode=figures,round-precision=3]{51.3311} & \num[round-mode=figures,round-precision=3]{62.547} $\pm$ \num[round-mode=figures,round-precision=3]{13.4738} & \num[round-mode=figures,round-precision=3]{31.8} $\pm$ \num[round-mode=figures,round-precision=3]{7.84773} & \num[round-mode=figures,round-precision=3]{23.5735} $\pm$ \num[round-mode=figures,round-precision=3]{5.96178} & \num[round-mode=figures,round-precision=3]{23.9244} $\pm$ \num[round-mode=figures,round-precision=3]{6.99115} & \num[round-mode=figures,round-precision=3]{4.56139} $\pm$ \num[round-mode=figures,round-precision=3]{1.46901} & \num[round-mode=figures,round-precision=3]{1.23133} $\pm$ \num[round-mode=figures,round-precision=3]{1.07988} & \num[round-mode=figures,round-precision=3]{0.116983} $\pm$ \num[round-mode=figures,round-precision=3]{0.164252} & \num[round-mode=figures,round-precision=3]{--} $\pm$ \num[round-mode=figures,round-precision=3]{--} \\ 
  $VV + l$   & \num[round-mode=figures,round-precision=3]{2240.34} $\pm$ \num[round-mode=figures,round-precision=3]{85.2546} & \num[round-mode=figures,round-precision=3]{114.763} $\pm$ \num[round-mode=figures,round-precision=3]{11.1222} & \num[round-mode=figures,round-precision=3]{31.8071} $\pm$ \num[round-mode=figures,round-precision=3]{5.36505} & \num[round-mode=figures,round-precision=3]{16.8257} $\pm$ \num[round-mode=figures,round-precision=3]{3.80662} & \num[round-mode=figures,round-precision=3]{6.47758} $\pm$ \num[round-mode=figures,round-precision=3]{2.25164} & \num[round-mode=figures,round-precision=3]{2.36682} $\pm$ \num[round-mode=figures,round-precision=3]{1.23121} & \num[round-mode=figures,round-precision=3]{0.357081} $\pm$ \num[round-mode=figures,round-precision=3]{0.365519} & \num[round-mode=figures,round-precision=3]{0.0167576} $\pm$ \num[round-mode=figures,round-precision=3]{0.0232322} & \num[round-mode=figures,round-precision=3]{--} $\pm$ \num[round-mode=figures,round-precision=3]{--} \\ 
  $t\bar{t}$   & \num[round-mode=figures,round-precision=3]{27.761} $\pm$ \num[round-mode=figures,round-precision=3]{357.388} & \num[round-mode=figures,round-precision=3]{--} $\pm$ \num[round-mode=figures,round-precision=3]{--} & \num[round-mode=figures,round-precision=3]{--} $\pm$ \num[round-mode=figures,round-precision=3]{--} & \num[round-mode=figures,round-precision=3]{0.000302665} $\pm$ \num[round-mode=figures,round-precision=3]{2.1126e-05} & \num[round-mode=figures,round-precision=3]{36.0985} $\pm$ \num[round-mode=figures,round-precision=3]{407.045} & \num[round-mode=figures,round-precision=3]{0.000302665} $\pm$ \num[round-mode=figures,round-precision=3]{2.1126e-05} & \num[round-mode=figures,round-precision=3]{0.000302665} $\pm$ \num[round-mode=figures,round-precision=3]{2.1126e-05} & \num[round-mode=figures,round-precision=3]{0.000302665} $\pm$ \num[round-mode=figures,round-precision=3]{2.1126e-05} & \num[round-mode=figures,round-precision=3]{0.48602} $\pm$ \num[round-mode=figures,round-precision=3]{1.7566} \\ 
  $t\bar{t}+\gamma$   & \num[round-mode=figures,round-precision=3]{0.480558} $\pm$ \num[round-mode=figures,round-precision=3]{1.05386} & \num[round-mode=figures,round-precision=3]{0.000320031} $\pm$ \num[round-mode=figures,round-precision=3]{0.000152476} & \num[round-mode=figures,round-precision=3]{0.197334} $\pm$ \num[round-mode=figures,round-precision=3]{1.00686} & \num[round-mode=figures,round-precision=3]{0.209157} $\pm$ \num[round-mode=figures,round-precision=3]{1.06389} & \num[round-mode=figures,round-precision=3]{0.887418} $\pm$ \num[round-mode=figures,round-precision=3]{2.72954} & \num[round-mode=figures,round-precision=3]{0.492239} $\pm$ \num[round-mode=figures,round-precision=3]{2.48655} & \num[round-mode=figures,round-precision=3]{0.000320031} $\pm$ \num[round-mode=figures,round-precision=3]{0.000152476} & \num[round-mode=figures,round-precision=3]{0.000320031} $\pm$ \num[round-mode=figures,round-precision=3]{0.000152476} & \num[round-mode=figures,round-precision=3]{0.241039} $\pm$ \num[round-mode=figures,round-precision=3]{1.23072} \\ 
  Single top t-chan   & \num[round-mode=figures,round-precision=3]{0.000302869} $\pm$ \num[round-mode=figures,round-precision=3]{1.86387e-05} & \num[round-mode=figures,round-precision=3]{0.000302869} $\pm$ \num[round-mode=figures,round-precision=3]{1.86387e-05} & \num[round-mode=figures,round-precision=3]{0.000302869} $\pm$ \num[round-mode=figures,round-precision=3]{1.86387e-05} & \num[round-mode=figures,round-precision=3]{0.000302869} $\pm$ \num[round-mode=figures,round-precision=3]{1.86387e-05} & \num[round-mode=figures,round-precision=3]{0.000302869} $\pm$ \num[round-mode=figures,round-precision=3]{1.86387e-05} & \num[round-mode=figures,round-precision=3]{0.000302869} $\pm$ \num[round-mode=figures,round-precision=3]{1.86387e-05} & \num[round-mode=figures,round-precision=3]{0.000302869} $\pm$ \num[round-mode=figures,round-precision=3]{1.86387e-05} & \num[round-mode=figures,round-precision=3]{0.000302869} $\pm$ \num[round-mode=figures,round-precision=3]{1.86387e-05} & \num[round-mode=figures,round-precision=3]{0.000302869} $\pm$ \num[round-mode=figures,round-precision=3]{1.86387e-05} \\ 
  Single top s-chan   & \num[round-mode=figures,round-precision=3]{0.000302869} $\pm$ \num[round-mode=figures,round-precision=3]{1.86387e-05} & \num[round-mode=figures,round-precision=3]{0.000302869} $\pm$ \num[round-mode=figures,round-precision=3]{1.86387e-05} & \num[round-mode=figures,round-precision=3]{0.000302869} $\pm$ \num[round-mode=figures,round-precision=3]{1.86387e-05} & \num[round-mode=figures,round-precision=3]{0.000302869} $\pm$ \num[round-mode=figures,round-precision=3]{1.86387e-05} & \num[round-mode=figures,round-precision=3]{0.000302869} $\pm$ \num[round-mode=figures,round-precision=3]{1.86387e-05} & \num[round-mode=figures,round-precision=3]{0.000302869} $\pm$ \num[round-mode=figures,round-precision=3]{1.86387e-05} & \num[round-mode=figures,round-precision=3]{0.000302869} $\pm$ \num[round-mode=figures,round-precision=3]{1.86387e-05} & \num[round-mode=figures,round-precision=3]{0.000302869} $\pm$ \num[round-mode=figures,round-precision=3]{1.86387e-05} & \num[round-mode=figures,round-precision=3]{0.000302869} $\pm$ \num[round-mode=figures,round-precision=3]{1.86387e-05} \\ 
  $Wt$   & \num[round-mode=figures,round-precision=3]{0.362598} $\pm$ \num[round-mode=figures,round-precision=3]{0.654777} & \num[round-mode=figures,round-precision=3]{0.000302787} $\pm$ \num[round-mode=figures,round-precision=3]{1.86339e-05} & \num[round-mode=figures,round-precision=3]{0.108486} $\pm$ \num[round-mode=figures,round-precision=3]{0.584642} & \num[round-mode=figures,round-precision=3]{0.000302787} $\pm$ \num[round-mode=figures,round-precision=3]{1.86339e-05} & \num[round-mode=figures,round-precision=3]{0.274423} $\pm$ \num[round-mode=figures,round-precision=3]{0.503357} & \num[round-mode=figures,round-precision=3]{0.000302787} $\pm$ \num[round-mode=figures,round-precision=3]{1.86339e-05} & \num[round-mode=figures,round-precision=3]{0.000302787} $\pm$ \num[round-mode=figures,round-precision=3]{1.86339e-05} & \num[round-mode=figures,round-precision=3]{0.000302787} $\pm$ \num[round-mode=figures,round-precision=3]{1.86339e-05} & \num[round-mode=figures,round-precision=3]{0.000302787} $\pm$ \num[round-mode=figures,round-precision=3]{1.86339e-05} \\ 
  Three top   & \num[round-mode=figures,round-precision=3]{0.0003606} $\pm$ \num[round-mode=figures,round-precision=3]{0.0017759} & \num[round-mode=figures,round-precision=3]{0.000302946} $\pm$ \num[round-mode=figures,round-precision=3]{0.000150851} & \num[round-mode=figures,round-precision=3]{0.000302946} $\pm$ \num[round-mode=figures,round-precision=3]{0.000150851} & \num[round-mode=figures,round-precision=3]{0.000302946} $\pm$ \num[round-mode=figures,round-precision=3]{0.000150851} & \num[round-mode=figures,round-precision=3]{0.000612547} $\pm$ \num[round-mode=figures,round-precision=3]{0.00110042} & \num[round-mode=figures,round-precision=3]{0.000302946} $\pm$ \num[round-mode=figures,round-precision=3]{0.000150851} & \num[round-mode=figures,round-precision=3]{0.000302946} $\pm$ \num[round-mode=figures,round-precision=3]{0.000150851} & \num[round-mode=figures,round-precision=3]{0.000302946} $\pm$ \num[round-mode=figures,round-precision=3]{0.000150851} & \num[round-mode=figures,round-precision=3]{0.000773585} $\pm$ \num[round-mode=figures,round-precision=3]{0.00117832} \\ 
  Four top   & \num[round-mode=figures,round-precision=3]{0.000302952} $\pm$ \num[round-mode=figures,round-precision=3]{0.000150853} & \num[round-mode=figures,round-precision=3]{0.000302952} $\pm$ \num[round-mode=figures,round-precision=3]{0.000150853} & \num[round-mode=figures,round-precision=3]{0.000302952} $\pm$ \num[round-mode=figures,round-precision=3]{0.000150853} & \num[round-mode=figures,round-precision=3]{0.000302952} $\pm$ \num[round-mode=figures,round-precision=3]{0.000150853} & \num[round-mode=figures,round-precision=3]{0.000302952} $\pm$ \num[round-mode=figures,round-precision=3]{0.000150853} & \num[round-mode=figures,round-precision=3]{0.000302952} $\pm$ \num[round-mode=figures,round-precision=3]{0.000150853} & \num[round-mode=figures,round-precision=3]{0.000302952} $\pm$ \num[round-mode=figures,round-precision=3]{0.000150853} & \num[round-mode=figures,round-precision=3]{0.000302952} $\pm$ \num[round-mode=figures,round-precision=3]{0.000150853} & \num[round-mode=figures,round-precision=3]{0.00126998} $\pm$ \num[round-mode=figures,round-precision=3]{0.00683589} \\ 
  $t\bar{t}WW$   & \num[round-mode=figures,round-precision=3]{0.00022652} $\pm$ \num[round-mode=figures,round-precision=3]{0.00351054} & \num[round-mode=figures,round-precision=3]{0.000302883} $\pm$ \num[round-mode=figures,round-precision=3]{3.64287e-05} & \num[round-mode=figures,round-precision=3]{0.00440409} $\pm$ \num[round-mode=figures,round-precision=3]{0.0238607} & \num[round-mode=figures,round-precision=3]{0.0129946} $\pm$ \num[round-mode=figures,round-precision=3]{0.0319054} & \num[round-mode=figures,round-precision=3]{0.0062717} $\pm$ \num[round-mode=figures,round-precision=3]{0.0322067} & \num[round-mode=figures,round-precision=3]{0.00440409} $\pm$ \num[round-mode=figures,round-precision=3]{0.0238607} & \num[round-mode=figures,round-precision=3]{0.000302883} $\pm$ \num[round-mode=figures,round-precision=3]{3.64287e-05} & \num[round-mode=figures,round-precision=3]{0.000302883} $\pm$ \num[round-mode=figures,round-precision=3]{3.64287e-05} & \num[round-mode=figures,round-precision=3]{0.00571661} $\pm$ \num[round-mode=figures,round-precision=3]{0.0304709} \\ 
  $V+\text{jets}$   & \num[round-mode=figures,round-precision=3]{15.2819} $\pm$ \num[round-mode=figures,round-precision=3]{20.5512} & \num[round-mode=figures,round-precision=3]{0.373017} $\pm$ \num[round-mode=figures,round-precision=3]{0.281037} & \num[round-mode=figures,round-precision=3]{0.334766} $\pm$ \num[round-mode=figures,round-precision=3]{0.214858} & \num[round-mode=figures,round-precision=3]{0.364992} $\pm$ \num[round-mode=figures,round-precision=3]{0.306013} & \num[round-mode=figures,round-precision=3]{2.61101} $\pm$ \num[round-mode=figures,round-precision=3]{1.33608} & \num[round-mode=figures,round-precision=3]{0.141037} $\pm$ \num[round-mode=figures,round-precision=3]{0.202839} & \num[round-mode=figures,round-precision=3]{0.000291241} $\pm$ \num[round-mode=figures,round-precision=3]{0.000116691} & \num[round-mode=figures,round-precision=3]{0.000291241} $\pm$ \num[round-mode=figures,round-precision=3]{0.000116691} & \num[round-mode=figures,round-precision=3]{0.0279002} $\pm$ \num[round-mode=figures,round-precision=3]{0.152993} \\ 
  low mass $V+\text{jets}$   & \num[round-mode=figures,round-precision=3]{0.000291241} $\pm$ \num[round-mode=figures,round-precision=3]{0.000116691} & \num[round-mode=figures,round-precision=3]{0.000291241} $\pm$ \num[round-mode=figures,round-precision=3]{0.000116691} & \num[round-mode=figures,round-precision=3]{0.000291241} $\pm$ \num[round-mode=figures,round-precision=3]{0.000116691} & \num[round-mode=figures,round-precision=3]{0.000291241} $\pm$ \num[round-mode=figures,round-precision=3]{0.000116691} & \num[round-mode=figures,round-precision=3]{0.000291241} $\pm$ \num[round-mode=figures,round-precision=3]{0.000116691} & \num[round-mode=figures,round-precision=3]{0.000291241} $\pm$ \num[round-mode=figures,round-precision=3]{0.000116691} & \num[round-mode=figures,round-precision=3]{0.000291241} $\pm$ \num[round-mode=figures,round-precision=3]{0.000116691} & \num[round-mode=figures,round-precision=3]{0.000291241} $\pm$ \num[round-mode=figures,round-precision=3]{0.000116691} & \num[round-mode=figures,round-precision=3]{0.000291241} $\pm$ \num[round-mode=figures,round-precision=3]{0.000116691} \\ 
  $V+\text{jets}$   & \num[round-mode=figures,round-precision=3]{0.000302868} $\pm$ \num[round-mode=figures,round-precision=3]{1.10053e-05} & \num[round-mode=figures,round-precision=3]{0.000302868} $\pm$ \num[round-mode=figures,round-precision=3]{1.10053e-05} & \num[round-mode=figures,round-precision=3]{0.000302868} $\pm$ \num[round-mode=figures,round-precision=3]{1.10053e-05} & \num[round-mode=figures,round-precision=3]{0.000302868} $\pm$ \num[round-mode=figures,round-precision=3]{1.10053e-05} & \num[round-mode=figures,round-precision=3]{0.000302868} $\pm$ \num[round-mode=figures,round-precision=3]{1.10053e-05} & \num[round-mode=figures,round-precision=3]{0.000302868} $\pm$ \num[round-mode=figures,round-precision=3]{1.10053e-05} & \num[round-mode=figures,round-precision=3]{0.000302868} $\pm$ \num[round-mode=figures,round-precision=3]{1.10053e-05} & \num[round-mode=figures,round-precision=3]{0.000302868} $\pm$ \num[round-mode=figures,round-precision=3]{1.10053e-05} & \num[round-mode=figures,round-precision=3]{0.000302868} $\pm$ \num[round-mode=figures,round-precision=3]{1.10053e-05} \\ 
  $V+\gamma$   & \num[round-mode=figures,round-precision=3]{40.5736} $\pm$ \num[round-mode=figures,round-precision=3]{14.4393} & \num[round-mode=figures,round-precision=3]{1.10864} $\pm$ \num[round-mode=figures,round-precision=3]{0.529551} & \num[round-mode=figures,round-precision=3]{1.45353} $\pm$ \num[round-mode=figures,round-precision=3]{0.889563} & \num[round-mode=figures,round-precision=3]{0.112777} $\pm$ \num[round-mode=figures,round-precision=3]{0.585685} & \num[round-mode=figures,round-precision=3]{0.356533} $\pm$ \num[round-mode=figures,round-precision=3]{0.273389} & \num[round-mode=figures,round-precision=3]{0.000302868} $\pm$ \num[round-mode=figures,round-precision=3]{1.10053e-05} & \num[round-mode=figures,round-precision=3]{0.000302868} $\pm$ \num[round-mode=figures,round-precision=3]{1.10053e-05} & \num[round-mode=figures,round-precision=3]{0.000302868} $\pm$ \num[round-mode=figures,round-precision=3]{1.10053e-05} & \num[round-mode=figures,round-precision=3]{0.000302868} $\pm$ \num[round-mode=figures,round-precision=3]{1.10053e-05} \\ 
  $tZ$   & \num[round-mode=figures,round-precision=3]{11.5641} $\pm$ \num[round-mode=figures,round-precision=3]{1.10741} & \num[round-mode=figures,round-precision=3]{3.83557} $\pm$ \num[round-mode=figures,round-precision=3]{0.344865} & \num[round-mode=figures,round-precision=3]{3.22506} $\pm$ \num[round-mode=figures,round-precision=3]{0.290839} & \num[round-mode=figures,round-precision=3]{4.41135} $\pm$ \num[round-mode=figures,round-precision=3]{0.403135} & \num[round-mode=figures,round-precision=3]{27.7444} $\pm$ \num[round-mode=figures,round-precision=3]{2.48965} & \num[round-mode=figures,round-precision=3]{1.56275} $\pm$ \num[round-mode=figures,round-precision=3]{0.159151} & \num[round-mode=figures,round-precision=3]{0.745832} $\pm$ \num[round-mode=figures,round-precision=3]{0.0819564} & \num[round-mode=figures,round-precision=3]{0.665967} $\pm$ \num[round-mode=figures,round-precision=3]{0.0689025} & \num[round-mode=figures,round-precision=3]{1.31404} $\pm$ \num[round-mode=figures,round-precision=3]{0.136992} \\ 
  $WtZ$   & \num[round-mode=figures,round-precision=3]{3.34641} $\pm$ \num[round-mode=figures,round-precision=3]{1.63649} & \num[round-mode=figures,round-precision=3]{1.01967} $\pm$ \num[round-mode=figures,round-precision=3]{0.532459} & \num[round-mode=figures,round-precision=3]{0.912386} $\pm$ \num[round-mode=figures,round-precision=3]{0.464022} & \num[round-mode=figures,round-precision=3]{0.7321} $\pm$ \num[round-mode=figures,round-precision=3]{0.373417} & \num[round-mode=figures,round-precision=3]{4.21335} $\pm$ \num[round-mode=figures,round-precision=3]{2.05616} & \num[round-mode=figures,round-precision=3]{0.458102} $\pm$ \num[round-mode=figures,round-precision=3]{0.246527} & \num[round-mode=figures,round-precision=3]{0.221432} $\pm$ \num[round-mode=figures,round-precision=3]{0.13283} & \num[round-mode=figures,round-precision=3]{0.248025} $\pm$ \num[round-mode=figures,round-precision=3]{0.138019} & \num[round-mode=figures,round-precision=3]{0.162435} $\pm$ \num[round-mode=figures,round-precision=3]{0.100658} \\ 
  $VVV$   & \num[round-mode=figures,round-precision=3]{6.22694} $\pm$ \num[round-mode=figures,round-precision=3]{3.09521} & \num[round-mode=figures,round-precision=3]{0.473923} $\pm$ \num[round-mode=figures,round-precision=3]{0.254746} & \num[round-mode=figures,round-precision=3]{0.10713} $\pm$ \num[round-mode=figures,round-precision=3]{0.0617345} & \num[round-mode=figures,round-precision=3]{0.117184} $\pm$ \num[round-mode=figures,round-precision=3]{0.065154} & \num[round-mode=figures,round-precision=3]{0.0592735} $\pm$ \num[round-mode=figures,round-precision=3]{0.03952} & \num[round-mode=figures,round-precision=3]{0.0291393} $\pm$ \num[round-mode=figures,round-precision=3]{0.0270894} & \num[round-mode=figures,round-precision=3]{0.00323562} $\pm$ \num[round-mode=figures,round-precision=3]{0.00432544} & \num[round-mode=figures,round-precision=3]{0.00030518} $\pm$ \num[round-mode=figures,round-precision=3]{0.000151427} & \num[round-mode=figures,round-precision=3]{0.00030518} $\pm$ \num[round-mode=figures,round-precision=3]{0.000151427} \\ 
  $VH$   & \num[round-mode=figures,round-precision=3]{23.9715} $\pm$ \num[round-mode=figures,round-precision=3]{6.0978} & \num[round-mode=figures,round-precision=3]{0.852682} $\pm$ \num[round-mode=figures,round-precision=3]{1.3381} & \num[round-mode=figures,round-precision=3]{0.309145} $\pm$ \num[round-mode=figures,round-precision=3]{1.5312} & \num[round-mode=figures,round-precision=3]{0.764878} $\pm$ \num[round-mode=figures,round-precision=3]{2.34321} & \num[round-mode=figures,round-precision=3]{0.000302868} $\pm$ \num[round-mode=figures,round-precision=3]{1.10053e-05} & \num[round-mode=figures,round-precision=3]{0.000302868} $\pm$ \num[round-mode=figures,round-precision=3]{1.10053e-05} & \num[round-mode=figures,round-precision=3]{0.000302868} $\pm$ \num[round-mode=figures,round-precision=3]{1.10053e-05} & \num[round-mode=figures,round-precision=3]{0.000302868} $\pm$ \num[round-mode=figures,round-precision=3]{1.10053e-05} & \num[round-mode=figures,round-precision=3]{0.000302868} $\pm$ \num[round-mode=figures,round-precision=3]{1.10053e-05} \\ 
  $tHjb$   & \num[round-mode=figures,round-precision=3]{0.0419189} $\pm$ \num[round-mode=figures,round-precision=3]{0.0144183} & \num[round-mode=figures,round-precision=3]{0.0093033} $\pm$ \num[round-mode=figures,round-precision=3]{0.00855833} & \num[round-mode=figures,round-precision=3]{0.000302911} $\pm$ \num[round-mode=figures,round-precision=3]{3.57751e-05} & \num[round-mode=figures,round-precision=3]{0.00324645} $\pm$ \num[round-mode=figures,round-precision=3]{0.00803724} & \num[round-mode=figures,round-precision=3]{0.0372004} $\pm$ \num[round-mode=figures,round-precision=3]{0.0112587} & \num[round-mode=figures,round-precision=3]{0.00311466} $\pm$ \num[round-mode=figures,round-precision=3]{0.00945198} & \num[round-mode=figures,round-precision=3]{0.00207475} $\pm$ \num[round-mode=figures,round-precision=3]{0.00777246} & \num[round-mode=figures,round-precision=3]{0.00169069} $\pm$ \num[round-mode=figures,round-precision=3]{0.00892229} & \num[round-mode=figures,round-precision=3]{0.00637422} $\pm$ \num[round-mode=figures,round-precision=3]{0.00823961} \\ 
  $WtH$   & \num[round-mode=figures,round-precision=3]{0.0261264} $\pm$ \num[round-mode=figures,round-precision=3]{0.00997852} & \num[round-mode=figures,round-precision=3]{0.000808187} $\pm$ \num[round-mode=figures,round-precision=3]{0.00273078} & \num[round-mode=figures,round-precision=3]{0.000302862} $\pm$ \num[round-mode=figures,round-precision=3]{2.95736e-05} & \num[round-mode=figures,round-precision=3]{0.000302862} $\pm$ \num[round-mode=figures,round-precision=3]{2.95736e-05} & \num[round-mode=figures,round-precision=3]{0.0200644} $\pm$ \num[round-mode=figures,round-precision=3]{0.0131099} & \num[round-mode=figures,round-precision=3]{0.00251338} $\pm$ \num[round-mode=figures,round-precision=3]{0.0136003} & \num[round-mode=figures,round-precision=3]{0.000302862} $\pm$ \num[round-mode=figures,round-precision=3]{2.95736e-05} & \num[round-mode=figures,round-precision=3]{0.00208648} $\pm$ \num[round-mode=figures,round-precision=3]{0.011255} & \num[round-mode=figures,round-precision=3]{0.00349508} $\pm$ \num[round-mode=figures,round-precision=3]{0.0091464} \\ 
  $t\bar{t}H$ (SM)   & \num[round-mode=figures,round-precision=3]{0.17707} $\pm$ \num[round-mode=figures,round-precision=3]{0.0544537} & \num[round-mode=figures,round-precision=3]{0.0737209} $\pm$ \num[round-mode=figures,round-precision=3]{0.0228372} & \num[round-mode=figures,round-precision=3]{0.0427629} $\pm$ \num[round-mode=figures,round-precision=3]{0.0429804} & \num[round-mode=figures,round-precision=3]{0.0848586} $\pm$ \num[round-mode=figures,round-precision=3]{0.0242696} & \num[round-mode=figures,round-precision=3]{0.477672} $\pm$ \num[round-mode=figures,round-precision=3]{0.0574756} & \num[round-mode=figures,round-precision=3]{0.0502371} $\pm$ \num[round-mode=figures,round-precision=3]{0.0127665} & \num[round-mode=figures,round-precision=3]{0.0321943} $\pm$ \num[round-mode=figures,round-precision=3]{0.0282768} & \num[round-mode=figures,round-precision=3]{0.0350634} $\pm$ \num[round-mode=figures,round-precision=3]{0.040962} & \num[round-mode=figures,round-precision=3]{0.0894654} $\pm$ \num[round-mode=figures,round-precision=3]{0.050545} \\ 
\hline 
  Total  & \num[round-mode=figures,round-precision=3]{2615.7} $\pm$ \num[round-mode=figures,round-precision=3]{363.733} & \num[round-mode=figures,round-precision=3]{193.584} $\pm$ \num[round-mode=figures,round-precision=3]{12.9399} & \num[round-mode=figures,round-precision=3]{78.5024} $\pm$ \num[round-mode=figures,round-precision=3]{8.45702} & \num[round-mode=figures,round-precision=3]{56.1389} $\pm$ \num[round-mode=figures,round-precision=3]{6.48792} & \num[round-mode=figures,round-precision=3]{158.311} $\pm$ \num[round-mode=figures,round-precision=3]{407.731} & \num[round-mode=figures,round-precision=3]{14.9387} $\pm$ \num[round-mode=figures,round-precision=3]{3.31296} & \num[round-mode=figures,round-precision=3]{5.93483} $\pm$ \num[round-mode=figures,round-precision=3]{1.50872} & \num[round-mode=figures,round-precision=3]{4.99329} $\pm$ \num[round-mode=figures,round-precision=3]{1.18476} & \num[round-mode=figures,round-precision=3]{10.1404} $\pm$ \num[round-mode=figures,round-precision=3]{3.96314} \\ 
\hline 
  Data   & 2605 & 187 & 67 & 70 & 109 & 18 & 4 & 6 & 17 \\ 
\hline 
\end{tabular} 



%        }
%\end{center} 
%\end{table} 
%\end{landscape}
%\newpage

%\begin{figure}[H]
%    \centering
%    \includegraphics[width=0.9\linewidth]{2j/NormFactors.png}
%    \caption{Normalization factors for WZ+b/c/light}
%    \label{fig:yields}
%\end{figure}

%\begin{table}[H]
%    \centering
%    \begin{tabular}{c}
%         $\mu_{WZ + b} = 1.26^{+0.52}_{-0.48}^{+0.37}_{-0.31}$ \\
%         $\mu_{WZ+c} = 1.20 \pm 0.22 \pm 0.14 $\\ 
%         $\mu_{WZ + l} = 1.04 \pm 0.04 \pm 0.03 $\\\\
%        WZ + $b$: $124.07^{+51.2}_{-43.32}(stat)^{+36.43}_{-35.45}(s%ys) = 124.07^{+62.84}_{-55.98}$ fb \\\\
%        WZ + c: $726.56 \pm 133.2(stat) \pm 84.77(sys) = 726.56 %\pm 157.89$ fb \\\\
%        WZ + hf: %$850.63^{+118.7}_{-119.63}(stat)^{+75.31}_{-75.35}(sys) = %850.63^{+140.57}_{-141.38}$ fb \\\\
%    \end{tabular}
%    \caption{Caption}
%    \label{tab:systematics}
%\end{table}


%\begin{figure}[H]
%    \centering
%    \includegraphics[width=0.8\linewidth]{2j/SignalRegions.png}
%    \caption{Significance of the fit regions.}
%    \label{fig:pie_chart}
%\end{figure}


%-------------------------------------------------------------------------------    

%------------------------------------------------------------------------------

\section{Conclusion}
\label{sec:conclusion}

A measurement of $WZ$ + heavy flavor is performed using 140 $fb^{-1}$ of $\sqrt{s} = 13$ TeV proton-proton collision data collected by the ATLAS detector at the LHC. \textbf{This section will be include final results once unblinded.}%A best fit value of X is observed.   

%-------------------------------------------------------------------------------
% If you use biblatex and either biber or bibtex to process the bibliography
% just say \printbibliography here
\printbibliography
% If you want to use the traditional BibTeX you need to use the syntax below.
%\bibliographystyle{bib/bst/atlasBibStyleWithTitle}
%\bibliography{wz_heavy_flavor,bib,ATLAS,bib/CMS,bib/ConfNotes,bib/PubNotes}
%-------------------------------------------------------------------------------

%-------------------------------------------------------------------------------
% Print the list of contributors to the analysis
% The argument gives the fraction of the text width used for the names
%-------------------------------------------------------------------------------
%\clearpage
%\PrintAtlasContribute{0.30}

\iffalse
%-------------------------------------------------------------------------------
\clearpage
\appendix
\part*{Appendices}
\addcontentsline{toc}{part}{Appendices}
%-------------------------------------------------------------------------------

\section{Top Mass Reconstruction}
\label{sec:topMass}

The top quark is reconstructed from the jet, lepton not included in the Z-candidate, and reconstructed neutrino. Since the selection requires exactly one jet in the event, there is only possible b-jet candidate. 

The neutrino from the W decay is expected to be the only source of $E_T^{miss}$. Therefore, the $E_T$ and $\phi$ of the neutrino are taken from the $E_T^{miss}$ measurement. This leaves the z-component of the neutrino momentum, $p_{\nu z}$ as the only unknown. 

This unknown is solved for by taking the combined invariant mass of the lepton and neutrino to give the invariant mass of the $W$ boson:

\begin{center}
   $(p_l + p_{\nu})^2 = m_W^2$ \\ 
\end{center} 

Written in terms of four-vectors, this equation gives:

The reconstructed top mass distribution for tZ and $WZ$ + b can be seen in figure \ref{fig:topMass}:

\begin{figure}[H]
    \centering
    \includegraphics[width=0.7\linewidth]{tZ_bdt/topMass.eps}
    \caption{Reconstructed top mass distributions for tZ and $WZ$ + b, measured in MeV.}
    \label{fig:topMass}
\end{figure}

%--------------------------------------

\fi


\end{document}
